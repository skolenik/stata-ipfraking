\inserttype[st0001]{article}
\author{S. Kolenikov}{%
  Stanislav Kolenikov\\Abt Associates\\stas\_kolenikovs@abtassoc.com
}
\title[Raking survey data: updates]{Updates to the ipfraking ecosystem}
\maketitle

\begin{abstract}
\citet{kolenikov:2014} introduced package \stcmd{ipfraking}
for weight calibration procedures known as iterative proportional fitting,
or raking, of complex survey weights.
This article briefly describes the original package,
and adds updates to the core program, as well as a host of
additional programs that are used to support the process of creating
survey weights in the authors' production code.

\keywords{\inserttag, survey, calibration, weights, raking}
\end{abstract}

\section{Introduction and background}

Large scale social, behavioral and health data are often collected
via complex survey designs that may involve some or all of stratification,
multiple stages of selection and unequal probabilities of selection
\citep{korn:graubard:1995,korn:graubard:1999}.
In an ideal setting, varying probabilities of selection are
accounted for by using the Horvitz-Thompson estimator of the totals
\citep{horvitz:thompson:1952,thompson:1997}, and the remaining
sampling fluctuations can be further ironed out by
post-stratification \citep{holt:smith:1979}.
However, on top of the planned differences in probabilities of obtaining
a response from a sampled unit, non-response is a practical problem
that has been growing more acute over the recent years
\citep{groves:dillman:eltinge:little:2001,pew:2012}.
The analysis weights that are provided along with the public use
microdata by data collecting agencies are designed to account
for unequal probabilities of selection, non-response, and other factors
affecting imbalance between the population and the sample, thus making
the analyses conducted on such microdata generalizable to the target population.

Earlier, I introduced \citep{kolenikov:2014} a Stata package
called \stcmd{ipfraking} that implements
calibration of survey weights to known control totals to ensure
that the resulting weighted data are representative of the population
of interest. The process of calibration is aimed at aligning the sample totals
of the key variables with those known for the population as a whole.
The remainder of this section provides a condensed treatment of estimation
with survey data using calibrated weights; full treatment was provided
in the original paper.

For a given finite population $\mathcal U$ of units indexed $i=1,\ldots,N$,
the interests of survey statisticians often lie in estimating the
population total of a variable $Y$
\begin{equation}
   T[Y] = \sum_{i \in \mathcal{U}} Y_i
   \label{eq:total:pop}
\end{equation}
A sample $\mathcal S$ of $n$ units indexed by $j=1,\ldots,n$
is taken from $\mathcal U$. If the probability to select the
$i$-th unit is known to be $\pi_i$, then
the {\it probability weights}, or {\it design weights}, are given by
the inverse probability of selection:
\begin{equation}
   w_{1i} = \pi_i^{-1}
   \label{eq:prob:weight}
\end{equation}
With these weights, an unbiased
(design-based, non-parametric) estimator
of the total (\ref{eq:total:pop}) is \citep{horvitz:thompson:1952}
\begin{equation}
   t_{1}[y] = \sum_{j \in \mathcal{S}} \frac{y_j}{\pi_j}
   \equiv \sum_{j \in \mathcal{S}} w_{1j} y_j
   \label{eq:total:sample},
\end{equation}
The subindex $1$ indicates that the weights $w_{1i}$ were
used in obtaining this estimator. Probability weights protect
the end user from potentially informative sampling designs, in which
the probabilities of selection are correlated with outcomes, and
the design-based methods generally ensure that inference can be generalized
to the finite population even when the statistical models used
by analysts and researchers are not specified correctly
\citep{pfeff:1993,binder:roberts:2003}.

Often, survey statisticians have auxiliary information on the units
in the frame, and such information can be included it at the sampling stage
to create more efficient designs. Unequal probabilities of selection
are then controlled with probability weights, implemented
as \stcmd{[pw=}{\it exp}\stcmd{]} in Stata (and can be permanently
affixed to the data set with \stcmd{svyset} command).

In many situations, however, usable information is not available beforehand,
and may only appear in the collected data. The census totals of the age and gender
distribution of the population may exist, but age and gender of
the sampled units is unknown until the survey measurement is taken on them.
It is still possible to capitalize on this additional data by
adjusting the weights in such a way that the reweighted data
conforms to these known figures. The procedures to perform these
reweighting steps are generally known as {\it weight calibration}
\citep{deville:sarndal:1992,deville:sarndal:sautory:1993,%
kott:2006,kott:2009,sarndal:2007}.

Suppose there are several (categorical) variables, referred to
as {\it control variables}, that are available for both
the population and the sample
(age groups, race, gender, educational attainment, etc.).
Weight calibration aims at adjusting the margins, or low level interactions,
via an iterative optimization aimed at satisfying
the {\it control totals} for the control variables $\mathbf{x}=(x_1, \ldots, x_p)$:
\begin{equation}
    \sum_{j \in \mathcal{S}} w_{3j} \mathbf{x}_j
    = T [ \mathbf{X}_j  ]
    \label{eq:control:totals}
\end{equation}
where the right hand side is assumed to be known from a census or
a higher quality survey.
\citet{deville:sarndal:1992} framed the problem of finding a suitable
set of weights as that of constrained optimization with the control
equations (\ref{eq:control:totals}) serving as constraints,
and optimization targeted at making the discrepancy between
the design weights $w_{1j}$ and calibrated weights
$w_{3j}$ as close as possible, in a suitable sense.

In package \stcmd{ipfraking} \citep{kolenikov:2014}, I implemented
a popular calibration algorithm, known as \textit{iterated proportional fitting},
or as \textit{raking}, which consists of iterative updating (post-stratification) of
each of the margins. (For an in-depth discussion of distinctions between
raking and post-stratification, see \citet{kolenikov:2016}.)
Since 2014, the continuing code development resulted
in additional features that this update documents.

\section{Updates to \stcmd{ipfraking} program and package}

Below, I provide full syntax, and list the new features in a dedicated section.

\subsection{Syntax of \stcmd{ipfraking}}
\label{subsec:syntax}

\begin{stsyntax}
ipfraking
\optif\
\optin\
\optweight\
,
\underbar{ctot}al({\it matname} [{\it matname \ldots}])
\optional{
\underbar{gen}erate(\newvarname)
replace
double
\underbar{iter}ate(\num)
\underbar{tol}erance(\num)
\underbar{ctrltol}erance(\num)
trace
\underbar{nodiv}ergence
trimhiabs(\num)
trimhirel(\num)
trimloabs(\num)
trimlorel(\num)
trimfrequency(once|sometimes|often)
double
meta
nograph
}
\end{stsyntax}

\hangpara
Note that the weight statement \stcmd{[pw=\varname]} is required, and must contain the initial weights.

\subsubsection{Required options}

\hangpara
\stcmd{\underbar{ctot}al(}{\it matname} \LB{\it matname \ldots}\RB\stcmd{)}
supplies the names of the matrices that contain the control
totals, as well as meta-data about the variables to be used
in calibration.

\begin{sttech}
The row and column names of the control total matrices
(see \pref{matrix rownames}) should be formatted as follows.
\begin{itemize}
    \item \stcmd{rownames}: the name of the control variable
    \item \stcmd{colnames}: the values the control variables takes
    \item \stcmd{coleq}: the name of the variable for which total is computed;
          typically it is identically equal to 1.
\end{itemize}
See examples in Section \ref{sec:examples}.
\end{sttech}

\hangpara
\stcmd{\underbar{gen}erate(\newvarname)}
contains the name of the new variable to contain the raked weights.

\hangpara
\stcmd{replace} indicates that the weight variable supplied in the
\stcmd{[pw=\varname]} expression should be overwritten with the new weights.

One and only one of \stcmd{generate()} or \stcmd{replace} must be specified.

\subsubsection{Linear calibration}

\hangpara
\stcmd{\underbar{lin}ear}
requests linear calibration of weights.

\subsubsection{Options to control convergence}

\hangpara
\stcmd{\underbar{tol}erance(\num)} defines convergence criteria
(the change of weights from one iteration to next). The default is $10^{-6}$.

\hangpara
\stcmd{\underbar{iter}ate(\num)} specifies the maximum number
of iterations. The default is 2000.

\hangpara
\stcmd{\underbar{nodiv}ergence} overrides the check
that the change in weights is greater at the current iteration
than in the previous one, i.e., ignores this termination condition.
It is generally recommended, especially in calibration with simultaneous trimming.

\hangpara
\stcmd{\underbar{ctrltol}erance(\num)} defines the criterion to
assess the accuracy of the control totals. It does not impact
iterations or convergence criteria, but rather only triggers alerts in the output.
The default value is $10^{-6}$.

\hangpara
\stcmd{trace} requests a trace plot to be added.

\subsubsection{Trimming options}
\label{subsubsec:trimming}

\hangpara
\stcmd{trimhiabs(\num)} specifies the upper bound $U$ on the greatest
    value of the raked weights.  The weights that
    exceed this value will be trimmed down, so that
    $w_{3j} \le U$ for every $j\in\mathcal{S}$.

\hangpara
\stcmd{trimhirel(\num)} specifies the upper bound $u$ on the adjustment
    factor over the baseline weight. The weights
    that exceed the baseline times this value will be trimmed down,
    so that $w_{3j} \le u w_{1j}$ for every $j\in\mathcal{S}$.

\hangpara
\stcmd{trimloabs(\num)} specifies the lower bound $L$ on the smallest value
    of the raked weights.  The weights that are smaller than this value will
    be increased, so that $w_{3j} \ge L$ for every $j\in\mathcal{S}$.

\hangpara
\stcmd{trimlorel(\num)} specifies the lower bound $l$ on the adjustment factor
    over the baseline weight.  The weights that are smaller than the baseline
    times this value will be increased, so that
    $w_{3j} \ge l w_{1j}$ for every $j\in\mathcal{S}$.

\hangpara
\stcmd{trimfreqency({\it keyword})} specifies when the trimming operations
    are to be performed. The following keywords are recognized:

\morehang \stcmd{often} means that trimming will be performed
    after each marginal adjustment.

\morehang \stcmd{sometimes} means that trimming will be performed
    after a full set of variables has been used for post-stratification.
    This is the default behavior if any of the numeric trimming
    options above are specified.

\morehang \stcmd{once}
    means that trimming will be performed after the raking process
    is declared to have converged.

The numeric trimming options \stcmd{trimhiabs(\num)}, \stcmd{trimhirel(\num)},
\stcmd{trimloabs(\num)}, \stcmd{trimlorel(\num)} can be specified in any combination,
or entirely omitted to produce untrimmed weights. By default, there is no trimming.

\subsubsection{Miscellaneous options}

\hangpara
\stcmd{double} specifies that the new variable named in \stcmd{generate()}
option should be generated as double type. See \dref{data types}.

\hangpara
\stcmd{meta} puts information taken by \stcmd{ipfraking} as inputs and produced
    throughout the process into characteristics stored with the variable specified in
    \stcmd{generate()} option. See Section \ref{subsec:example:meta}.

\hangpara
\stcmd{nograph} omits the histogram of the calibrated weights, which can be
used to speed up \stcmd{ipfraking} (e.g., in replicate weight production).

\subsection{New features of \stcmd{ipfraking}}

Since the first publication, the following features and options were added.

Reporting of results and errors by \stcmd{ipfraking} was improved in several directions.
\begin{enumerate}
    \item The discrepancy for the worst fitting category is now being reported.
    \item The number of trimmed observations is reported.
    \item If \stcmd{ipfraking} determines that the categories do not match
        in the control totals received from \stcmd{ctotals()} and those found in
        the data, a full listing of categories is provided, and the categories
        not found in one or the other are explicitly shown.
\end{enumerate}

Linear calibration (Case 1 of \citet{deville:sarndal:1992}) is provided with
\stcmd{linear} option. The weights are calculated analytically:
\begin{equation}
    \label{eq:linear:weight}
    w_{j,\rm{lin}} = w_{1j} (1+\mathbf{x}_j'\mathbf{\lambda}).
    \quad
    \mathbf{\lambda} = \Bigl( \sum_{j \in \mathcal{S}} w_{1j} \mathbf{x}_j \mathbf{x}_j' \Bigr)^{-1}
        ( T [ \mathbf{X}_j  ] - t_{1}[y] )
\end{equation}
This works very fast, but has an undesirable artefact of producing negative weights,
as the range of weights is not controlled. (As raking works by multiplying the currents
weights by positive factors, if the input weights are all positive, the output weights
will be positive as well.) Negative weights are not allowed by the official \stcmd{svy} commands
or commands that work with \stcmd{[pweights]}.
In author's experience, running linear weights first,
pulling up the negative and small positive weights (\stcmd{replace weight = 1 if weight <= 1})
and re-raking using the ``proper'' iterative proportional fitting runs faster than
raking from scratch. An example of linearly calibrated weights is given below
in Section \ref{subsec:linear}.

Option \stcmd{meta} saves more information in characteristics of the calibrated
weight variables.

\begin{stlog}
. capture drop rakedwgt3
{\smallskip}
. ipfraking [pw=finalwgt], gen( rakedwgt3 ) ///
>     ctotal( ACS2011_sex_age Census2011_region Census2011_race ) ///
>     trimhiabs(200000) trimloabs(2000) meta
{\smallskip}
 Iteration 1, max rel difference of raked weights = 14.95826
\oom
 Iteration 10, max rel difference of raked weights = 2.254e-07
{\smallskip}
\cnp
   Summary of the weight changes
{\smallskip}
              {\VBAR}    Mean    Std. dev.    Min        Max       CV
\HLI{14}{\PLUS}\HLI{50}
Orig weights  {\VBAR}    11318       7304      2000       79634   .6453
Raked weights {\VBAR}    22055      18908      4033      200000   .8573
Adjust factor {\VBAR}   2.1486               0.9220     18.9828
{\smallskip}
. char li rakedwgt3[]
  rakedwgt3[command]:         [pw=finalwgt], gen( rakedwgt3 ) ctotal( ACS2011_s
> ex_age Ce..
  rakedwgt3[trimloabs]:       trimloabs(2000)
  rakedwgt3[trimhiabs]:       trimhiabs(200000)
  rakedwgt3[trimfrequency]:   sometimes
  rakedwgt3[objfcn]:          2.25435521346e-07
  rakedwgt3[maxctrl]:         3.00266822363e-08
  rakedwgt3[converged]:       1
  rakedwgt3[Census2011_race]: 7.48567503861e-09
  rakedwgt3[Census2011_region]:
                              3.00266822363e-08
  rakedwgt3[ACS2011_sex_age]: 4.13778410340e-09
  rakedwgt3[note1]:           Raking controls used: ACS2011_sex_age Census2011_
> region Ce..
  rakedwgt3[note0]:           1
{\smallskip}
\nullskip
\end{stlog}

The following characteristics are stored with the newly created weight variable
(see \pref{char}).

\begin{tabular}{ll}
    \stcmd{command} & The full command as typed by the user \\
    {\it matrix name} & The relative matrix difference from the corresponding \\
                    & control total, see \dref{functions} \\
    \stcmd{trimhiabs}, \stcmd{trimloabs}, & Corresponding trimming options,
                    if specified \\
    \stcmd{trimhirel}, \stcmd{trimlorel}, & \\
    \stcmd{trimfrequency} & \\
    \stcmd{maxctrl} & the greatest \stcmd{mreldif} between the targets and the achieved \\
                    & weighted totals \\
    \stcmd{objfcn}  & the value of the relative weight change at exit \\
    \stcmd{converged} & whether \stcmd{ipfraking} exited due to convergence (1) \\
                    & vs. due to an increase in the objective function \\
                    & or reaching the limit on the number of iterations (0) \\\
    \stcmd{source}  & weight variable specified as the \stcmd{[pw=]} input \\
    \stcmd{worstvar}& the variable in which the greatest discrepancy between \\
                    & the targets and the achieved weighted totals \\
                    & (\stcmd{maxctrl}) was observed \\
    \stcmd{worstcat}& the category of the \stcmd{worstvar} variable in which the  \\
                    & greatest discrepancy was observed
\end{tabular}

For the control total matrices \num$=1,2,\ldots$, the following
meta-information is stored.

\begin{tabular}{ll}
    \stcmd{mat\num} & the name of the control total matrix \\
    \stcmd{totalof\num}& the multiplier variable (matrix' \stcmd{coleq} \\
    \stcmd{over\num}& the margin associated with the matrix \\
                    & (i.e., the categories represented by the columns)
\end{tabular}

Also, \stcmd{ipfraking} stores the notes regarding the control matrices
used, and which of the margins did not match the control totals, if any.
See \dref{notes}.

\subsection{Utility programs}
\label{subsec:utility}

The original package \stcmd{ipfraking} provided two additional utility programs,
\stcmd{mat2do} and \stcmd{xls2row}. An additional utility program was added
to compute the design effects and margins of error, common tasks associated
with describing survey weights. Specifically, the Transparency Initiative
of the American Association for Public Opinion Research
\citep{aapor:2014:ti:terms}
requires that

\begin{quote}
For probability samples, the estimates of sampling error will be reported, and the discussion will state whether or not the reported margins of sampling error or statistical analyses have been adjusted for the design effect due to weighting, clustering, or other factors.
\end{quote}

\begin{stsyntax}
whatsdeff
{\it weight\_variable}
\optif\
\optin\
,
\optional{
by(\varlist)
}
\end{stsyntax}

The utility program \stcmd{whatsdeff} calculates the apparent design effect due to unequal weighting,
${\rm DEFF_{UWE}}=1 + {CV}^2_w = $ \stcmd{1 + r(Var)/(r(mean))\^2} from \stcmd{summarize} {\it weight\_variable}.
Additionally, it reports the effective sample size, $n/{\rm DEFF_{UWE}}$, and also returns
the margins of error for the sample proportions that estimate the population proportions of
10\% and 50\%.

\begin{stlog}
. webuse nhanes2, clear
{\smallskip}
. whatsdeff finalwgt
{\smallskip}
    Group     {\VBAR}   Min     {\VBAR}   Mean    {\VBAR}   Max     {\VBAR}    CV   {\VBAR}   DEFF  {\VBAR}   N   {\VBAR}  N eff
\HLI{14}{\PLUS}\HLI{11}{\PLUS}\HLI{11}{\PLUS}\HLI{11}{\PLUS}\HLI{9}{\PLUS}\HLI{9}{\PLUS}\HLI{7}{\PLUS}\HLI{8}
      Overall {\VBAR}   2000.00 {\VBAR}  11318.47 {\VBAR}  79634.00 {\VBAR}  0.6453 {\VBAR}  1.4164 {\VBAR} 10351 {\VBAR} 7307.97
{\smallskip}
. return list
{\smallskip}
scalars:
                  r(N) =  10351
              r(MOE10) =  .0068792766212984
              r(MOE50) =  .0114654610354974
       r(Neff_Overall) =  7307.97435325364
       r(DEFF_Overall) =  1.416397964696134
{\smallskip}
\nullskip
\end{stlog}

\subsection{New programs in the package}

Two new programs are added to the package: \stcmd{ipfraking\_report} and \stcmd{wgtcellcollapse},
and are documented in the subsequent sections of this article. The former provides reports on the raked weights,
including summaries of the unweighted data, data with the input weights, and data with the raked weights. 
The latter creates a mostly automated flow of collapsing weighting cells that are too detailed
(and hence have low sample sizes).

\section{Excel reports on raked weights:  \stcmd{ipfraking\_report}}

\begin{stsyntax}
ipfraking\_report
using \textit{filename}
,
raked\_weight(\varname)
\optional{
matrices(\textit{namelist})
by(\varlist)
xls
replace
force
}
\end{stsyntax}

The utility command \stcmd{ipfraking\_report} produces a detailed report
describing the raked weights, and places it into \textit{filename}\stcmd{.dta} file
(or, if \stcmd{xls} option is specified, both \textit{filename}\stcmd{.dta} and \textit{filename}\stcmd{.xls}
files).

Along the way, \stcmd{ipfraking\_report} runs a regression of the log raking ratio $w_{3j}/w_{1j}$
on the calibration variables. This regression is expected to have $R^2$ very close to 1,
and the regression coefficients provide insights regarding which categories received
greater vs. smaller adjustments.

\cnp

\begin{stlog}
. ipfraking_report using rakedwgt3-report, raked_weight(rakedwgt3) replace by(_one)
Margin variable sex_age (total variable: _one; categories: 11 12 13 21 22 23).
Margin variable region (total variable: _one; categories: 1 2 3 4).
Margin variable race (total variable: _one; categories: 1 2 3).
Auxiliary variable _one (categories: 1).
{\smallskip}
file rakedwgt3-report.dta saved
{\smallskip}
      Source {\VBAR}       SS           df       MS      Number of obs   =    10,351
\HLI{13}{\PLUS}\HLI{34}   F(10, 10340)    >  99999.00
       Model {\VBAR}  2086.13859        10  208.613859   Prob > F        =    0.0000
    Residual {\VBAR}   .78315703    10,340  .000075741   R-squared       =    0.9996
\HLI{13}{\PLUS}\HLI{34}   Adj R-squared   =    0.9996
       Total {\VBAR}  2086.92175    10,350  .201634952   Root MSE        =     .0087
{\smallskip}
\HLI{13}{\TOPT}\HLI{64}
    __000003 {\VBAR}      Coef.   Std. Err.      t    P>|t|     [95\% Conf. Interval]
\HLI{13}{\PLUS}\HLI{64}
     sex_age {\VBAR}
         11  {\VBAR}   .0644365   .0002775   232.21   0.000     .0638925    .0649804
         12  {\VBAR}   .4545577   .0003154  1441.25   0.000     .4539395     .455176
         13  {\VBAR}   .6782466   .0002804  2418.71   0.000     .6776969    .6787963
         22  {\VBAR}   .3966406   .0003049  1300.84   0.000     .3960429    .3972383
         23  {\VBAR}   .7304392   .0002726  2679.97   0.000     .7299049    .7309734
             {\VBAR}
      region {\VBAR}
         NE  {\VBAR}  -.4455127   .0002536 -1756.49   0.000    -.4460099   -.4450155
         MW  {\VBAR}  -.4428144   .0002335 -1896.53   0.000    -.4432721   -.4423567
          W  {\VBAR}  -.6672675   .0002407 -2772.21   0.000    -.6677393   -.6667957
             {\VBAR}
        race {\VBAR}
      Black  {\VBAR}   .3360321   .0002848  1180.08   0.000     .3354739    .3365902
      Other  {\VBAR}   1.613276   .0006303  2559.34   0.000     1.612041    1.614512
             {\VBAR}
       _cons {\VBAR}   .5864801   .0002455  2388.48   0.000     .5859988    .5869614
\HLI{13}{\BOTT}\HLI{64}
{\smallskip}
Raking adjustments for sex_age variable:
  the smallest was        1.798 for category 21 (21)
  the greatest was        3.732 for category 23 (23)
Raking adjustments for region variable (1=NE, 2=MW, 3=S, 4=W):
  the smallest was        0.922 for category 4 (W)
  the greatest was        1.798 for category 3 (S)
Raking adjustments for race variable (1=white, 2=black, 3=other):
  the smallest was        1.798 for category 1 (White)
  the greatest was        9.023 for category 3 (Other)
\nullskip
\end{stlog}

It looks like \stcmd{ipfraking} had to work harder to adjust the weights of older females,
and especially other race individuals.

\subsection{Options of \stcmd{ipfraking\_report}}

\hangpara
\stcmd{raked\_weight(\varname)} specifies the name of the raked weight variable to create
    the report for. This is a required option.

\hangpara
\stcmd{matrices(\textit{namelist})} specifies a list of matrices (formatted as the matrices
    supplied to \stcmd{ctotal()} option of \stcmd{ipfraking}) to produce weighting reports for.
    In particular, the variables and their categories are picked up from these matrices;
    and the control totals/proportions are compared to those defined by the weight being reported on.

\hangpara
\stcmd{by(\varlist)} specifies a list of additional variables for which the weights are to
    be tabulated in the raking weights report. The difference with the \stcmd{matrices()} option
    is that the control totals for these variables may not be known (or may not be relevant).
    In particular, \stcmd{by(\_one)}, where \stcmd{\_one} is identically one, will produce
    the overall report.

\hangpara
\stcmd{xls} requests exporting the report to an Excel file.

\hangpara
\stcmd{replace} specifies that the files produced by \stcmd{ipfraking\_report} (i.e., the \stcmd{.dta}
    and the {\stmcd{.xls}} file if \stcmd{xls} option is specified) should be overwritten.

\hangpara
\stcmd{force} requires that a variable that may be found repeatedly (between the calibration variables
    supplied originally to \stcmd{ipfraking}, the variables found in the independent total \stcmd{matrices()},
    and the variables without the control totals provided in \stcmd{by()} option) is processed every
    time it is encountered. (Otherwise, it is only processed once.)

\subsection{Variables in the raking report}

The raking report file contains the following variables.

\noindent
\begin{tabular}{ll}
  \hline
  Variable name & Definition \\
  \hline
  \stcmd{Weight\_Variable} & The name of the weight variable, \stcmd{generate()} \\
  \stcmd{C\_Total\_Margin\_Variable\_Name} & The name of the control margin, \\
            & \stcmd{rowname} of the corresponding \stcmd{ctotal()} matrix \\
  \stcmd{C\_Total\_Margin\_Variable\_Label} & The label of the control margin variable \\
  \stcmd{Variable\_Class} & The role of the variable in the report: \\
        & Raking margin: a variable used as a calibration margin \\
        & (picked up automatically from the \stcmd{ctotal()} \\
        & matrix, provided \stcmd{meta} option was specified) \\
        & Other known target: supplied with \stcmd{matrices()} \\
        & option of \stcmd{ipfraking\_report} \\
        & Auxiliary variable: additional variable supplied \\
        & with \stcmd{by()} option of \stcmd{ipfraking\_report} \\
  \stcmd{C\_Total\_Arg\_Variable\_Name} & The name of the multiplier variable \\
  \stcmd{C\_Total\_Arg\_Variable\_Label} & The label of the multiplier variable \\
  \stcmd{C\_Total\_Margin\_Category\_Number} & Numeric value of the control total category \\
  \stcmd{C\_Total\_Margin\_Category\_Label} &  Label of the control total category \\
  \stcmd{Category\_Total\_Target} & The control total to be calibrated to \\
        & (the specific entry in the \stcmd{ctotal()} matrix) \\
  \stcmd{Category\_Total\_Prop} & Control total proportion \\
        & (the ratio of the specific entry in the \stcmd{ctotal()} \\
        & matrix to the matrix total) \\
  \stcmd{Unweighted\_Count} & Number of sample observations in the category \\
  \stcmd{Unweighted\_Prop} & Unweighted proportion \\
  \stcmd{Unweighted\_Prop\_Discrep} & Difference \stcmd{Unweighted\_Prop} - \stcmd{Category\_Total\_Prop} \\
  \stcmd{Category\_Total\_SRCWGT} & Weighted category total, with source weight \\
  \stcmd{Category\_Prop\_SRCWGT} & Weighted category proportion, with source weight \\
  \stcmd{Category\_Total\_Discrep\_SRCWGT} & Difference \stcmd{Category\_Total\_SRCWGT} - \\
        & - \stcmd{Category\_Total\_Target} \\
  \stcmd{Category\_Prop\_Discrep\_SRCWGT} & Difference \stcmd{Category\_Prop\_SRCWGT} - \\
        & - \stcmd{Category\_Total\_Prop} \\
  \stcmd{Category\_RelDiff\_SRCWGT} & \stcmd{reldif(Category\_Total\_SRCWGT,} \\
        & \stcmd{Category\_Total\_Target)} \\
  \stcmd{Overall\_Total\_SRCWGT} & Sum of source weights \\
  \stcmd{Source} & The name of the matrix from which the totals \\
        & were obtained \\
  \stcmd{Comment} & Placeholder for comments, to be entered during \\
        & manual review \\
  \hline
\end{tabular}

For each of the input weights (\stcmd{SRCWGT} suffix), raked weights (\stcmd{RKDWGT} suffix) and raking ratio
(the ratio of raked and input weights, \stcmd{RKDRATIO} suffix), the following summaries are provided.

\begin{tabular}{ll}
  \hline
  Variable name & Definition \\
  \hline
  \stcmd{Min\_\textit{WEIGHT}} & Min of source weights \\
  \stcmd{P25\_\textit{WEIGHT}} & 25th percentile of source weights \\
  \stcmd{P50\_\textit{WEIGHT}} & Median of source weights \\
  \stcmd{P75\_\textit{WEIGHT}} & 75th percentile of source weights \\
  \stcmd{Max\_\textit{WEIGHT}} & Max of source weights \\
  \stcmd{Mean\_\textit{WEIGHT}} & Mean of source weights \\
  \stcmd{SD\_\textit{WEIGHT}} & Standard deviation of source weights \\
  \stcmd{DEFF\_\textit{WEIGHT}} & Apparent UWE DEFF of source weights \\
  \hline
\end{tabular}

\subsection{Example}

\begin{stlog}
. use rakedwgt3-report, clear
(Weighting report on rakedwgt3)
{\smallskip}
. list C_Total_Margin_Variable_Name C_Total_Margin_Category_Label ///
>         Category_Total_Target Category_Total_RKDWGT DEFF_SRCWGT DEFF_RKDWGT , ///
>         sepby( C_Total_Margin_Variable_Name )
{\smallskip}
     {\TLC}\HLI{69}{\TRC}
     {\VBAR} C_Tota..   {\tytilde}y_Label   Categor{\tytilde}t   Categor..   DEFF_SR{\tytilde}T   DEFF_RK{\tytilde}T {\VBAR}
     {\LFTT}\HLI{69}{\RGTT}
  1. {\VBAR}  sex_age         11    41995394    41995394   1.2148059   1.6259899 {\VBAR}
  2. {\VBAR}  sex_age         12    42148662    42148662   1.2462168   1.5716613 {\VBAR}
  3. {\VBAR}  sex_age         13    26515340    26515340   1.2241095   1.5460785 {\VBAR}
  4. {\VBAR}  sex_age         21    41164255    41164255   1.2325105   1.5639529 {\VBAR}
  5. {\VBAR}  sex_age         22    43697440    43697440   1.1937826   1.5175312 {\VBAR}
  6. {\VBAR}  sex_age         23    32773080    32773080    1.233902    1.664307 {\VBAR}
     {\LFTT}\HLI{69}{\RGTT}
  7. {\VBAR}   region         NE    40679030    40679030   1.3056639   1.3657837 {\VBAR}
  8. {\VBAR}   region         MW    49205289    49205289   1.3475551   1.4909581 {\VBAR}
  9. {\VBAR}   region          S    85024007    85024006   1.4950056   1.4912995 {\VBAR}
 10. {\VBAR}   region          W    53385843    53385844    1.459859   2.3772667 {\VBAR}
     {\LFTT}\HLI{69}{\RGTT}
 11. {\VBAR}     race      White   1.784e+08   1.784e+08   1.4059259   1.4337901 {\VBAR}
 12. {\VBAR}     race      Black    29856865    29856865   1.5173846   1.5092533 {\VBAR}
 13. {\VBAR}     race      Other    20053682    20053682   1.3179136   1.2264706 {\VBAR}
     {\LFTT}\HLI{69}{\RGTT}
 14. {\VBAR}     _one          1           .   2.283e+08   1.4164382   1.7349278 {\VBAR}
     {\BLC}\HLI{69}{\BRC}
\nullskip
\end{stlog}

Functionality of \stcmd{ipfraking\_report} is aimed at manual quality control review,
which typically involves (i) categories with raking factors that differ the most (in the output),
and (ii) the resulting report file in Excel,
although for some aspects of automated quality control, it can be useful, as well.

\section{Collapsing weighting cells:  \stcmd{wgtcellcollapse}}

An additional new component of \stcmd{ipfraking} package is a tool to
semi-automatically collapse weighting cells, in order to achieve
a required minimal size of the weighting cell. (A typical recommendation
is to have cells of size 30 to 50.)

\begin{stsyntax}
wgtcellcollapse \textit{task}
\optif\
\optin\
,
\optional{task\_options}
}
\end{stsyntax}

where \textit{task} is one of:

\hangpara
\stcmd{define} to define collapsing rules explicitly

\hangpara
\stcmd{sequence} to create collapsing rules for a sequence of categories

\hangpara
\stcmd{report} to list the currently defined collapsing rules

\hangpara
\stcmd{candidate} to find rules applicable to a given category

\hangpara
\stcmd{collapse} to perform cell collapsing

\hangpara
\stcmd{label} to label collapsed cells using the original labels after \stcmd{wgtcellcollapse collapse}

\subsection{Syntax of \stcmd{wgtcellcollapse report}}

\begin{stsyntax}
wgtcellcollapse report
,
\underbar{var}iables(\varlist)
\optional{
break
}
\end{stsyntax}

\hangpara
\stcmd{\underbar{var}iables(\varlist)} is the list of variables for which the collapsing rule are to be reported

\hangpara
\stcmd{break} requires \stcmd{wgtcellcollapse report} to exit with error when technical inconsistencies are encountered

\subsection{Syntax of \stcmd{wgtcellcollapse define}}

\begin{stsyntax}
wgtcellcollapse define
,
\underbar{var}iables(\varlist)
\optional{
from(\textit{numlist})
to(\num)
label(\ststring)
max(\num)
clear
}
\end{stsyntax}

\hangpara
\stcmd{\underbar{var}iables(\varlist)} is the list of variables for which the collapsing rule can be used

\hangpara
\stcmd{from(\textit{numlist})} is the list of categories that can be collapsed according to this rule

\hangpara
\stcmd{to(\num)} is the numeric value of the new, collapsed category

\hangpara
\stcmd{label(\ststring)} is the value label to be attached to the new, collapsed category

\hangpara
\stcmd{max(\num)} overrides the automatically determined max value of the collapsed variable

\hangpara
\stcmd{clear} clears all the rules currently defined

Individual collapsing rules can be defined as follows.

\begin{stlog}
{\smallskip}
. 
. clear
{\smallskip}
. 
. set obs 4 
number of observations (_N) was 0, now 4
{\smallskip}
. 
. gen byte x = _n
{\smallskip}
. 
. label define x_lbl 1 "One" 2 "Two" 3 "Three" 4 "Four"
{\smallskip}
. 
. label values x x_lbl
{\smallskip}
. 
. wgtcellcollapse define, var(x) from(1 2 3) to(123)
{\smallskip}
. 
. wgtcellcollapse report, var(x)
{\smallskip}
Rule (1): collapse together
  x == 1 (One)
  x == 2 (Two)
  x == 3 (Three)
  into x == 123 (123)
  WARNING: unlabeled value x == 123
{\smallskip}
{\smallskip}
. 
\nullskip
\end{stlog}

Note how \stcmd{break} option of \stcmd{wgtcellcollapse} can be used to abort the execution
when technical deficiencies in the rules or in the data are encountered. In this case,
the label of the new category 123 was not defined, and this is considered a serious
enough deficiency to stop.

\begin{stlog}
{\smallskip}
. 
. wgtcellcollapse report, var(x) break
{\smallskip}
Rule (1): collapse together
  x == 1 (One)
  x == 2 (Two)
  x == 3 (Three)
  into x == 123 (123)
  ERROR: unlabeled value x == 123
assertion is false
r(9);
{\smallskip}
. 
. wgtcellcollapse define, var(x) clear
{\smallskip}
. 
. wgtcellcollapse define, var(x) from(1 2 3) to(123) label("One through three")
{\smallskip}
. 
. wgtcellcollapse report, var(x) break
{\smallskip}
Rule (1): collapse together
  x == 1 (One)
  x == 2 (Two)
  x == 3 (Three)
  into x == 123 (One through three)
{\smallskip}
{\smallskip}
. 
\nullskip
\end{stlog}

\subsection{Syntax of \stcmd{wgtcellcollapse sequence}}


\begin{stsyntax}
wgtcellcollapse sequence
,
\underbar{var}iables(\varlist)
from(\textit{numlist})
depth(\num)
\end{stsyntax}

\hangpara
\stcmd{\underbar{var}iables(\varlist)} is the list of variables for which the collapsing rule can be used

\hangpara
\stcmd{from(\textit{numlist})} is the sequence of values from which the plausible subsequences can be constructed

\hangpara
\stcmd{depth(\num)} is the maximum number of the original categories that can be collapsed

Moderate length sequences of collapsing categories can be defined as follows.

\begin{stlog}
{\smallskip}
. 
. clear
{\smallskip}
. 
. set obs 4
number of observations (_N) was 0, now 4
{\smallskip}
. 
. gen byte x = _n
{\smallskip}
. 
. label define x_lbl 1 "One" 2 "Two" 3 "Three" 4 "Four"
{\smallskip}
. 
. label values x x_lbl
{\smallskip}
. 
. wgtcellcollapse sequence, var(x) from(1 2 3 4) depth(3)
{\smallskip}
. 
. wgtcellcollapse report, var(x)
{\smallskip}
Rule (1): collapse together
  x == 1 (One)
  x == 2 (Two)
  into x == 212 (One to Two)
{\smallskip}
Rule (2): collapse together
  x == 2 (Two)
  x == 3 (Three)
  into x == 223 (Two to Three)
{\smallskip}
Rule (3): collapse together
  x == 3 (Three)
  x == 4 (Four)
  into x == 234 (Three to Four)
{\smallskip}
Rule (4): collapse together
  x == 1 (One)
  x == 2 (Two)
  x == 3 (Three)
  into x == 313 (One to Three)
{\smallskip}
Rule (5): collapse together
  x == 1 (One)
  x == 223 (Two to Three)
  into x == 313 (One to Three)
{\smallskip}
Rule (6): collapse together
  x == 3 (Three)
  x == 212 (One to Two)
  into x == 313 (One to Three)
{\smallskip}
Rule (7): collapse together
  x == 2 (Two)
  x == 3 (Three)
  x == 4 (Four)
  into x == 324 (Two to Four)
{\smallskip}
Rule (8): collapse together
  x == 2 (Two)
  x == 234 (Three to Four)
  into x == 324 (Two to Four)
{\smallskip}
Rule (9): collapse together
  x == 4 (Four)
  x == 223 (Two to Three)
  into x == 324 (Two to Four)
{\smallskip}
{\smallskip}
. 
\nullskip
\end{stlog}

Note how \stcmd{wgtcellcollapse sequence} automatically created labels for the collapsed cells.

When creating sequential collapses, \stcmd{wgtcellcollapse sequence} uses the following mnemonics
in creating the new categories:
\begin{itemize}
    \item First comes the length of the collapsed subsequence (up to \stcmd{depth(\num)}).
    \item Then comes the starting value of the category in the subsequence (padded by zeroes as needed).
    \item Then comes the ending value of the category in the subsequence (padded by zeroes as needed).
\end{itemize}

In the example above, rules 7 through 9 lead to collapsing into the new category 324. This
should be interpreted as ``the subsequence of length 3 that starts with category 2 and ends with category 4''.
A numeric value of the collapsed category that reads like 50412 means
``the subsequence of length 5 that starts with category 4 and ends with category 12''.
In that second example, \stcmd{wgtcellcollapse sequence} padded the value of 4 with an additional zero,
so that the length of resulting collapsed category value is always (\stnum of digits of the sequence length) +
twice (\stnum of digits of the largest original category).

Note that \stcmd{wgtcellcollapse sequence} respects the order in which the categories are
supplied in the \stcmd{from()} option, and does not sort them. If the categories are supplied 
in the order 2, 4, 1 and 3, then \stcmd{wgtcellcollapse sequence} would collapse 2 with 4, 4 with 1, 
and 1 with 3:

\begin{stlog}
. wgtcellcollapse define, var(x) clear
{\smallskip}
. wgtcellcollapse sequence, var(x) from(2 4 1 3) depth(2)
{\smallskip}
. wgtcellcollapse report, var(x)
{\smallskip}
Rule (1): collapse together
  x == 2 (Two)
  x == 4 (Four)
  into x == 224 (Two to Four)
{\smallskip}
Rule (2): collapse together
  x == 4 (Four)
  x == 1 (One)
  into x == 241 (Four to One)
{\smallskip}
Rule (3): collapse together
  x == 1 (One)
  x == 3 (Three)
  into x == 213 (One to Three)
{\smallskip}
\nullskip
\end{stlog}


\subsection{Syntax of \stcmd{wgtcellcollapse candidate}}

\begin{stsyntax}
wgtcellcollapse candidate
,
\underbar{var}iable(\varname)
category(\num)
\optional{ \max{\num} }
\end{stsyntax}

\hangpara
\stcmd{\underbar{var}iable(\varname)} is the variable whose collapsing rules are to be searched

\hangpara
\stcmd{category(\num)} is the category for which the candidate rules are to be identified

\hangpara
\stcmd{max(\num)} is the maximum value of the categories in the candidate rules to be returned

The rules found are quietly returned through the mechanism of \stcmd{sreturn},
see \pref{return}, as they are intended to stay in memory sufficiently long for
\stcmd{wgtcellcollapse collapse} to evaluate each rule. Going back to the example
with sequential collapses of depth 3, we can identify the following candidates
for categories 2, 212 (collapsed values of 1 and 2), and a non-existent category of 55:

\begin{stlog}
. wgtcellcollapse candidate, var(x) cat(2)
{\smallskip}
. sreturn list
{\smallskip}
macros:
           s(goodrule) : "1 2 4 7 8"
              s(rule8) : "2:234=324"
              s(rule7) : "2:3:4=324"
              s(rule4) : "1:2:3=313"
              s(rule2) : "2:3=223"
              s(rule1) : "1:2=212"
                s(cat) : "2"
                  s(x) : "x"
{\smallskip}
. wgtcellcollapse candidate, var(x) cat(2) max(9)
{\smallskip}
. sreturn list
{\smallskip}
macros:
           s(goodrule) : "1 2 4 7"
              s(rule7) : "2:3:4=324"
              s(rule4) : "1:2:3=313"
              s(rule2) : "2:3=223"
              s(rule1) : "1:2=212"
                s(cat) : "2"
                  s(x) : "x"
{\smallskip}
. wgtcellcollapse candidate, var(x) cat(212)
{\smallskip}
. sreturn list
{\smallskip}
macros:
           s(goodrule) : "6"
              s(rule6) : "3:212=313"
                s(cat) : "212"
                  s(x) : "x"
{\smallskip}
. wgtcellcollapse candidate, var(x) cat(55)
{\smallskip}
. sreturn list
{\smallskip}
macros:
                s(cat) : "55"
                  s(x) : "x"
\nullskip
\end{stlog}

In the second call to
the option \stcmd{max(9)} was used to restrict the returned rules to the rules
that deal with the original categories only (so rule 8 that involved a collapsed category 234 
was omitted). In the third call, a list of rules
that involve a collapsed category \stcmd{cat(212)} was requested. Requests
for nonexisting categories are not considered errors, but simply produce empty lists
of ``good rules''

\subsection{Syntax of \stcmd{wgtcellcollapse label}}

\begin{stsyntax}
wgtcellcollapse label
,
\underbar{var}iable(\varname)
\optional{ verbose force }
\end{stsyntax}

\hangpara
\stcmd{\underbar{var}iable(\varname)} is the collapsed variable to be labeled.

\hangpara
\stcmd{verbose} outputs the labeling results. There may be a lot of output.

\hangpara
\stcmd{force} instructs \stcmd{wgtcellcollapse label} to only use categories present in the data.

Example is given in section \ref{subsec:wgtcellcollapse:labels} below.

\subsection{Syntax of \stcmd{wgtcellcollapse collapse}}

\begin{stsyntax}
wgtcellcollapse collapse \optif \optin
,
\underbar{var}iables(\varlist)
mincellsize(\num)
\underbar{sav}ing(\textit{dofile\_name})
\optional{
\underbar{gen}erate(\newvarname)
replace
append
feed(\varname)
strict
sort(\varlist)
run
maxpass(\num)
\underbar{maxcat}egory(\num)
\underbar{zer}oes(\textit{numlist})
greedy
}
\end{stsyntax}

\hangpara
\stcmd{\underbar{var}iables(\varlist)} provides the list of variables whose cells are to be collapsed.
When more than one variable is specified, \stcmd{wgtcellcollapse collapse} proceeds from right to left,
i.e., first attempts to collapse the rightmost variable.

\hangpara
\stcmd{mincellsize(\num)} specifies the minimum cell size for the collapsed cells. For most weighting
purposes, values of 30 to 50 can be recommended.

\hangpara
\stcmd{\underbar{gen}erate(\newvarname)} specifies the name of the collapsed variable to be created.

\hangpara
\stcmd{feed(\varname)} provides the name of an already existing collapsed variable.

\hangpara
\stcmd{strict} modifies the behavior of \stcmd{wgtcellcollapse collapse} so that only
collapsing rules for which all participating categories have nonzero counts are utilized.

\hangpara
\stcmd{sort(\varlist)} sorts the data set before proceeding to collapse the cell.
The default sort order is in terms of the values of the collapsed variable.
A different sort order may produce a different set of collapsed cell when
cells are tied on size.

\hangpara
\stcmd{maxpass(\num)} specifies the maximum number of passes through the data set. The default value is 10000.}

\hangpara
\stcmd{\underbar{maxcat}egory(\num)} is the maximum category value of the variable being collapsed.
It is passed to the internal calls to \stcmd{wgtcellcollapse candidate}, see above.

\hangpara
\stcmd{\underbar{zer}oes(\textit{numlist})} provides a list of the categories of the collapsed
variable that may have zero counts in the data.

\hangpara
\stcmd{greedy} modifies the behavior \stcmd{wgtcellcollapse collapse} to prefer the rules
that collapse the maximum number of categories.

Options to deal with the do-file to write the collapsing code to:	

\hangpara
\stcmd{\underbar{sav}ing(\textit{dofile\_name})} specifies the name of the do-file that will contain the cell collapsing code.

\hangpara
\stcmd{replace} overwrites the do-file if one exists.

\hangpara
\stcmd{append} appends the code to the existing do-file.

\hangpara
\stcmd{run} specifies that the do-file created is run upon completion. This option is typically specified with most runs.

The primary intent of \stcmd{wgtcellcollapse collapse} is to create the code that can be
utilized for both the survey data file and the population targets data file that
are assumed to have identically named variables. Thus it does not only manipulate the data in the memory
and collapses the cells, but also produces the do-file code that can be recycled.
To that effect, when a do-file is created with the \stcmd{replace} and \stcmd{saving()} options,
the user needs to specify \stcmd{generate()} option to provide the name of the collapsed variable;
and when the said do-file is appended with the the \stcmd{append} and \stcmd{saving()} options,
the name of that variable is provided with the \stcmd{feed()} option.

The algorithm \stcmd{wgtcellcollapse collapse} uses to identify the cells to be collapsed is
a variation of greedy search.
It first identifies the cells with the lowest (positive) counts; finds the candidate rules
for the variable(s) to be collapsed; evaluates the counts of the collapsed cells across all 
these candidate rules; and uses the rule that has produces the smallest size of the
collapsed cell across all applicable rules. So when it finds several rules that are applicable
to the cell being currently processed that has a size of 5, and the candidate rules produce cells
of sizes 7, 10 and 15, \stcmd{wgtcellcollapse collapse} will use the rule that produces the cell
of size 7. The algorithm runs until all cells have sizes of at least
\stcmd{mincellsize(\num)} or until \stcmd{maxpass(\num)} passes through the data are executed.
It is a pretty dumb algorithm, actually, and it fails quite often. 
For that reason, a number of hooks are provided to modify its behavior.
Section \ref{subsec:example} will demonstrate the typical failures, and the ways to overcome them.

\textit{Hint 1}. Since \stcmd{wgtcellcollapse collapse} works with the sample data,
it will not be able to identify categories that are not observed in the sample (e.g., rare categories),
but may be present in the population. This will lead to errors at the raking stage,
when the control total matrices have more categories than the data, forcing \stcmd{ipfraking} to stop.
To help with that, the option \stcmd{zeroes()} allows the user to pass the categories
of the variables that are known to exist in the population but not in the sample.

\textit{Hint 2}. The behavior of \stcmd{wgtcellcollapse collapse, zeroes()} may still not be
satisfactory. As it evaluates the sample sizes of the collapsed cells across a number
of candidate rules that involve zero cells, it will probably pick up the rule with lowest
number, and that rule may as well leave some other candidate rules with zero cells untouched.
This may create problems when \stcmd{wgtcellcollapse collapse} returns to those untouched cells,
and looks for the existing cells to collapse them with, creating collapsing rules with breaks
in the sequences. To improve upon that behavior, option \stcmd{greedy} makes
\stcmd{wgtcellcollapse collapse} look for a rule that has as many categories as possible, thus collapsing
as many categories with zero counts in one swipe as it can.

\textit{Hint 3}. Other than for dealing with zero cells, the option \stcmd{strict} should be specified
most of the times. It effectively makes sure that the candidate rules correspond to the actual data.

\textit{Hint 4}. Sometimes, you see some combinations in the data that seem like a nobrainer
to collapse. Well, they are nobrainers to you, but \stcmd{wgtcellcollapse collapse} is not that smart.
If you want to guarantee some specific combination of cells to be collapsed by \stcmd{wgtcellcollapse collapse},
your best bet may be to explicitly identify them with the \ifexp\ condition, and specify some
ridiculously large cell size like \stcmd{mincellsize(10000)} so that \stcmd{wgtcellcollapse collapse} makes every possible
effort to collapse those cells. It will exit with a complaint that this size could not be achieved,
but hopefully the cells will be collapsed as needed.

\subsection{Motivating example}
\label{subsec:example}

Development of \stcmd{wgtcellcollapse} was to address the need
to collapse cells of the margin variables so that each cell has a minimum sample size;
and to do so in a way that can be easily made consistent between the sample data
and the population targets data. The problem arises when some of the target
variables have dozens of categories, most of which have small counts.
While the primary motivation comes from transportation surveys,
the ideas are also applicable to other domains, e.g.,
continuous age variables or highly detailed race/ethnicity or region of origin
categories in health or economic surveys.

The workflow of \stcmd{wgtcellcollapse} is demonstrated with the following
simulated data set of trips along a metro line composed of 21 stations:

\begin{stlog}
. use stations, clear
{\smallskip}
. list station_id, sep(0)
{\smallskip}
     {\TLC}\HLI{22}{\TRC}
     {\VBAR}           station_id {\VBAR}
     {\LFTT}\HLI{22}{\RGTT}
  1. {\VBAR}           3. Alewife {\VBAR}
  2. {\VBAR}         6. Brookline {\VBAR}
  3. {\VBAR}        10. Carmenton {\VBAR}
  4. {\VBAR}         13. Dogville {\VBAR}
  5. {\VBAR}         17. East End {\VBAR}
  6. {\VBAR}       21. Framington {\VBAR}
  7. {\VBAR}   25. Grand Junction {\VBAR}
  8. {\VBAR}       28. High Point {\VBAR}
  9. {\VBAR}       32. Irvingtown {\VBAR}
 10. {\VBAR}       36. Johnsville {\VBAR}
 11. {\VBAR}      39. King Street {\VBAR}
 12. {\VBAR}         42. Limerick {\VBAR}
 13. {\VBAR}      46. Moscow City {\VBAR}
 14. {\VBAR}     49. Ninth Street {\VBAR}
 15. {\VBAR}     52. Ontario Lake {\VBAR}
 16. {\VBAR} 56. Picadilly Square {\VBAR}
 17. {\VBAR}       59. Queens Zoo {\VBAR}
 18. {\VBAR}   63. Redline Circle {\VBAR}
 19. {\VBAR}    66. Silver Spring {\VBAR}
 20. {\VBAR}      70. Toledo Town {\VBAR}
 21. {\VBAR}    73. Union Station {\VBAR}
     {\BLC}\HLI{22}{\BRC}
\nullskip
\end{stlog}

Turnstile counts were collected at entrances and exits of the stations, producing the following
population figures.

\noindent
\begin{stlog}
. use trip_population, clear
{\smallskip}
. table board_id daypart if alight_id == 73, c(mean geton) cellwidth(10)
{\smallskip}
note: cellwidth too small, variable name truncated;
      to increase cellwidth, specify cellwidth(\#)
{\smallskip}
\HLI{21}{\TOPT}\HLI{59}
                     {\VBAR}                          daypart                          
            board_id {\VBAR}    AM Peak      Midday  PM Reverse       Night     Weekend
\HLI{21}{\PLUS}\HLI{59}
          3. Alewife {\VBAR}       1423          34         219         113          44
        6. Brookline {\VBAR}       7198         298         773         169         144
       10. Carmenton {\VBAR}      19254         181        3739         872         422
        13. Dogville {\VBAR}      12626         872        3476         769        1270
        17. East End {\VBAR}       2470         143        1263         145         114
      21. Framington {\VBAR}        634          50        1296         133          60
  25. Grand Junction {\VBAR}       2208         233         439          88         166
      28. High Point {\VBAR}       4319         424        3740         482         115
      32. Irvingtown {\VBAR}       1221          34         444          30         167
      36. Johnsville {\VBAR}         93           4          64           2           6
     39. King Street {\VBAR}        398          46          76          11          13
        42. Limerick {\VBAR}       1021          19         129          53          34
     46. Moscow City {\VBAR}       3300         776         984         140         301
    49. Ninth Street {\VBAR}         38          22         191           5           5
    52. Ontario Lake {\VBAR}        606          22          80          18          23
56. Picadilly Square {\VBAR}        642          71         622         153          69
      59. Queens Zoo {\VBAR}        331          23         174          15          19
  63. Redline Circle {\VBAR}        270           4          63          13           3
   66. Silver Spring {\VBAR}       3402         240         950         206         445
     70. Toledo Town {\VBAR}       5085          61         744         272         112
\HLI{21}{\BOTT}\HLI{59}
{\smallskip}
. table alight_id daypart if board_id==3, c(mean getoff) cellwidth(10)
{\smallskip}
note: cellwidth too small, variable name truncated;
      to increase cellwidth, specify cellwidth(\#)
{\smallskip}
\HLI{21}{\TOPT}\HLI{59}
                     {\VBAR}                          daypart                          
           alight_id {\VBAR}    AM Peak      Midday  PM Reverse       Night     Weekend
\HLI{21}{\PLUS}\HLI{59}
        6. Brookline {\VBAR}         19           0           3           2           0
       10. Carmenton {\VBAR}        492          18          56          23          15
        13. Dogville {\VBAR}       2475          42         423         153          80
        17. East End {\VBAR}        929          31         193          67          68
      21. Framington {\VBAR}        404          13          91          28          27
  25. Grand Junction {\VBAR}        576          20         147          42          41
      28. High Point {\VBAR}       2189          89         560         165         167
      32. Irvingtown {\VBAR}        288          10          91          21          18
      36. Johnsville {\VBAR}         41           0          11           2           1
     39. King Street {\VBAR}        131           3          38           8           6
        42. Limerick {\VBAR}        277           9          87          20          18
     46. Moscow City {\VBAR}       1746          78         556         142         128
    49. Ninth Street {\VBAR}         88           2          25           3           4
    52. Ontario Lake {\VBAR}        232          11          70          14          14
56. Picadilly Square {\VBAR}        633          33         198          47          47
      59. Queens Zoo {\VBAR}        230          10          71          13          14
  63. Redline Circle {\VBAR}         90           2          26           3           4
   66. Silver Spring {\VBAR}       1134          67         369          91          85
     70. Toledo Town {\VBAR}       1372          81         444         112         118
   73. Union Station {\VBAR}      53193        3038       16007        2733        2677
\HLI{21}{\BOTT}\HLI{59}
\nullskip
\end{stlog}

A survey was administered to a sample of the metro line users, with the following counts
of cases collected.

% make this a table?

\noindent
\begin{stlog}
. use trip_sample, clear
{\smallskip}
. tab board_id daypart
{\smallskip}
                     {\VBAR}                        daypart
            board_id {\VBAR}   AM Peak     Midday  PM Revers      Night    Weekend {\VBAR}     Total
\HLI{21}{\PLUS}\HLI{55}{\PLUS}\HLI{10}
          3. Alewife {\VBAR}        46          4         11          7          3 {\VBAR}        71 
        6. Brookline {\VBAR}       235          4         35          6          7 {\VBAR}       287 
       10. Carmenton {\VBAR}       652          4        185         46         24 {\VBAR}       911 
        13. Dogville {\VBAR}       411         41        166         35         56 {\VBAR}       709 
        17. East End {\VBAR}        86          5         64          4          4 {\VBAR}       163 
      21. Framington {\VBAR}        30          3         74          3          1 {\VBAR}       111 
  25. Grand Junction {\VBAR}        73         13         23          6          6 {\VBAR}       121 
      28. High Point {\VBAR}       160         20        188         25         12 {\VBAR}       405 
      32. Irvingtown {\VBAR}        35          2         25          1         15 {\VBAR}        78 
      36. Johnsville {\VBAR}         5          1          1          0          0 {\VBAR}         7 
     39. King Street {\VBAR}        17          1          2          0          1 {\VBAR}        21 
        42. Limerick {\VBAR}        28          0          9          1          3 {\VBAR}        41 
     46. Moscow City {\VBAR}        95         31         50          7         13 {\VBAR}       196 
    49. Ninth Street {\VBAR}         0          0          9          0          0 {\VBAR}         9 
    52. Ontario Lake {\VBAR}        13          1          4          1          1 {\VBAR}        20 
56. Picadilly Square {\VBAR}        23          4         35          7          5 {\VBAR}        74 
      59. Queens Zoo {\VBAR}        10          1         14          0          2 {\VBAR}        27 
  63. Redline Circle {\VBAR}        13          0          5          0          0 {\VBAR}        18 
   66. Silver Spring {\VBAR}       106         18         38         12         17 {\VBAR}       191 
     70. Toledo Town {\VBAR}       154          6         33         11          3 {\VBAR}       207 
\HLI{21}{\PLUS}\HLI{55}{\PLUS}\HLI{10}
               Total {\VBAR}     2,192        159        971        172        173 {\VBAR}     3,667 
{\smallskip}
{\smallskip}
. tab alight_id daypart
{\smallskip}
                     {\VBAR}                        daypart
           alight_id {\VBAR}   AM Peak     Midday  PM Revers      Night    Weekend {\VBAR}     Total
\HLI{21}{\PLUS}\HLI{55}{\PLUS}\HLI{10}
        6. Brookline {\VBAR}         1          0          0          0          0 {\VBAR}         1 
       10. Carmenton {\VBAR}        11          1          1          0          1 {\VBAR}        14 
        13. Dogville {\VBAR}        85          1         14          6          5 {\VBAR}       111 
        17. East End {\VBAR}        35          1         18          1          4 {\VBAR}        59 
      21. Framington {\VBAR}        14          1          2          2          2 {\VBAR}        21 
  25. Grand Junction {\VBAR}        14          2          8          1          1 {\VBAR}        26 
      28. High Point {\VBAR}        72          4         22         10          8 {\VBAR}       116 
      32. Irvingtown {\VBAR}         9          0          4          2          2 {\VBAR}        17 
      36. Johnsville {\VBAR}         3          0          1          0          0 {\VBAR}         4 
     39. King Street {\VBAR}         0          0          3          0          0 {\VBAR}         3 
        42. Limerick {\VBAR}        13          0          2          0          2 {\VBAR}        17 
     46. Moscow City {\VBAR}        80          6         22          6          6 {\VBAR}       120 
    49. Ninth Street {\VBAR}         3          1          1          0          0 {\VBAR}         5 
    52. Ontario Lake {\VBAR}         2          0          1          2          1 {\VBAR}         6 
56. Picadilly Square {\VBAR}        23          1          8          3          2 {\VBAR}        37 
      59. Queens Zoo {\VBAR}         6          0          5          1          0 {\VBAR}        12 
  63. Redline Circle {\VBAR}         5          0          0          0          0 {\VBAR}         5 
   66. Silver Spring {\VBAR}        49          0         19          3          9 {\VBAR}        80 
     70. Toledo Town {\VBAR}        43          3         24          6          7 {\VBAR}        83 
   73. Union Station {\VBAR}     1,724        138        816        129        123 {\VBAR}     2,930 
\HLI{21}{\PLUS}\HLI{55}{\PLUS}\HLI{10}
               Total {\VBAR}     2,192        159        971        172        173 {\VBAR}     3,667 
{\smallskip}
{\smallskip}
\nullskip
\end{stlog}

As only 3654\nullskip surveys were collected from a total of
96783\nullskip riders, we would reasonably expect
that things do not align quite well. We expect weighting to correct for at least a portion of
that nonresponse. The data available for calibration includes the population turnstile counts
listed above, and we will produce interactions of daypart and station that will serve as two
weighting margins (one for the stations where the metro users boarded, and one for the stations
where they got off).

First, we need to define the weighting rules. In this case, the stations are numbered sequentially,
with the northernmost, say, station Alewife being number 3, and the southernmost station,
Union Station, where everybody gets off to rush to their city jobs or attractions, being number 69.
Below, we create a list of stations and provide it to \stcmd{wgtcellcollapse sequence}.
We would be collapsing stations along the line, with the expectation that travelers boarding or leaving
at adjacent stations within the same day part are more similar to one another than the travelers
boarding or leaving a particular station at different times of the day. Collapsing rules
need to be defined for the \stcmd{daypart} variable as well --- mostly because \stcmd{wgtcellcollapse collapse}
expects all variables to have collapsing rules defined.

\begin{stlog}
. use trip_sample, clear
{\smallskip}
. wgtcellcollapse sequence , var(daypart) from(2 3 4) depth(3)
{\smallskip}
. levelsof board_id, local(stations_on)
3 6 10 13 17 21 25 28 32 36 39 42 46 49 52 56 59 63 66 70
{\smallskip}
. levelsof alight_id, local(stations_off)
6 10 13 17 21 25 28 32 36 39 42 46 49 52 56 59 63 66 70 73
{\smallskip}
. local all_stations : list stations_on | stations_off
{\smallskip}
. wgtcellcollapse sequence , var(board_id alight_id) from(`all_stations') depth(20)
{\smallskip}
\nullskip
\end{stlog}

The number of collapsing rules for variables \stcmd{board\_id} and \stcmd{alight\_id}
created by \stcmd{wgtcellcollapse sequence}
is   2961\nullskip each.

\subsubsection{The first pass of cell collapse and raking}

Let us say that we want to define weighting cells with at least 20 cases in each.
We will thus start with weighting cells defined as station-by-daypart interaction,
and collapsing stations within daypart to achieve the cell sizes of at least 20 cases.
Here is what a simple run of \stcmd{wgtcellcollapse collapse} might look like.

\begin{stlog}
. use trip_sample_rules, clear
{\smallskip}
. wgtcellcollapse collapse, variables(daypart board_id) mincellsize(20) ///
>         generate(dpston1) saving(dpston1.do) replace run
(note: file dpston1.do not found)
Pass 0 through the data...
  smallest count = 1 in the cell      2000039
  Invoking rule 39:40=23940
  replace dpston1 = 2023940 if inlist(dpston1, 2000039, 2000040)
Pass 1 through the data...
  smallest count = 1 in the cell      2000050
  Invoking rule 50:53=25053
  replace dpston1 = 2025053 if inlist(dpston1, 2000050, 2000053)
Pass 2 through the data...
  smallest count = 1 in the cell      2000055
  Invoking rule 55:25053=35055
  replace dpston1 = 2035055 if inlist(dpston1, 2000055, 2025053)
Pass 3 through the data...
  smallest count = 1 in the cell      3000039
  Invoking rule 39:40=23940
  replace dpston1 = 3023940 if inlist(dpston1, 3000039, 3000040)
Pass 4 through the data...
  smallest count = 1 in the cell      4000036
  Invoking rule 36:39:40:44=43644
  replace dpston1 = 4043644 if inlist(dpston1, 4000036, 4000039, 4000040, 4000044)
Pass 5 through the data...
  smallest count = 1 in the cell      4000050
  Invoking rule 43644:24749:50=73650
  replace dpston1 = 4073650 if inlist(dpston1, 4043644, 4024749, 4000050)
Pass 6 through the data...
  smallest count = 1 in the cell      5000024
  Invoking rule 18:24=21824
  replace dpston1 = 5021824 if inlist(dpston1, 5000018, 5000024)
Pass 7 through the data...
  smallest count = 1 in the cell      5000040
  Invoking rule 40:44=24044
  replace dpston1 = 5024044 if inlist(dpston1, 5000040, 5000044)
Pass 8 through the data...
  smallest count = 1 in the cell      5000050
  Invoking rule 24044:24749:50=54050
  replace dpston1 = 5054050 if inlist(dpston1, 5024044, 5024749, 5000050)
Pass 9 through the data...
  smallest count = 2 in the cell      2000036
  Invoking rule 36:23940=33640
  replace dpston1 = 2033640 if inlist(dpston1, 2000036, 2023940)
Pass 10 through the data...
  smallest count = 2 in the cell      5000055
  Invoking rule 53:55=25355
  replace dpston1 = 5025355 if inlist(dpston1, 5000053, 5000055)
Pass 11 through the data...
  smallest count = 3 in the cell      2000024
  Invoking rule 24:22630:33640=62440
  replace dpston1 = 2062440 if inlist(dpston1, 2000024, 2022630, 2033640)
Pass 12 through the data...
  smallest count = 3 in the cell      3023940
  Invoking rule 44:23940=33944
  replace dpston1 = 3033944 if inlist(dpston1, 3000044, 3023940)
Pass 13 through the data...
  smallest count = 3 in the cell      4000024
  Invoking rule 24:22630:73650=102450
  replace dpston1 = 4102450 if inlist(dpston1, 4000024, 4022630, 4073650)
Pass 14 through the data...
  smallest count = 3 in the cell      5000001
  Invoking rule 1:2=20102
  replace dpston1 = 5020102 if inlist(dpston1, 5000001, 5000002)
Pass 15 through the data...
  smallest count = 3 in the cell      5000068
  Invoking rule 25355:26062:68=55368
  replace dpston1 = 5055368 if inlist(dpston1, 5025355, 5026062, 5000068)
Pass 16 through the data...
  smallest count = 4 in the cell      2000001
  Invoking rule 1:2=20102
  replace dpston1 = 2020102 if inlist(dpston1, 2000001, 2000002)
Pass 17 through the data...
  smallest count = 4 in the cell      2000008
  Invoking rule 8:21118:62440=90840
  replace dpston1 = 2090840 if inlist(dpston1, 2000008, 2021118, 2062440)
Pass 18 through the data...
  smallest count = 4 in the cell      3000050
  Invoking rule 49:50=24950
  replace dpston1 = 3024950 if inlist(dpston1, 3000049, 3000050)
Pass 19 through the data...
  smallest count = 4 in the cell      4000018
  Invoking rule 18:24:26=31826
  replace dpston1 = 4031826 if inlist(dpston1, 4000018, 4000024, 4000026)
Pass 20 through the data...
  smallest count = 5 in the cell      1000039
  Invoking rule 39:40=23940
  replace dpston1 = 1023940 if inlist(dpston1, 1000039, 1000040)
Pass 21 through the data...
  smallest count = 5 in the cell      2000018
  Invoking rule 20102:20811:18=50118
  replace dpston1 = 2050118 if inlist(dpston1, 2020102, 2020811, 2000018)
Pass 22 through the data...
  smallest count = 5 in the cell      3000060
  Invoking rule 24950:25355:60=54960
  replace dpston1 = 3054960 if inlist(dpston1, 3024950, 3025355, 3000060)
Pass 23 through the data...
  smallest count = 5 in the cell      5021824
  Invoking rule 26:21824=31826
  replace dpston1 = 5031826 if inlist(dpston1, 5000026, 5021824)
Pass 24 through the data...
  smallest count = 5 in the cell      5054050
  Invoking rule 54050:55368=104068
  replace dpston1 = 5104068 if inlist(dpston1, 5054050, 5055368)
Pass 25 through the data...
  smallest count = 6 in the cell      2000068
  Invoking rule 35055:26062:68=65068
  replace dpston1 = 2065068 if inlist(dpston1, 2035055, 2026062, 2000068)
Pass 26 through the data...
  smallest count = 6 in the cell      4000002
  Invoking rule 1:2=20102
  replace dpston1 = 4020102 if inlist(dpston1, 4000001, 4000002)
Pass 27 through the data...
  smallest count = 6 in the cell      4102450
  Invoking rule 53:102450=112453
  replace dpston1 = 4112453 if inlist(dpston1, 4000053, 4102450)
Pass 28 through the data...
  smallest count = 7 in the cell      4000047
  Invoking rule 47:49:50:53:55:60:62=74762
  replace dpston1 = 4074762 if inlist(dpston1, 4000047, 4000049, 4000050, 4000053, 4000055, 4000060, 400006
> 2)
Pass 29 through the data...
  smallest count = 9 in the cell      4031826
  Invoking rule 30:31826=41830
  replace dpston1 = 4041830 if inlist(dpston1, 4000030, 4031826)
Pass 30 through the data...
  smallest count = 10 in the cell      1000055
  Invoking rule 55:60=25560
  replace dpston1 = 1025560 if inlist(dpston1, 1000055, 1000060)
Pass 31 through the data...
  smallest count = 10 in the cell      5020102
  Invoking rule 8:20102=30108
  replace dpston1 = 5030108 if inlist(dpston1, 5000008, 5020102)
Pass 32 through the data...
  smallest count = 11 in the cell      2090840
  Invoking rule 90840:44:47=110847
  replace dpston1 = 2110847 if inlist(dpston1, 2090840, 2000044, 2000047)
Pass 33 through the data...
  smallest count = 11 in the cell      3000001
  Invoking rule 1:2=20102
  replace dpston1 = 3020102 if inlist(dpston1, 3000001, 3000002)
Pass 34 through the data...
  smallest count = 11 in the cell      4000068
  Invoking rule 68:74762=84768
  replace dpston1 = 4084768 if inlist(dpston1, 4000068, 4074762)
Pass 35 through the data...
  smallest count = 11 in the cell      5031826
  Invoking rule 30:31826=41830
  replace dpston1 = 5041830 if inlist(dpston1, 5000030, 5031826)
Pass 36 through the data...
  smallest count = 12 in the cell      2065068
  WARNING: could not find any rules to collapse dpston1 == 2065068
Pass 37 through the data...
  smallest count = 12 in the cell      3033944
  Invoking rule 26:23036:33944=62644
  replace dpston1 = 3062644 if inlist(dpston1, 3000026, 3023036, 3033944)
Pass 38 through the data...
  smallest count = 13 in the cell      1000050
  Invoking rule 50:53=25053
  replace dpston1 = 1025053 if inlist(dpston1, 1000050, 1000053)
Pass 39 through the data...
  smallest count = 13 in the cell      2000026
  Invoking rule 50118:24:26=70126
  replace dpston1 = 2070126 if inlist(dpston1, 2050118, 2000024, 2000026)
Pass 40 through the data...
  smallest count = 13 in the cell      4020102
  Invoking rule 8:20102=30108
  replace dpston1 = 4030108 if inlist(dpston1, 4000008, 4020102)
Pass 41 through the data...
  smallest count = 13 in the cell      4112453
  Invoking rule 11:18:112453=131153
  replace dpston1 = 4131153 if inlist(dpston1, 4000011, 4000018, 4112453)
Pass 42 through the data...
  smallest count = 13 in the cell      5000047
  Invoking rule 36:39:40:44:47=53647
  replace dpston1 = 5053647 if inlist(dpston1, 5000036, 5000039, 5000040, 5000044, 5000047)
Pass 43 through the data...
  smallest count = 14 in the cell      3000055
  Invoking rule 53:55=25355
  replace dpston1 = 3025355 if inlist(dpston1, 3000053, 3000055)
Pass 44 through the data...
  smallest count = 15 in the cell      5104068
  WARNING: could not find any rules to collapse dpston1 == 5104068
Pass 45 through the data...
  smallest count = 17 in the cell      5000062
  Invoking rule 11:18:24:26:30:36:39:40:44:47:49:50:53:55:60:62=161162
  replace dpston1 = 5161162 if inlist(dpston1, 5000011, 5000018, 5000024, 5000026, 5000030, 5000036, 500003
> 9, 5000040, 5000044, 5000047, 5000049, 5000050, 5000053, 5000055, 5000060, 5000062)
Pass 46 through the data...
  smallest count = 18 in the cell      2000062
  Invoking rule 30:36:39:40:44:47:49:50:53:55:60:62=123062
  replace dpston1 = 2123062 if inlist(dpston1, 2000030, 2000036, 2000039, 2000040, 2000044, 2000047, 200004
> 9, 2000050, 2000053, 2000055, 2000060, 2000062)
Pass 47 through the data...
  smallest count = 18 in the cell      3054960
  Invoking rule 62:54960=64962
  replace dpston1 = 3064962 if inlist(dpston1, 3000062, 3054960)
Pass 48 through the data...
  smallest count = 22 in the cell      1023940
  Done collapsing! Exiting...
{\smallskip}
. return list
{\smallskip}
scalars:
         r(arg_min_id) =  1023940
                r(min) =  22
{\smallskip}
macros:
            r(cfailed) : "2065068,5104068"
             r(failed) : "2065068 5104068"
{\smallskip}
. wgtcellcollapse collapse, variables(daypart alight_id) mincellsize(20) ///
>         generate(dpstoff1) saving(dpstoff1.do) replace run
(note: file dpstoff1.do not found)
Pass 0 through the data...
  smallest count = 1 in the cell      1000002
  Invoking rule 2:8=20208
  replace dpstoff1 = 1020208 if inlist(dpstoff1, 1000002, 1000008)
Pass 1 through the data...
  smallest count = 1 in the cell      2000008
  Invoking rule 8:11=20811
  replace dpstoff1 = 2020811 if inlist(dpstoff1, 2000008, 2000011)
Pass 2 through the data...
  smallest count = 1 in the cell      2000018
  Invoking rule 18:24=21824
  replace dpstoff1 = 2021824 if inlist(dpstoff1, 2000018, 2000024)
Pass 3 through the data...
  smallest count = 1 in the cell      2000049
  Invoking rule 49:50:53=34953
  replace dpstoff1 = 2034953 if inlist(dpstoff1, 2000049, 2000050, 2000053)
Pass 4 through the data...
  smallest count = 1 in the cell      3000008
  Invoking rule 8:11=20811
  replace dpstoff1 = 3020811 if inlist(dpstoff1, 3000008, 3000011)
Pass 5 through the data...
  smallest count = 1 in the cell      3000039
  Invoking rule 39:40=23940
  replace dpstoff1 = 3023940 if inlist(dpstoff1, 3000039, 3000040)
Pass 6 through the data...
  smallest count = 1 in the cell      3000049
  Invoking rule 49:50=24950
  replace dpstoff1 = 3024950 if inlist(dpstoff1, 3000049, 3000050)
Pass 7 through the data...
  smallest count = 1 in the cell      4000018
  Invoking rule 18:24=21824
  replace dpstoff1 = 4021824 if inlist(dpstoff1, 4000018, 4000024)
Pass 8 through the data...
  smallest count = 1 in the cell      4000026
  Invoking rule 26:21824=31826
  replace dpstoff1 = 4031826 if inlist(dpstoff1, 4000026, 4021824)
Pass 9 through the data...
  smallest count = 1 in the cell      4000055
  Invoking rule 53:55=25355
  replace dpstoff1 = 4025355 if inlist(dpstoff1, 4000053, 4000055)
Pass 10 through the data...
  smallest count = 1 in the cell      5000008
  Invoking rule 8:11=20811
  replace dpstoff1 = 5020811 if inlist(dpstoff1, 5000008, 5000011)
Pass 11 through the data...
  smallest count = 1 in the cell      5000026
  Invoking rule 24:26=22426
  replace dpstoff1 = 5022426 if inlist(dpstoff1, 5000024, 5000026)
Pass 12 through the data...
  smallest count = 1 in the cell      5000050
  Invoking rule 50:53=25053
  replace dpstoff1 = 5025053 if inlist(dpstoff1, 5000050, 5000053)
Pass 13 through the data...
  smallest count = 2 in the cell      1000050
  Invoking rule 49:50=24950
  replace dpstoff1 = 1024950 if inlist(dpstoff1, 1000049, 1000050)
Pass 14 through the data...
  smallest count = 2 in the cell      2000026
  Invoking rule 26:21824=31826
  replace dpstoff1 = 2031826 if inlist(dpstoff1, 2000026, 2021824)
Pass 15 through the data...
  smallest count = 2 in the cell      2020811
  Invoking rule 20811:31826=50826
  replace dpstoff1 = 2050826 if inlist(dpstoff1, 2020811, 2031826)
Pass 16 through the data...
  smallest count = 2 in the cell      2034953
  Invoking rule 47:34953=44753
  replace dpstoff1 = 2044753 if inlist(dpstoff1, 2000047, 2034953)
Pass 17 through the data...
  smallest count = 2 in the cell      3000024
  Invoking rule 24:26=22426
  replace dpstoff1 = 3022426 if inlist(dpstoff1, 3000024, 3000026)
Pass 18 through the data...
  smallest count = 2 in the cell      3000044
  Invoking rule 44:23940=33944
  replace dpstoff1 = 3033944 if inlist(dpstoff1, 3000044, 3023940)
Pass 19 through the data...
  smallest count = 2 in the cell      3024950
  Invoking rule 53:24950=34953
  replace dpstoff1 = 3034953 if inlist(dpstoff1, 3000053, 3024950)
Pass 20 through the data...
  smallest count = 2 in the cell      4000036
  Invoking rule 36:39:40:44:47=53647
  replace dpstoff1 = 4053647 if inlist(dpstoff1, 4000036, 4000039, 4000040, 4000044, 4000047)
Pass 21 through the data...
  smallest count = 2 in the cell      4000050
  Invoking rule 50:53:55:60:62=55062
  replace dpstoff1 = 4055062 if inlist(dpstoff1, 4000050, 4000053, 4000055, 4000060, 4000062)
Pass 22 through the data...
  smallest count = 2 in the cell      5000036
  Invoking rule 36:39:40:44=43644
  replace dpstoff1 = 5043644 if inlist(dpstoff1, 5000036, 5000039, 5000040, 5000044)
Pass 23 through the data...
  smallest count = 3 in the cell      1000039
  Invoking rule 36:39=23639
  replace dpstoff1 = 1023639 if inlist(dpstoff1, 1000036, 1000039)
Pass 24 through the data...
  smallest count = 3 in the cell      2000068
  Invoking rule 30:36:39:40:44:47:49:50:53:55:60:62:68=133068
  replace dpstoff1 = 2133068 if inlist(dpstoff1, 2000030, 2000036, 2000039, 2000040, 2000044, 2000047, 2000
> 049, 2000050, 2000053, 2000055, 2000060, 2000062, 2000068)
Pass 25 through the data...
  smallest count = 3 in the cell      5022426
  Invoking rule 18:22426=31826
  replace dpstoff1 = 5031826 if inlist(dpstoff1, 5000018, 5022426)
Pass 26 through the data...
  smallest count = 3 in the cell      5025053
  Invoking rule 47:49:25053=44753
  replace dpstoff1 = 5044753 if inlist(dpstoff1, 5000047, 5000049, 5025053)
Pass 27 through the data...
  smallest count = 4 in the cell      3000036
  Invoking rule 36:33944=43644
  replace dpstoff1 = 3043644 if inlist(dpstoff1, 3000036, 3033944)
Pass 28 through the data...
  smallest count = 4 in the cell      4025355
  Invoking rule 25355:60:62:68=55368
  replace dpstoff1 = 4055368 if inlist(dpstoff1, 4025355, 4000060, 4000062, 4000068)
Pass 29 through the data...
  smallest count = 4 in the cell      4031826
  Invoking rule 11:31826=41126
  replace dpstoff1 = 4041126 if inlist(dpstoff1, 4000011, 4031826)
Pass 30 through the data...
  smallest count = 4 in the cell      5043644
  Invoking rule 30:43644=53044
  replace dpstoff1 = 5053044 if inlist(dpstoff1, 5000030, 5043644)
Pass 31 through the data...
  smallest count = 5 in the cell      1000060
  Invoking rule 24950:25355:60=54960
  replace dpstoff1 = 1054960 if inlist(dpstoff1, 1024950, 1025355, 1000060)
Pass 32 through the data...
  smallest count = 5 in the cell      3000055
  Invoking rule 55:34953=44955
  replace dpstoff1 = 3044955 if inlist(dpstoff1, 3000055, 3034953)
Pass 33 through the data...
  smallest count = 5 in the cell      4055062
  Invoking rule 55062:68:69=75069
  replace dpstoff1 = 4075069 if inlist(dpstoff1, 4055062, 4000068, 4000069)
Pass 34 through the data...
  smallest count = 6 in the cell      1000055
  Invoking rule 53:55=25355
  replace dpstoff1 = 1025355 if inlist(dpstoff1, 1000053, 1000055)
Pass 35 through the data...
  smallest count = 6 in the cell      2050826
  Invoking rule 50826:133068=180868
  replace dpstoff1 = 2180868 if inlist(dpstoff1, 2050826, 2133068)
Pass 36 through the data...
  smallest count = 6 in the cell      5020811
  Invoking rule 20811:31826=50826
  replace dpstoff1 = 5050826 if inlist(dpstoff1, 5020811, 5031826)
Pass 37 through the data...
  smallest count = 7 in the cell      5000068
  Invoking rule 62:68=26268
  replace dpstoff1 = 5026268 if inlist(dpstoff1, 5000062, 5000068)
Pass 38 through the data...
  smallest count = 8 in the cell      2044753
  WARNING: could not find any rules to collapse dpstoff1 == 2044753
Pass 39 through the data...
  smallest count = 8 in the cell      4053647
  Invoking rule 30:53647=63047
  replace dpstoff1 = 4063047 if inlist(dpstoff1, 4000030, 4053647)
Pass 40 through the data...
  smallest count = 9 in the cell      5044753
  Invoking rule 53044:44753=93053
  replace dpstoff1 = 5093053 if inlist(dpstoff1, 5053044, 5044753)
Pass 41 through the data...
  smallest count = 10 in the cell      1054960
  Invoking rule 62:54960=64962
  replace dpstoff1 = 1064962 if inlist(dpstoff1, 1000062, 1054960)
Pass 42 through the data...
  smallest count = 10 in the cell      3022426
  Invoking rule 18:22426=31826
  replace dpstoff1 = 3031826 if inlist(dpstoff1, 3000018, 3022426)
Pass 43 through the data...
  smallest count = 10 in the cell      3043644
  Invoking rule 30:43644=53044
  replace dpstoff1 = 3053044 if inlist(dpstoff1, 3000030, 3043644)
Pass 44 through the data...
  smallest count = 10 in the cell      4041126
  Invoking rule 41126:63047=101147
  replace dpstoff1 = 4101147 if inlist(dpstoff1, 4041126, 4063047)
Pass 45 through the data...
  smallest count = 10 in the cell      4055368
  WARNING: could not find any rules to collapse dpstoff1 == 4055368
Pass 46 through the data...
  smallest count = 12 in the cell      1020208
  Invoking rule 20208:21118:24=50224
  replace dpstoff1 = 1050224 if inlist(dpstoff1, 1020208, 1021118, 1000024)
Pass 47 through the data...
  smallest count = 12 in the cell      1023639
  Invoking rule 23639:40:44=43644
  replace dpstoff1 = 1043644 if inlist(dpstoff1, 1023639, 1000040, 1000044)
Pass 48 through the data...
  smallest count = 13 in the cell      2180868
  Invoking rule 69:180868=190869
  replace dpstoff1 = 2190869 if inlist(dpstoff1, 2000069, 2180868)
Pass 49 through the data...
  smallest count = 13 in the cell      5050826
  Invoking rule 50826:93053=140853
  replace dpstoff1 = 5140853 if inlist(dpstoff1, 5050826, 5093053)
Pass 50 through the data...
  smallest count = 15 in the cell      1000026
  Invoking rule 26:50224=60226
  replace dpstoff1 = 1060226 if inlist(dpstoff1, 1000026, 1050224)
Pass 51 through the data...
  smallest count = 15 in the cell      3020811
  Invoking rule 20811:31826=50826
  replace dpstoff1 = 3050826 if inlist(dpstoff1, 3020811, 3031826)
Pass 52 through the data...
  smallest count = 15 in the cell      3044955
  Invoking rule 44955:60:62=64962
  replace dpstoff1 = 3064962 if inlist(dpstoff1, 3044955, 3000060, 3000062)
Pass 53 through the data...
  smallest count = 16 in the cell      5026268
  Invoking rule 69:26268=36269
  replace dpstoff1 = 5036269 if inlist(dpstoff1, 5000069, 5026268)
Pass 54 through the data...
  smallest count = 22 in the cell      3000047
  Done collapsing! Exiting...
{\smallskip}
. return list
{\smallskip}
scalars:
         r(arg_min_id) =  3000047
                r(min) =  22
{\smallskip}
macros:
            r(cfailed) : "2044753,4055368"
             r(failed) : "2044753 4055368"
\nullskip
\end{stlog}

The collapsed values of the variables \stcmd{dpston} (DayPart-STation-ON) and
\stcmd{dpstoff} (DayPart-STation-OFF) combine the values of the parent variables. The value
of \stcmd{dpston==1000003} indicates \stcmd{daypart==1} and station ID 3.
The value of \stcmd{dpston==2065270} indicates \stcmd{daypart==2} and sequence of
six stations from 52 to 70.
\label{page:dpston:nomenclature}

Note that \stcmd{wgtcellcollapse} returns a list of the cells that it could not
collapse in \stcmd{r(failed)} macro (and a comma delimited list, in \stcmd{f(cfailed)}).
These returned values should be used in production code by making an \stcmd{assert}
\citep{gould:2003:tip3} that these macros are empty.
While we know that some cell counts are less than 20, we will ignore the issue
for the moment, as there are bigger concerns with the collapsed cells at the moment,
as will become clear once we follow through with the workflow and attempt raking.

From the above run, \stcmd{wgtcellcollapse} produced two files, one for each
weighting margin, called \stcmd{dpston.do} and \stcmd{dpstoff.do}. An interested reader
is welcome to \stcmd{list} them; they contain long sequences of \stcmd{replace}
commands to perform the cell collapsing. The point of creating these is that they
can be run on the population data to create identical categories:

\begin{stlog}
. use trip_population, clear
{\smallskip}
. run dpston1.do
{\smallskip}
. total num_pass , over(dpston1)
{\smallskip}
Total estimation                  Number of obs   =        719
{\smallskip}
      1000001: dpston1 = 1000001
      1000002: dpston1 = 1000002
      1000008: dpston1 = 1000008
      1000011: dpston1 = 1000011
      1000018: dpston1 = 1000018
      1000024: dpston1 = 1000024
      1000026: dpston1 = 1000026
      1000030: dpston1 = 1000030
      1000036: dpston1 = 1000036
      1000044: dpston1 = 1000044
      1000047: dpston1 = 1000047
      1000049: dpston1 = 1000049
      1000062: dpston1 = 1000062
      1000068: dpston1 = 1000068
      1023940: dpston1 = 1023940
      1025053: dpston1 = 1025053
      1025560: dpston1 = 1025560
      2000011: dpston1 = 2000011
      2065068: dpston1 = 2065068
      2070126: dpston1 = 2070126
      2110847: dpston1 = 2110847
      2123062: dpston1 = 2123062
      3000008: dpston1 = 3000008
      3000011: dpston1 = 3000011
      3000018: dpston1 = 3000018
      3000024: dpston1 = 3000024
      3000030: dpston1 = 3000030
      3000036: dpston1 = 3000036
      3000047: dpston1 = 3000047
      3000068: dpston1 = 3000068
      3020102: dpston1 = 3020102
      3025355: dpston1 = 3025355
      3062644: dpston1 = 3062644
      3064962: dpston1 = 3064962
      4030108: dpston1 = 4030108
      4041830: dpston1 = 4041830
      4084768: dpston1 = 4084768
      4131153: dpston1 = 4131153
      5030108: dpston1 = 5030108
      5041830: dpston1 = 5041830
      5053647: dpston1 = 5053647
      5104068: dpston1 = 5104068
      5161162: dpston1 = 5161162
{\smallskip}
\HLI{13}{\TOPT}\HLI{48}
        Over {\VBAR}      Total   Std. Err.     [95\% Conf. Interval]
\HLI{13}{\PLUS}\HLI{48}
num_pass     {\VBAR}
     1000001 {\VBAR}       1423   967.7508     -476.9595    3322.959
     1000002 {\VBAR}       7198    4895.91     -2414.011    16810.01
     1000008 {\VBAR}      19254   13675.81     -7595.347    46103.35
     1000011 {\VBAR}      12626   9682.022     -6382.456    31634.46
     1000018 {\VBAR}       2470   1943.224     -1345.081    6285.081
     1000024 {\VBAR}        634   509.3549     -366.0031    1634.003
     1000026 {\VBAR}       2208   1774.996     -1276.802    5692.802
     1000030 {\VBAR}       4319   3665.427     -2877.235    11515.24
     1000036 {\VBAR}       1221   1046.817     -834.1873    3276.187
     1000044 {\VBAR}       1021    881.426     -709.4802     2751.48
     1000047 {\VBAR}       3300   2970.321     -2531.552    9131.552
     1000049 {\VBAR}         38         35     -30.71457    106.7146
     1000062 {\VBAR}       3402       3176     -2833.357    9637.357
     1000068 {\VBAR}       5085          .             .           .
     1023940 {\VBAR}        491    348.709     -193.6112    1175.611
     1025053 {\VBAR}       1248   765.1955     -254.2881    2750.288
     1025560 {\VBAR}        601     350.65     -87.42178    1289.422
     2000011 {\VBAR}        872    675.393     -453.9812    2197.981
     2065068 {\VBAR}        177   69.36426        40.819     313.181
     2070126 {\VBAR}        708    299.066      120.8517    1295.148
     2110847 {\VBAR}       1110   711.7168     -287.2948    2507.295
     2123062 {\VBAR}        690   412.3311     -119.5187    1499.519
     3000008 {\VBAR}       3739   2665.175     -1493.467    8971.467
     3000011 {\VBAR}       3476   2669.777     -1765.503    8717.503
     3000018 {\VBAR}       1263    997.019     -694.4209    3220.421
     3000024 {\VBAR}       1296   1032.175     -730.4418    3322.442
     3000030 {\VBAR}       3740   3175.677     -2494.723    9974.723
     3000036 {\VBAR}        444   382.5382     -307.0272    1195.027
     3000047 {\VBAR}        984   888.5095     -760.3871    2728.387
     3000068 {\VBAR}        744          .             .           .
     3020102 {\VBAR}        992   553.0017     -93.69354    2077.694
     3025355 {\VBAR}        796   573.1597     -329.2692    1921.269
     3062644 {\VBAR}        708   375.9286      -30.0507    1446.051
     3064962 {\VBAR}       1284    911.761     -506.0362    3074.036
     4030108 {\VBAR}       1154   529.0201      115.3888    2192.611
     4041830 {\VBAR}        715   393.6075     -57.75914    1487.759
     4084768 {\VBAR}        651   318.6053      25.49059    1276.509
     4131153 {\VBAR}       1169   534.4403      119.7475    2218.253
     5030108 {\VBAR}        610   263.2061      93.25444    1126.746
     5041830 {\VBAR}        455   172.8013      115.7439    794.2561
     5053647 {\VBAR}        474   283.9144     -83.40157    1031.402
     5104068 {\VBAR}        270   116.7702      40.74822    499.2518
     5161162 {\VBAR}       1723   909.6551     -62.90172    3508.902
\HLI{13}{\BOTT}\HLI{48}
{\smallskip}
. matrix dpston1 = e(b)
{\smallskip}
. matrix coleq dpston1 = _one
{\smallskip}
. matrix rownames dpston1 = dpston1
{\smallskip}
. run dpstoff1.do
{\smallskip}
. total num_pass , over(dpstoff1)
{\smallskip}
Total estimation                  Number of obs   =        719
{\smallskip}
      1000011: dpstoff1 = 1000011
      1000018: dpstoff1 = 1000018
      1000030: dpstoff1 = 1000030
      1000047: dpstoff1 = 1000047
      1000068: dpstoff1 = 1000068
      1000069: dpstoff1 = 1000069
      1025355: dpstoff1 = 1025355
      1043644: dpstoff1 = 1043644
      1060226: dpstoff1 = 1060226
      1064962: dpstoff1 = 1064962
      2044753: dpstoff1 = 2044753
      2190869: dpstoff1 = 2190869
      3000002: dpstoff1 = 3000002
      3000047: dpstoff1 = 3000047
      3000068: dpstoff1 = 3000068
      3000069: dpstoff1 = 3000069
      3050826: dpstoff1 = 3050826
      3053044: dpstoff1 = 3053044
      3064962: dpstoff1 = 3064962
      4000002: dpstoff1 = 4000002
      4000008: dpstoff1 = 4000008
      4000049: dpstoff1 = 4000049
      4055368: dpstoff1 = 4055368
      4075069: dpstoff1 = 4075069
      4101147: dpstoff1 = 4101147
      5000055: dpstoff1 = 5000055
      5000060: dpstoff1 = 5000060
      5036269: dpstoff1 = 5036269
      5140853: dpstoff1 = 5140853
{\smallskip}
\HLI{13}{\TOPT}\HLI{48}
        Over {\VBAR}      Total   Std. Err.     [95\% Conf. Interval]
\HLI{13}{\PLUS}\HLI{48}
num_pass     {\VBAR}
     1000011 {\VBAR}       2475   1468.807     -408.6691    5358.669
     1000018 {\VBAR}        929   360.7303      220.7878    1637.212
     1000030 {\VBAR}       2189   868.0319      484.8161    3893.184
     1000047 {\VBAR}       1746   630.7528      507.6598     2984.34
     1000068 {\VBAR}       1372   426.3969      534.8662    2209.134
     1000069 {\VBAR}      53193   15995.88      21788.72    84597.28
     1025355 {\VBAR}        863   233.1424      405.2777    1320.722
     1043644 {\VBAR}        737   159.5597      423.7407    1050.259
     1060226 {\VBAR}       1491   432.5204      641.8441    2340.156
     1064962 {\VBAR}       1544   426.4228      706.8155    2381.185
     2044753 {\VBAR}        124   33.15528      58.90711    189.0929
     2190869 {\VBAR}       3433   1082.925      1306.921    5559.079
     3000002 {\VBAR}          3          .             .           .
     3000047 {\VBAR}        556   187.4945      187.8971    924.1029
     3000068 {\VBAR}        444   126.0503      196.5289    691.4711
     3000069 {\VBAR}      16007   4295.998      7572.781    24441.22
     3050826 {\VBAR}        910   333.6639      254.9266    1565.073
     3053044 {\VBAR}        787    249.935      296.3092    1277.691
     3064962 {\VBAR}        759   141.1029      481.9765    1036.023
     4000002 {\VBAR}          2          .             .           .
     4000008 {\VBAR}         23          5      13.18363    32.81637
     4000049 {\VBAR}          3          0             .           .
     4055368 {\VBAR}        172   35.93822      101.4434    242.5566
     4075069 {\VBAR}       2841   833.6481      1204.321    4477.679
     4101147 {\VBAR}        648    147.123      359.1573    936.8427
     5000055 {\VBAR}         14   6.595453      1.051322    26.94868
     5000060 {\VBAR}          4          2      .0734531    7.926547
     5036269 {\VBAR}       2880   980.8909      954.2428    4805.757
     5140853 {\VBAR}        634   139.2172      360.6787    907.3213
\HLI{13}{\BOTT}\HLI{48}
{\smallskip}
. matrix dpstoff1 = e(b)
{\smallskip}
. matrix coleq dpstoff1 = _one
{\smallskip}
. matrix rownames dpstoff1 = dpstoff1
{\smallskip}
\nullskip
\end{stlog}

Once that is done, we can go back to the sample data and try to create raking weights:

\begin{stlog}
. use trip_sample, clear
{\smallskip}
. run dpston1
{\smallskip}
. run dpstoff1
{\smallskip}
. gen byte _one = 1       
{\smallskip}
. ipfraking [pw=_one], ctotal(dpston1 dpstoff1) gen(raked_weight1)
{\smallskip}
categories of dpston1 do not match in the control dpston1 and in the data (nolab opt
> ion)
This is what dpston1 gives: 
  _one:1000001 _one:1000002 _one:1000008 _one:1000011 _one:1000018 _one:1000024 _one
> :1000026 _one:1000030 _one:1000036 _one:1000044 _one:1000047 _one:1000049 _one:100
> 0062 _one:1000068 _one:1023940 _one:1025053 _one:1025560 _one:2000011 _one:2065068
>  _one:2070126 _one:2110847 _one:2123062 _one:3000008 _one:3000011 _one:3000018 _on
> e:3000024 _one:3000030 _one:3000036 _one:3000047 _one:3000068 _one:3020102 _one:30
> 25355 _one:3062644 _one:3064962 _one:4030108 _one:4041830 _one:4084768 _one:413115
> 3 _one:5030108 _one:5041830 _one:5053647 _one:5104068 _one:5161162
This is what I found in data: 
  _one:1000001 _one:1000002 _one:1000008 _one:1000011 _one:1000018 _one:1000024 _one
> :1000026 _one:1000030 _one:1000036 _one:1000044 _one:1000047 _one:1000062 _one:100
> 0068 _one:1023940 _one:1025053 _one:1025560 _one:2000011 _one:2065068 _one:2070126
>  _one:2110847 _one:2123062 _one:3000008 _one:3000011 _one:3000018 _one:3000024 _on
> e:3000030 _one:3000036 _one:3000047 _one:3000068 _one:3020102 _one:3025355 _one:30
> 62644 _one:3064962 _one:4030108 _one:4041830 _one:4084768 _one:4131153 _one:503010
> 8 _one:5041830 _one:5053647 _one:5104068 _one:5161162
This is what dpston1 has that data don't: 
  _one:1000049
This is what data have that dpston1 doesn't: 
  
r(111);
{\smallskip}
end of do-file
{\smallskip}
r(111);
{\smallskip}
. ipfraking [pw=_one], ctotal(dpstoff1 dpston1) gen(raked_weight1)
{\smallskip}
categories of dpstoff1 do not match in the control dpstoff1 and in the data (nolab o
> ption)
This is what dpstoff1 gives: 
  _one:1000011 _one:1000018 _one:1000030 _one:1000047 _one:1000068 _one:1000069 _one
> :1025355 _one:1043644 _one:1060226 _one:1064962 _one:2044753 _one:2190869 _one:300
> 0002 _one:3000047 _one:3000068 _one:3000069 _one:3050826 _one:3053044 _one:3064962
>  _one:4000002 _one:4000008 _one:4000049 _one:4055368 _one:4075069 _one:4101147 _on
> e:5000055 _one:5000060 _one:5036269 _one:5140853
This is what I found in data: 
  _one:1000011 _one:1000018 _one:1000030 _one:1000047 _one:1000068 _one:1000069 _one
> :1025355 _one:1043644 _one:1060226 _one:1064962 _one:2044753 _one:2190869 _one:300
> 0047 _one:3000068 _one:3000069 _one:3050826 _one:3053044 _one:3064962 _one:4055368
>  _one:4075069 _one:4101147 _one:5036269 _one:5140853
This is what dpstoff1 has that data don't: 
  _one:3000002 _one:4000002 _one:4000008 _one:4000049 _one:5000055 _one:5000060
This is what data have that dpstoff1 doesn't: 
  
r(111);
{\smallskip}
. 
\nullskip
\end{stlog}

We see that raking failed, because survey nonresponse wiped out some of the smaller
stations from the sample. (Note also the informative error message with
diagnostics of missing categories produced by \stcmd{ipfraking}. This is a functionality
added since the first 2010 publication in \textit{The Stata Journal}. The message lists
the categories found in the data, in the control totals, and in the mismatch.)

\subsubsection{The second pass of cell collapse and raking: \stcmd{zeroes()} option}

Having identified the issue, we can overcome it with \stcmd{zeroes()} option
of \stcmd{wgtcellcollapse collapse} whose purpose is specifically to add missing categories.
This option provides the list of stations that may have zero sample counts
in a given daypart.
For instance, notice that the sample registers only one alighting at Brookline (2)
in AM Peak daypart, even though there are passengers exiting in other dayparts. All in all,
\stcmd{wgtcellcollapse} needs to be made aware of the zero sample boardings at
Johnsville (39), King Street (40), Limerick (44), Ninth Street (49),
Queens Zoo (55) and Redline Circle (60); as well as zero alightings at Brookline (2),
Carmenton (8), Irvingtown (36),
Johnsville (39), King Street (40), Limerick (44), Moscow City (47), Ninth Street (49),
Ontario Lake (50), Queens Zoo (55), Redline Circle (60), and Silver Spring (62).

\begin{stlog}
. use trip_sample_rules, clear
{\smallskip}
. wgtcellcollapse collapse, variables(daypart board_id) mincellsize(20) ///
>         zeroes(39 40 44 49 55 60) ///
>         generate(dpston2) saving(dpston2.do) replace run
Pass 0 through the data...
  smallest count = 1 in the cell      2000039
{\smallskip}
Processing zero cells...
{\smallskip}
  Invoking rule 49:50=24950 to collapse zero cells
  replace dpston2 = 1024950 if inlist(dpston2, 1000049, 1000050)
Pass 0 through the data...
  smallest count = 1 in the cell      2000039
  Invoking rule 40:44=24044 to collapse zero cells
  replace dpston2 = 2024044 if inlist(dpston2, 2000040, 2000044)
Pass 0 through the data...
  smallest count = 1 in the cell      2000039
  Invoking rule 49:50=24950 to collapse zero cells
  replace dpston2 = 2024950 if inlist(dpston2, 2000049, 2000050)
Pass 0 through the data...
  smallest count = 1 in the cell      2000039
  Invoking rule 55:60=25560 to collapse zero cells
  replace dpston2 = 2025560 if inlist(dpston2, 2000055, 2000060)
Pass 0 through the data...
  smallest count = 1 in the cell      2000039
  Invoking rule 36:39=23639 to collapse zero cells
  replace dpston2 = 4023639 if inlist(dpston2, 4000036, 4000039)
Pass 0 through the data...
  smallest count = 1 in the cell      2000039
  Invoking rule 40:44=24044 to collapse zero cells
  replace dpston2 = 4024044 if inlist(dpston2, 4000040, 4000044)
Pass 0 through the data...
  smallest count = 1 in the cell      2000039
  Invoking rule 49:50=24950 to collapse zero cells
  replace dpston2 = 4024950 if inlist(dpston2, 4000049, 4000050)
Pass 0 through the data...
  smallest count = 1 in the cell      2000039
  Invoking rule 53:55=25355 to collapse zero cells
  replace dpston2 = 4025355 if inlist(dpston2, 4000053, 4000055)
Pass 0 through the data...
  smallest count = 1 in the cell      2000039
  Invoking rule 24950:53:55:60=54960 to collapse zero cells
  replace dpston2 = 4054960 if inlist(dpston2, 4024950, 4000053, 4000055, 4000060)
Pass 0 through the data...
  smallest count = 1 in the cell      2000039
  Invoking rule 39:40=23940 to collapse zero cells
  replace dpston2 = 5023940 if inlist(dpston2, 5000039, 5000040)
Pass 0 through the data...
  smallest count = 1 in the cell      2000039
  Invoking rule 49:50=24950 to collapse zero cells
  replace dpston2 = 5024950 if inlist(dpston2, 5000049, 5000050)
Pass 0 through the data...
  smallest count = 1 in the cell      2000039
  Invoking rule 24950:25355:60=54960 to collapse zero cells
  replace dpston2 = 5054960 if inlist(dpston2, 5024950, 5025355, 5000060)
Pass 0 through the data...
  smallest count = 1 in the cell      2000039
Pass 12 through the data...
  smallest count = 1 in the cell      2000039
  Invoking rule 39:24044=33944
  replace dpston2 = 2033944 if inlist(dpston2, 2000039, 2024044)
Pass 13 through the data...
  smallest count = 1 in the cell      2024950
  Invoking rule 53:24950=34953
  replace dpston2 = 2034953 if inlist(dpston2, 2000053, 2024950)
Pass 14 through the data...
  smallest count = 1 in the cell      2025560
  Invoking rule 34953:25560=54960
  replace dpston2 = 2054960 if inlist(dpston2, 2034953, 2025560)
Pass 15 through the data...
  smallest count = 1 in the cell      3000039
  Invoking rule 39:40=23940
  replace dpston2 = 3023940 if inlist(dpston2, 3000039, 3000040)
Pass 16 through the data...
  smallest count = 1 in the cell      4023639
  Invoking rule 23639:24044=43644
  replace dpston2 = 4043644 if inlist(dpston2, 4023639, 4024044)
Pass 17 through the data...
  smallest count = 1 in the cell      4054960
  Invoking rule 47:54960=64760
  replace dpston2 = 4064760 if inlist(dpston2, 4000047, 4054960)
Pass 18 through the data...
  smallest count = 1 in the cell      5000024
  Invoking rule 18:24=21824
  replace dpston2 = 5021824 if inlist(dpston2, 5000018, 5000024)
Pass 19 through the data...
  smallest count = 1 in the cell      5023940
  Invoking rule 44:23940=33944
  replace dpston2 = 5033944 if inlist(dpston2, 5000044, 5023940)
Pass 20 through the data...
  smallest count = 1 in the cell      5054960
  Invoking rule 47:54960=64760
  replace dpston2 = 5064760 if inlist(dpston2, 5000047, 5054960)
Pass 21 through the data...
  smallest count = 2 in the cell      2000036
  Invoking rule 36:33944=43644
  replace dpston2 = 2043644 if inlist(dpston2, 2000036, 2033944)
Pass 22 through the data...
  smallest count = 2 in the cell      4043644
  Invoking rule 24:22630:43644=72444
  replace dpston2 = 4072444 if inlist(dpston2, 4000024, 4022630, 4043644)
Pass 23 through the data...
  smallest count = 2 in the cell      5000055
  Invoking rule 53:55=25355
  replace dpston2 = 5025355 if inlist(dpston2, 5000053, 5000055)
Pass 24 through the data...
  smallest count = 3 in the cell      2000024
  Invoking rule 24:22630:43644=72444
  replace dpston2 = 2072444 if inlist(dpston2, 2000024, 2022630, 2043644)
Pass 25 through the data...
  smallest count = 3 in the cell      3023940
  Invoking rule 44:23940=33944
  replace dpston2 = 3033944 if inlist(dpston2, 3000044, 3023940)
Pass 26 through the data...
  smallest count = 3 in the cell      5000001
  Invoking rule 1:2=20102
  replace dpston2 = 5020102 if inlist(dpston2, 5000001, 5000002)
Pass 27 through the data...
  smallest count = 3 in the cell      5000068
  Invoking rule 25355:26062:68=55368
  replace dpston2 = 5055368 if inlist(dpston2, 5025355, 5026062, 5000068)
Pass 28 through the data...
  smallest count = 4 in the cell      2000001
  Invoking rule 1:2=20102
  replace dpston2 = 2020102 if inlist(dpston2, 2000001, 2000002)
Pass 29 through the data...
  smallest count = 4 in the cell      2000008
  Invoking rule 8:21118:72444=100844
  replace dpston2 = 2100844 if inlist(dpston2, 2000008, 2021118, 2072444)
Pass 30 through the data...
  smallest count = 4 in the cell      3000050
  Invoking rule 49:50=24950
  replace dpston2 = 3024950 if inlist(dpston2, 3000049, 3000050)
Pass 31 through the data...
  smallest count = 4 in the cell      4000018
  Invoking rule 18:24:26=31826
  replace dpston2 = 4031826 if inlist(dpston2, 4000018, 4000024, 4000026)
Pass 32 through the data...
  smallest count = 4 in the cell      5033944
  Invoking rule 26:23036:33944=62644
  replace dpston2 = 5062644 if inlist(dpston2, 5000026, 5023036, 5033944)
Pass 33 through the data...
  smallest count = 5 in the cell      1000039
  Invoking rule 39:40=23940
  replace dpston2 = 1023940 if inlist(dpston2, 1000039, 1000040)
Pass 34 through the data...
  smallest count = 5 in the cell      2000018
  Invoking rule 20102:20811:18=50118
  replace dpston2 = 2050118 if inlist(dpston2, 2020102, 2020811, 2000018)
Pass 35 through the data...
  smallest count = 5 in the cell      3000060
  Invoking rule 24950:25355:60=54960
  replace dpston2 = 3054960 if inlist(dpston2, 3024950, 3025355, 3000060)
Pass 36 through the data...
  smallest count = 5 in the cell      4072444
  Invoking rule 72444:64760=132460
  replace dpston2 = 4132460 if inlist(dpston2, 4072444, 4064760)
Pass 37 through the data...
  smallest count = 5 in the cell      5021824
  Invoking rule 21824:62644=81844
  replace dpston2 = 5081844 if inlist(dpston2, 5021824, 5062644)
Pass 38 through the data...
  smallest count = 6 in the cell      2000068
  Invoking rule 62:68=26268
  replace dpston2 = 2026268 if inlist(dpston2, 2000062, 2000068)
Pass 39 through the data...
  smallest count = 6 in the cell      2054960
  Invoking rule 54960:26268=74968
  replace dpston2 = 2074968 if inlist(dpston2, 2054960, 2026268)
Pass 40 through the data...
  smallest count = 6 in the cell      4000002
  Invoking rule 1:2=20102
  replace dpston2 = 4020102 if inlist(dpston2, 4000001, 4000002)
Pass 41 through the data...
  smallest count = 7 in the cell      4025355
  Invoking rule 25355:26062:68=55368
  replace dpston2 = 4055368 if inlist(dpston2, 4025355, 4026062, 4000068)
Pass 42 through the data...
  smallest count = 9 in the cell      4031826
  Invoking rule 30:31826=41830
  replace dpston2 = 4041830 if inlist(dpston2, 4000030, 4031826)
Pass 43 through the data...
  smallest count = 10 in the cell      1000055
  Invoking rule 55:60=25560
  replace dpston2 = 1025560 if inlist(dpston2, 1000055, 1000060)
Pass 44 through the data...
  smallest count = 10 in the cell      5020102
  Invoking rule 8:20102=30108
  replace dpston2 = 5030108 if inlist(dpston2, 5000008, 5020102)
Pass 45 through the data...
  smallest count = 10 in the cell      5055368
  WARNING: could not find any rules to collapse dpston2 == 5055368
Pass 46 through the data...
  smallest count = 11 in the cell      2100844
  Invoking rule 47:100844=110847
  replace dpston2 = 2110847 if inlist(dpston2, 2000047, 2100844)
Pass 47 through the data...
  smallest count = 11 in the cell      3000001
  Invoking rule 1:2=20102
  replace dpston2 = 3020102 if inlist(dpston2, 3000001, 3000002)
Pass 48 through the data...
  smallest count = 12 in the cell      3033944
  Invoking rule 26:23036:33944=62644
  replace dpston2 = 3062644 if inlist(dpston2, 3000026, 3023036, 3033944)
Pass 49 through the data...
  smallest count = 12 in the cell      4000062
  Invoking rule 62:132460=142462
  replace dpston2 = 4142462 if inlist(dpston2, 4000062, 4132460)
Pass 50 through the data...
  smallest count = 12 in the cell      5000030
  Invoking rule 30:36=23036
  replace dpston2 = 5023036 if inlist(dpston2, 5000030, 5000036)
Pass 51 through the data...
  smallest count = 13 in the cell      1024950
  Invoking rule 53:24950=34953
  replace dpston2 = 1034953 if inlist(dpston2, 1000053, 1024950)
Pass 52 through the data...
  smallest count = 13 in the cell      2000026
  Invoking rule 50118:24:26=70126
  replace dpston2 = 2070126 if inlist(dpston2, 2050118, 2000024, 2000026)
Pass 53 through the data...
  smallest count = 13 in the cell      4020102
  Invoking rule 8:20102=30108
  replace dpston2 = 4030108 if inlist(dpston2, 4000008, 4020102)
Pass 54 through the data...
  smallest count = 14 in the cell      3000055
  Invoking rule 53:55=25355
  replace dpston2 = 3025355 if inlist(dpston2, 3000053, 3000055)
Pass 55 through the data...
  smallest count = 14 in the cell      5064760
  Invoking rule 81844:64760=141860
  replace dpston2 = 5141860 if inlist(dpston2, 5081844, 5064760)
Pass 56 through the data...
  smallest count = 17 in the cell      5000062
  Invoking rule 62:141860=151862
  replace dpston2 = 5151862 if inlist(dpston2, 5000062, 5141860)
Pass 57 through the data...
  smallest count = 18 in the cell      3054960
  Invoking rule 62:54960=64962
  replace dpston2 = 3064962 if inlist(dpston2, 3000062, 3054960)
Pass 58 through the data...
  smallest count = 18 in the cell      4055368
  WARNING: could not find any rules to collapse dpston2 == 4055368
Pass 59 through the data...
  smallest count = 20 in the cell      2000030
  Done collapsing! Exiting...
{\smallskip}
. return list
{\smallskip}
scalars:
         r(arg_min_id) =  2000030
                r(min) =  20
{\smallskip}
macros:
            r(cfailed) : "5055368,4055368"
             r(failed) : "5055368 4055368"
{\smallskip}
. wgtcellcollapse collapse, variables(daypart alight_id) mincellsize(20) ///
>         zeroes(2 8 36 39 40 44 47 49 50 55 60 62) ///
>         generate(dpstoff2) saving(dpstoff2.do) replace run
Pass 0 through the data...
  smallest count = 1 in the cell      1000002
{\smallskip}
Processing zero cells...
{\smallskip}
  Invoking rule 39:40=23940 to collapse zero cells
  replace dpstoff2 = 1023940 if inlist(dpstoff2, 1000039, 1000040)
Pass 0 through the data...
  smallest count = 1 in the cell      1000002
  Invoking rule 2:8=20208 to collapse zero cells
  replace dpstoff2 = 2020208 if inlist(dpstoff2, 2000002, 2000008)
Pass 0 through the data...
  smallest count = 1 in the cell      1000002
  Invoking rule 30:36=23036 to collapse zero cells
  replace dpstoff2 = 2023036 if inlist(dpstoff2, 2000030, 2000036)
Pass 0 through the data...
  smallest count = 1 in the cell      1000002
  Invoking rule 26:30:36:39=42639 to collapse zero cells
  replace dpstoff2 = 2042639 if inlist(dpstoff2, 2000026, 2000030, 2000036, 2000039)
Pass 0 through the data...
  smallest count = 1 in the cell      1000002
  Invoking rule 24:26:30:36:39:40=62440 to collapse zero cells
  replace dpstoff2 = 2062440 if inlist(dpstoff2, 2000024, 2000026, 2000030, 2000036, 2000039, 2000040)
Pass 0 through the data...
  smallest count = 1 in the cell      1000002
  Invoking rule 44:62440=72444 to collapse zero cells
  replace dpstoff2 = 2072444 if inlist(dpstoff2, 2000044, 2062440)
Pass 0 through the data...
  smallest count = 1 in the cell      1000002
  Invoking rule 49:50=24950 to collapse zero cells
  replace dpstoff2 = 2024950 if inlist(dpstoff2, 2000049, 2000050)
Pass 0 through the data...
  smallest count = 1 in the cell      1000002
  Invoking rule 53:55=25355 to collapse zero cells
  replace dpstoff2 = 2025355 if inlist(dpstoff2, 2000053, 2000055)
Pass 0 through the data...
  smallest count = 1 in the cell      1000002
  Invoking rule 60:25355=35360 to collapse zero cells
  replace dpstoff2 = 2035360 if inlist(dpstoff2, 2000060, 2025355)
Pass 0 through the data...
  smallest count = 1 in the cell      1000002
  Invoking rule 62:35360=45362 to collapse zero cells
  replace dpstoff2 = 2045362 if inlist(dpstoff2, 2000062, 2035360)
Pass 0 through the data...
  smallest count = 1 in the cell      1000002
  Invoking rule 2:8=20208 to collapse zero cells
  replace dpstoff2 = 3020208 if inlist(dpstoff2, 3000002, 3000008)
Pass 0 through the data...
  smallest count = 1 in the cell      1000002
  Invoking rule 55:60=25560 to collapse zero cells
  replace dpstoff2 = 3025560 if inlist(dpstoff2, 3000055, 3000060)
Pass 0 through the data...
  smallest count = 1 in the cell      1000002
  Invoking rule 2:8:11=30211 to collapse zero cells
  replace dpstoff2 = 4030211 if inlist(dpstoff2, 4000002, 4000008, 4000011)
Pass 0 through the data...
  smallest count = 1 in the cell      1000002
  Invoking rule 36:39=23639 to collapse zero cells
  replace dpstoff2 = 4023639 if inlist(dpstoff2, 4000036, 4000039)
Pass 0 through the data...
  smallest count = 1 in the cell      1000002
  Invoking rule 40:23639=33640 to collapse zero cells
  replace dpstoff2 = 4033640 if inlist(dpstoff2, 4000040, 4023639)
Pass 0 through the data...
  smallest count = 1 in the cell      1000002
  Invoking rule 44:33640=43644 to collapse zero cells
  replace dpstoff2 = 4043644 if inlist(dpstoff2, 4000044, 4033640)
Pass 0 through the data...
  smallest count = 1 in the cell      1000002
  Invoking rule 49:50=24950 to collapse zero cells
  replace dpstoff2 = 4024950 if inlist(dpstoff2, 4000049, 4000050)
Pass 0 through the data...
  smallest count = 1 in the cell      1000002
  Invoking rule 55:60=25560 to collapse zero cells
  replace dpstoff2 = 4025560 if inlist(dpstoff2, 4000055, 4000060)
Pass 0 through the data...
  smallest count = 1 in the cell      1000002
  Invoking rule 2:8=20208 to collapse zero cells
  replace dpstoff2 = 5020208 if inlist(dpstoff2, 5000002, 5000008)
Pass 0 through the data...
  smallest count = 1 in the cell      1000002
  Invoking rule 36:39=23639 to collapse zero cells
  replace dpstoff2 = 5023639 if inlist(dpstoff2, 5000036, 5000039)
Pass 0 through the data...
  smallest count = 1 in the cell      1000002
  Invoking rule 40:44=24044 to collapse zero cells
  replace dpstoff2 = 5024044 if inlist(dpstoff2, 5000040, 5000044)
Pass 0 through the data...
  smallest count = 1 in the cell      1000002
  Invoking rule 49:50=24950 to collapse zero cells
  replace dpstoff2 = 5024950 if inlist(dpstoff2, 5000049, 5000050)
Pass 0 through the data...
  smallest count = 1 in the cell      1000002
  Invoking rule 53:55=25355 to collapse zero cells
  replace dpstoff2 = 5025355 if inlist(dpstoff2, 5000053, 5000055)
Pass 0 through the data...
  smallest count = 1 in the cell      1000002
  Invoking rule 24950:53:55:60=54960 to collapse zero cells
  replace dpstoff2 = 5054960 if inlist(dpstoff2, 5024950, 5000053, 5000055, 5000060)
Pass 0 through the data...
  smallest count = 1 in the cell      1000002
Pass 24 through the data...
  smallest count = 1 in the cell      1000002
  Invoking rule 2:8=20208
  replace dpstoff2 = 1020208 if inlist(dpstoff2, 1000002, 1000008)
Pass 25 through the data...
  smallest count = 1 in the cell      2000011
  Invoking rule 11:18=21118
  replace dpstoff2 = 2021118 if inlist(dpstoff2, 2000011, 2000018)
Pass 26 through the data...
  smallest count = 1 in the cell      2020208
  Invoking rule 20208:21118=40218
  replace dpstoff2 = 2040218 if inlist(dpstoff2, 2020208, 2021118)
Pass 27 through the data...
  smallest count = 1 in the cell      2024950
  Invoking rule 24950:45362=64962
  replace dpstoff2 = 2064962 if inlist(dpstoff2, 2024950, 2045362)
Pass 28 through the data...
  smallest count = 1 in the cell      2072444
  Invoking rule 40218:72444=110244
  replace dpstoff2 = 2110244 if inlist(dpstoff2, 2040218, 2072444)
Pass 29 through the data...
  smallest count = 1 in the cell      3000039
  Invoking rule 39:40=23940
  replace dpstoff2 = 3023940 if inlist(dpstoff2, 3000039, 3000040)
Pass 30 through the data...
  smallest count = 1 in the cell      3000049
  Invoking rule 49:50=24950
  replace dpstoff2 = 3024950 if inlist(dpstoff2, 3000049, 3000050)
Pass 31 through the data...
  smallest count = 1 in the cell      3020208
  Invoking rule 20208:21118:24=50224
  replace dpstoff2 = 3050224 if inlist(dpstoff2, 3020208, 3021118, 3000024)
Pass 32 through the data...
  smallest count = 1 in the cell      4000018
  Invoking rule 18:24=21824
  replace dpstoff2 = 4021824 if inlist(dpstoff2, 4000018, 4000024)
Pass 33 through the data...
  smallest count = 1 in the cell      4000026
  Invoking rule 26:21824=31826
  replace dpstoff2 = 4031826 if inlist(dpstoff2, 4000026, 4021824)
Pass 34 through the data...
  smallest count = 1 in the cell      4025560
  Invoking rule 53:25560=35360
  replace dpstoff2 = 4035360 if inlist(dpstoff2, 4000053, 4025560)
Pass 35 through the data...
  smallest count = 1 in the cell      5000026
  Invoking rule 24:26=22426
  replace dpstoff2 = 5022426 if inlist(dpstoff2, 5000024, 5000026)
Pass 36 through the data...
  smallest count = 1 in the cell      5020208
  Invoking rule 11:20208=30211
  replace dpstoff2 = 5030211 if inlist(dpstoff2, 5000011, 5020208)
Pass 37 through the data...
  smallest count = 1 in the cell      5054960
  Invoking rule 47:54960=64760
  replace dpstoff2 = 5064760 if inlist(dpstoff2, 5000047, 5054960)
Pass 38 through the data...
  smallest count = 2 in the cell      1000050
  Invoking rule 49:50=24950
  replace dpstoff2 = 1024950 if inlist(dpstoff2, 1000049, 1000050)
Pass 39 through the data...
  smallest count = 2 in the cell      2042639
  Invoking rule 42639:40:44:47=72647
  replace dpstoff2 = 2072647 if inlist(dpstoff2, 2042639, 2000040, 2000044, 2000047)
Pass 40 through the data...
  smallest count = 2 in the cell      2064962
  Invoking rule 68:64962=74968
  replace dpstoff2 = 2074968 if inlist(dpstoff2, 2000068, 2064962)
Pass 41 through the data...
  smallest count = 2 in the cell      3000044
  Invoking rule 44:23940=33944
  replace dpstoff2 = 3033944 if inlist(dpstoff2, 3000044, 3023940)
Pass 42 through the data...
  smallest count = 2 in the cell      3024950
  Invoking rule 53:24950=34953
  replace dpstoff2 = 3034953 if inlist(dpstoff2, 3000053, 3024950)
Pass 43 through the data...
  smallest count = 2 in the cell      4024950
  Invoking rule 24950:35360=54960
  replace dpstoff2 = 4054960 if inlist(dpstoff2, 4024950, 4035360)
Pass 44 through the data...
  smallest count = 2 in the cell      4043644
  Invoking rule 47:43644=53647
  replace dpstoff2 = 4053647 if inlist(dpstoff2, 4000047, 4043644)
Pass 45 through the data...
  smallest count = 2 in the cell      5023639
  Invoking rule 23639:24044=43644
  replace dpstoff2 = 5043644 if inlist(dpstoff2, 5023639, 5024044)
Pass 46 through the data...
  smallest count = 2 in the cell      5025355
  Invoking rule 25355:26062:68=55368
  replace dpstoff2 = 5055368 if inlist(dpstoff2, 5025355, 5026062, 5000068)
Pass 47 through the data...
  smallest count = 3 in the cell      1023940
  Invoking rule 36:23940=33640
  replace dpstoff2 = 1033640 if inlist(dpstoff2, 1000036, 1023940)
Pass 48 through the data...
  smallest count = 3 in the cell      3050224
  Invoking rule 50224:22630:36=80236
  replace dpstoff2 = 3080236 if inlist(dpstoff2, 3050224, 3022630, 3000036)
Pass 49 through the data...
  smallest count = 3 in the cell      4000062
  Invoking rule 62:68=26268
  replace dpstoff2 = 4026268 if inlist(dpstoff2, 4000062, 4000068)
Pass 50 through the data...
  smallest count = 3 in the cell      5022426
  Invoking rule 18:22426=31826
  replace dpstoff2 = 5031826 if inlist(dpstoff2, 5000018, 5022426)
Pass 51 through the data...
  smallest count = 4 in the cell      2023036
  WARNING: could not find any rules to collapse dpstoff2 == 2023036
Pass 52 through the data...
  smallest count = 4 in the cell      2110244
  WARNING: could not find any rules to collapse dpstoff2 == 2110244
Pass 53 through the data...
  smallest count = 4 in the cell      4031826
  Invoking rule 30211:31826=60226
  replace dpstoff2 = 4060226 if inlist(dpstoff2, 4030211, 4031826)
Pass 54 through the data...
  smallest count = 4 in the cell      5043644
  Invoking rule 43644:64760=103660
  replace dpstoff2 = 5103660 if inlist(dpstoff2, 5043644, 5064760)
Pass 55 through the data...
  smallest count = 5 in the cell      1000060
  Invoking rule 24950:25355:60=54960
  replace dpstoff2 = 1054960 if inlist(dpstoff2, 1024950, 1025355, 1000060)
Pass 56 through the data...
  smallest count = 5 in the cell      2074968
  Invoking rule 72647:74968=142668
  replace dpstoff2 = 2142668 if inlist(dpstoff2, 2072647, 2074968)
Pass 57 through the data...
  smallest count = 5 in the cell      3025560
  Invoking rule 34953:25560=54960
  replace dpstoff2 = 3054960 if inlist(dpstoff2, 3034953, 3025560)
Pass 58 through the data...
  smallest count = 6 in the cell      1000055
  Invoking rule 53:55=25355
  replace dpstoff2 = 1025355 if inlist(dpstoff2, 1000053, 1000055)
Pass 59 through the data...
  smallest count = 6 in the cell      3033944
  Invoking rule 80236:33944=110244
  replace dpstoff2 = 3110244 if inlist(dpstoff2, 3080236, 3033944)
Pass 60 through the data...
  smallest count = 6 in the cell      4054960
  Invoking rule 53647:54960=103660
  replace dpstoff2 = 4103660 if inlist(dpstoff2, 4053647, 4054960)
Pass 61 through the data...
  smallest count = 6 in the cell      5030211
  Invoking rule 30211:31826=60226
  replace dpstoff2 = 5060226 if inlist(dpstoff2, 5030211, 5031826)
Pass 62 through the data...
  smallest count = 8 in the cell      3000026
  Invoking rule 18:24:26=31826
  replace dpstoff2 = 3031826 if inlist(dpstoff2, 3000018, 3000024, 3000026)
Pass 63 through the data...
  smallest count = 8 in the cell      5000030
  Invoking rule 30:36:39:40:44:47:49:50:53:55:60:62=123062
  replace dpstoff2 = 5123062 if inlist(dpstoff2, 5000030, 5000036, 5000039, 5000040, 5000044, 5000047, 5000
> 049, 5000050, 5000053, 5000055, 5000060, 5000062)
Pass 64 through the data...
  smallest count = 9 in the cell      4026268
  Invoking rule 103660:26268=123668
  replace dpstoff2 = 4123668 if inlist(dpstoff2, 4103660, 4026268)
Pass 65 through the data...
  smallest count = 9 in the cell      5055368
  Invoking rule 69:55368=65369
  replace dpstoff2 = 5065369 if inlist(dpstoff2, 5000069, 5055368)
Pass 66 through the data...
  smallest count = 10 in the cell      1054960
  Invoking rule 62:54960=64962
  replace dpstoff2 = 1064962 if inlist(dpstoff2, 1000062, 1054960)
Pass 67 through the data...
  smallest count = 10 in the cell      4060226
  Invoking rule 30:60226=70230
  replace dpstoff2 = 4070230 if inlist(dpstoff2, 4000030, 4060226)
Pass 68 through the data...
  smallest count = 11 in the cell      5103660
  WARNING: could not find any rules to collapse dpstoff2 == 5103660
Pass 69 through the data...
  smallest count = 12 in the cell      1020208
  Invoking rule 20208:21118:24=50224
  replace dpstoff2 = 1050224 if inlist(dpstoff2, 1020208, 1021118, 1000024)
Pass 70 through the data...
  smallest count = 12 in the cell      1033640
  Invoking rule 44:33640=43644
  replace dpstoff2 = 1043644 if inlist(dpstoff2, 1000044, 1033640)
Pass 71 through the data...
  smallest count = 13 in the cell      2142668
  Invoking rule 69:142668=152669
  replace dpstoff2 = 2152669 if inlist(dpstoff2, 2000069, 2142668)
Pass 72 through the data...
  smallest count = 13 in the cell      3110244
  Invoking rule 47:110244=120247
  replace dpstoff2 = 3120247 if inlist(dpstoff2, 3000047, 3110244)
Pass 73 through the data...
  smallest count = 13 in the cell      5060226
  Invoking rule 60226:123062=180262
  replace dpstoff2 = 5180262 if inlist(dpstoff2, 5060226, 5123062)
Pass 74 through the data...
  smallest count = 14 in the cell      3000011
  Invoking rule 11:18:24:26:30=51130
  replace dpstoff2 = 3051130 if inlist(dpstoff2, 3000011, 3000018, 3000024, 3000026, 3000030)
Pass 75 through the data...
  smallest count = 15 in the cell      1000026
  Invoking rule 26:50224=60226
  replace dpstoff2 = 1060226 if inlist(dpstoff2, 1000026, 1050224)
Pass 76 through the data...
  smallest count = 15 in the cell      3054960
  Invoking rule 62:54960=64962
  replace dpstoff2 = 3064962 if inlist(dpstoff2, 3000062, 3054960)
Pass 77 through the data...
  smallest count = 21 in the cell      4070230
  Done collapsing! Exiting...
{\smallskip}
. return list
{\smallskip}
scalars:
         r(arg_min_id) =  4070230
                r(min) =  21
{\smallskip}
macros:
            r(cfailed) : "2023036,2110244,5103660"
             r(failed) : "2023036 2110244 5103660"
\nullskip
\end{stlog}

We will continue to disregard the cell counts of insufficient size for the time being.
Running the resulting do-files \stcmd{dpston.do} and \stcmd{dpstoff.do}
on the population data to create control totals, and providing these control totals
to \stcmd{ipfraking} program produces an apparently successful raking result:

\begin{stlog}
. use trip_sample, clear
{\smallskip}
. run dpston2
{\smallskip}
. run dpstoff2
{\smallskip}
. gen byte _one = 1       
{\smallskip}
. ipfraking [pw=_one], ctotal(dpston2 dpstoff2) gen(raked_weight2)
{\smallskip}
 Iteration 1, max rel difference of raked weights = 36.208881
 Iteration 2, max rel difference of raked weights = .05484732
 Iteration 3, max rel difference of raked weights = .0055794
 Iteration 4, max rel difference of raked weights = .00053851
 Iteration 5, max rel difference of raked weights = .00005171
 Iteration 6, max rel difference of raked weights = 4.962e-06
 Iteration 7, max rel difference of raked weights = 4.762e-07
The worst relative discrepancy of  3.9e-08 is observed for dpstoff2 == 5180262     
Target value =        483; achieved value =        483
{\smallskip}
   Summary of the weight changes
{\smallskip}
              {\VBAR}    Mean    Std. dev.    Min        Max       CV
\HLI{14}{\PLUS}\HLI{50} 
Orig weights  {\VBAR}        1          0         1           1       0
Raked weights {\VBAR}   26.487     5.9013    8.1096      37.001   .2228
Adjust factor {\VBAR}  26.4869               8.1096     37.0014
{\smallskip}
. whatsdeff raked_weight2
{\smallskip}
    Group     {\VBAR}   Min     {\VBAR}   Mean    {\VBAR}   Max     {\VBAR}    CV   {\VBAR}   DEFF  {\VBAR}   N   {\VBAR}  N e
> ff
\HLI{14}{\PLUS}\HLI{11}{\PLUS}\HLI{11}{\PLUS}\HLI{11}{\PLUS}\HLI{9}{\PLUS}\HLI{9}{\PLUS}\HLI{7}{\PLUS}\HLI{5}
\HLI{3}
      Overall {\VBAR}      8.11 {\VBAR}     26.49 {\VBAR}     37.00 {\VBAR}  0.2228 {\VBAR}  1.0496 {\VBAR}  3654 {\VBAR} 3481
> .24
{\smallskip}
\nullskip
\end{stlog}

Note the use of utility program \stcmd{whatsdeff} to compute the design effect
due to unequal weighting; see section \ref{subsec:utility}. The problem of zero cells
appeared to have been solved: each and every population combination of daypart and station
is properly reflected in control total categories, and there are

The weighting cells, however, are still not without problems. Consider this
cross-tab of original and collapsed stations (the first part of the \ifexp\ expression identifies
the daypart, AM Peak; the second part identifies collapsed stations, given the nomenclature
of \stcmd{dpstoff} variable described on page \ref{page:dpston:nomenclature} as the concatenation
of the first variable of the interaction, \stcmd{daypart}; the length of the collapsed sequence,
and its starting and end points; station numbers take up to two characters, and hence the collapsed 
values would use categories of \stcmd{alight\_id} like 20102, and \stcmd{mod} by \stcmd{100*100}
would be greater than the maximum two-digit number, 99).

\begin{stlog}
. tab alight_id dpstoff2 if daypart == 1 \& mod(dpstoff2,100*100)>100
{\smallskip}
                     {\VBAR}         Long ID of the interaction
           alight_id {\VBAR}   1025355    1043644    1060226    1064962 {\VBAR}     Total
\HLI{21}{\PLUS}\HLI{44}{\PLUS}\HLI{10}
        2. Brookline {\VBAR}         0          0          1          0 {\VBAR}         1 
        8. Carmenton {\VBAR}         0          0         11          0 {\VBAR}        11 
      24. Framington {\VBAR}         0          0         15          0 {\VBAR}        15 
  26. Grand Junction {\VBAR}         0          0         15          0 {\VBAR}        15 
      36. Irvingtown {\VBAR}         0          9          0          0 {\VBAR}         9 
      39. Johnsville {\VBAR}         0          3          0          0 {\VBAR}         3 
        44. Limerick {\VBAR}         0         13          0          0 {\VBAR}        13 
    49. Ninth Street {\VBAR}         0          0          0          3 {\VBAR}         3 
    50. Ontario Lake {\VBAR}         0          0          0          2 {\VBAR}         2 
53. Picadilly Square {\VBAR}        23          0          0          0 {\VBAR}        23 
      55. Queens Zoo {\VBAR}         6          0          0          0 {\VBAR}         6 
  60. Redline Circle {\VBAR}         0          0          0          5 {\VBAR}         5 
   62. Silver Spring {\VBAR}         0          0          0         49 {\VBAR}        49 
\HLI{21}{\PLUS}\HLI{44}{\PLUS}\HLI{10}
               Total {\VBAR}        29         25         42         59 {\VBAR}       155 
{\smallskip}
{\smallskip}
\nullskip
\end{stlog}

To the human eye, it is obvious that Picadilly Square (53) and Queens Zoo (55) should have been
a part of the six-station sequence 1064962 spanning from Ninth Street (49) to Silver Spring (62).
Instead, \stcmd{wgtcellcollapse} decided to separate these two stations out into their own cell.
How did that happen? The logic of \stcmd{wgtcellcollapse} is to collapse categories in such a way
as to produce the result with the smallest possible count. Thus, within AM Peak daypart,
the sequence of collapsing steps was as follows.

\begin{description}
    \item[Pass 0] The zero cells were collapsed first: Johnsville (39) and King Street (40) resulting
        in an intermediate cell of size 3.
    \item[Pass 24] The smallest cell of size 1 (Brookline (2)) was collapsed with its neighbor
        (Carmenton (8)) resulting in an intermediate cell of size 12.
    \item[Pass 38] The smallest cell of size 2 (Ontario Lake (50)) was collapsed with its neighbor
        (Ninth Street (49)) resulting in an intermediate cell of size 5.
    \item[Pass 47] The smallest cell of size 3, collapsed Johnsville (39) and King Street (40),
        was further collapsed with its neighbor Irvingtown (36) resulting in an intermediate cell of size .
    \item[Pass 55] The smallest cell of size 5, Redline Circle (60), was collapsed by a three-way rule 
        with a duo Picadilly Square (53) + Queens Zoo (55), which actually was empty, and a small cell 
        Ontario Lake (50) + Ninth Street (49), resulting in an intermediate cell of size 10.
\end{description}

Let us look at that last step in more detail. At this stage, Redline Circle (60) with 5 exiting passengers in
the sample could be collapsed with:
\begin{enumerate}
    \item Silver Spring (62), to form a cell of size 54;
    \item Queens Zoo (55), to form a cell of size 11;
    \item a sequence of Picadilly Square (53) and Queens Zoo (55), to form a cell of size 34;
    \item \ldots and many other options
\end{enumerate}
However, at pass 55, \stcmd{wgtcellcollapse} picked the rule 24950:25355:60=54960 which, at the time it was
processed, had a count of 5 in the cell 24950, a count of zero in the cell 25355, and a count of 5 in the
original station Redline Circle (60). (Note that the cell 25355 would actually form eventually at pass 58.)
The problem lies with the zero count of the ghost of the cell 25355.

To overcome this problem, \stcmd{wgtcellcollapse} have a \stcmd{strict} option that only allows the rules
that have a non-zero count in every component of the rule (so 24950:25355:60=54960 would not be a legal
merge under that option). As is easily seen, this option directly contradicts the \stcmd{zeroes()} option,
and that necessitates separate runs.

\subsubsection{The third pass of cell collapse and raking: \stcmd{strict} and \stcmd{feed} options}

We will separate the two runs of \stcmd{wgtcellcollapse} into a run that only deals with zeroes,
and another run that deals with everything else. To prevent \stcmd{wgtcellcollapse} from any further
merges, \stcmd{mincellsize(1)} can be specified in the first run. As the relevant variables will have already been
created by the first run, the option to pass the variable name to be further modified is
\stcmd{feed()}. To make sure that the relevant variable exists in the data set,
the option \stcmd{run} instructs \stcmd{wgtcellcollapse} to run the do-file it just created,
thus creating or modifying the collapsed cell variable.
Finally, instead of specifying \stcmd{replace} to overwrite the do-files that 
\stcmd{wgtcellcollapse} creates, we need to specify \stcmd{append} to keep adding to these files.

\begin{stlog}
. use trip_sample_rules, clear
{\smallskip}
. wgtcellcollapse collapse, variables(daypart board_id) mincellsize(1) ///
>         zeroes(39 40 44 49 55 60) ///
>         generate(dpston3) saving(dpston3.do) replace run
Pass 0 through the data...
  smallest count = 1 in the cell      2000039
{\smallskip}
Processing zero cells...
{\smallskip}
  Invoking rule 49:50=24950 to collapse zero cells
  replace dpston3 = 1024950 if inlist(dpston3, 1000049, 1000050)
Pass 0 through the data...
  smallest count = 1 in the cell      2000039
  Invoking rule 40:44=24044 to collapse zero cells
  replace dpston3 = 2024044 if inlist(dpston3, 2000040, 2000044)
Pass 0 through the data...
  smallest count = 1 in the cell      2000039
  Invoking rule 49:50=24950 to collapse zero cells
  replace dpston3 = 2024950 if inlist(dpston3, 2000049, 2000050)
Pass 0 through the data...
  smallest count = 1 in the cell      2000039
  Invoking rule 55:60=25560 to collapse zero cells
  replace dpston3 = 2025560 if inlist(dpston3, 2000055, 2000060)
Pass 0 through the data...
  smallest count = 1 in the cell      2000039
  Invoking rule 36:39=23639 to collapse zero cells
  replace dpston3 = 4023639 if inlist(dpston3, 4000036, 4000039)
Pass 0 through the data...
  smallest count = 1 in the cell      2000039
  Invoking rule 40:44=24044 to collapse zero cells
  replace dpston3 = 4024044 if inlist(dpston3, 4000040, 4000044)
Pass 0 through the data...
  smallest count = 1 in the cell      2000039
  Invoking rule 49:50=24950 to collapse zero cells
  replace dpston3 = 4024950 if inlist(dpston3, 4000049, 4000050)
Pass 0 through the data...
  smallest count = 1 in the cell      2000039
  Invoking rule 53:55=25355 to collapse zero cells
  replace dpston3 = 4025355 if inlist(dpston3, 4000053, 4000055)
Pass 0 through the data...
  smallest count = 1 in the cell      2000039
  Invoking rule 24950:53:55:60=54960 to collapse zero cells
  replace dpston3 = 4054960 if inlist(dpston3, 4024950, 4000053, 4000055, 4000060)
Pass 0 through the data...
  smallest count = 1 in the cell      2000039
  Invoking rule 39:40=23940 to collapse zero cells
  replace dpston3 = 5023940 if inlist(dpston3, 5000039, 5000040)
Pass 0 through the data...
  smallest count = 1 in the cell      2000039
  Invoking rule 49:50=24950 to collapse zero cells
  replace dpston3 = 5024950 if inlist(dpston3, 5000049, 5000050)
Pass 0 through the data...
  smallest count = 1 in the cell      2000039
  Invoking rule 24950:25355:60=54960 to collapse zero cells
  replace dpston3 = 5054960 if inlist(dpston3, 5024950, 5025355, 5000060)
Pass 0 through the data...
  smallest count = 1 in the cell      2000039
Pass 12 through the data...
  smallest count = 1 in the cell      2000039
  Done collapsing! Exiting...
{\smallskip}
. wgtcellcollapse collapse, variables(daypart board_id) mincellsize(20) ///
>         strict feed(dpston3) saving(dpston3.do) append run
Pass 12 through the data...
  smallest count = 1 in the cell      2000039
  Invoking rule 39:24044=33944
  replace dpston3 = 2033944 if inlist(dpston3, 2000039, 2024044)
Pass 13 through the data...
  smallest count = 1 in the cell      2024950
  Invoking rule 53:24950=34953
  replace dpston3 = 2034953 if inlist(dpston3, 2000053, 2024950)
Pass 14 through the data...
  smallest count = 1 in the cell      2025560
  Invoking rule 34953:25560=54960
  replace dpston3 = 2054960 if inlist(dpston3, 2034953, 2025560)
Pass 15 through the data...
  smallest count = 1 in the cell      3000039
  Invoking rule 39:40=23940
  replace dpston3 = 3023940 if inlist(dpston3, 3000039, 3000040)
Pass 16 through the data...
  smallest count = 1 in the cell      4023639
  Invoking rule 23639:24044=43644
  replace dpston3 = 4043644 if inlist(dpston3, 4023639, 4024044)
Pass 17 through the data...
  smallest count = 1 in the cell      4054960
  Invoking rule 47:54960=64760
  replace dpston3 = 4064760 if inlist(dpston3, 4000047, 4054960)
Pass 18 through the data...
  smallest count = 1 in the cell      5000024
  Invoking rule 18:24=21824
  replace dpston3 = 5021824 if inlist(dpston3, 5000018, 5000024)
Pass 19 through the data...
  smallest count = 1 in the cell      5023940
  Invoking rule 44:23940=33944
  replace dpston3 = 5033944 if inlist(dpston3, 5000044, 5023940)
Pass 20 through the data...
  smallest count = 1 in the cell      5054960
  Invoking rule 47:54960=64760
  replace dpston3 = 5064760 if inlist(dpston3, 5000047, 5054960)
Pass 21 through the data...
  smallest count = 2 in the cell      2000036
  Invoking rule 36:33944=43644
  replace dpston3 = 2043644 if inlist(dpston3, 2000036, 2033944)
Pass 22 through the data...
  smallest count = 2 in the cell      4043644
  Invoking rule 43644:64760=103660
  replace dpston3 = 4103660 if inlist(dpston3, 4043644, 4064760)
Pass 23 through the data...
  smallest count = 2 in the cell      5000055
  Invoking rule 53:55=25355
  replace dpston3 = 5025355 if inlist(dpston3, 5000053, 5000055)
Pass 24 through the data...
  smallest count = 3 in the cell      2000024
  Invoking rule 18:24=21824
  replace dpston3 = 2021824 if inlist(dpston3, 2000018, 2000024)
Pass 25 through the data...
  smallest count = 3 in the cell      3023940
  Invoking rule 44:23940=33944
  replace dpston3 = 3033944 if inlist(dpston3, 3000044, 3023940)
Pass 26 through the data...
  smallest count = 3 in the cell      4000024
  Invoking rule 18:24=21824
  replace dpston3 = 4021824 if inlist(dpston3, 4000018, 4000024)
Pass 27 through the data...
  smallest count = 3 in the cell      5000001
  Invoking rule 1:2=20102
  replace dpston3 = 5020102 if inlist(dpston3, 5000001, 5000002)
Pass 28 through the data...
  smallest count = 3 in the cell      5000068
  Invoking rule 62:68=26268
  replace dpston3 = 5026268 if inlist(dpston3, 5000062, 5000068)
Pass 29 through the data...
  smallest count = 4 in the cell      2000001
  Invoking rule 1:2=20102
  replace dpston3 = 2020102 if inlist(dpston3, 2000001, 2000002)
Pass 30 through the data...
  smallest count = 4 in the cell      2000008
  Invoking rule 8:20102=30108
  replace dpston3 = 2030108 if inlist(dpston3, 2000008, 2020102)
Pass 31 through the data...
  smallest count = 4 in the cell      2043644
  Invoking rule 30:43644=53044
  replace dpston3 = 2053044 if inlist(dpston3, 2000030, 2043644)
Pass 32 through the data...
  smallest count = 4 in the cell      3000050
  Invoking rule 49:50=24950
  replace dpston3 = 3024950 if inlist(dpston3, 3000049, 3000050)
Pass 33 through the data...
  smallest count = 4 in the cell      5033944
  Invoking rule 33944:64760=93960
  replace dpston3 = 5093960 if inlist(dpston3, 5033944, 5064760)
Pass 34 through the data...
  smallest count = 5 in the cell      1000039
  Invoking rule 39:40=23940
  replace dpston3 = 1023940 if inlist(dpston3, 1000039, 1000040)
Pass 35 through the data...
  smallest count = 5 in the cell      3000060
  Invoking rule 55:60=25560
  replace dpston3 = 3025560 if inlist(dpston3, 3000055, 3000060)
Pass 36 through the data...
  smallest count = 5 in the cell      4000026
  Invoking rule 26:21824=31826
  replace dpston3 = 4031826 if inlist(dpston3, 4000026, 4021824)
Pass 37 through the data...
  smallest count = 5 in the cell      5021824
  Invoking rule 26:21824=31826
  replace dpston3 = 5031826 if inlist(dpston3, 5000026, 5021824)
Pass 38 through the data...
  smallest count = 6 in the cell      2000068
  Invoking rule 62:68=26268
  replace dpston3 = 2026268 if inlist(dpston3, 2000062, 2000068)
Pass 39 through the data...
  smallest count = 6 in the cell      2054960
  Invoking rule 54960:26268=74968
  replace dpston3 = 2074968 if inlist(dpston3, 2054960, 2026268)
Pass 40 through the data...
  smallest count = 6 in the cell      4000002
  Invoking rule 1:2=20102
  replace dpston3 = 4020102 if inlist(dpston3, 4000001, 4000002)
Pass 41 through the data...
  smallest count = 7 in the cell      4025355
  WARNING: could not find any rules to collapse dpston3 == 4025355
Pass 42 through the data...
  smallest count = 7 in the cell      5025355
  WARNING: could not find any rules to collapse dpston3 == 5025355
Pass 43 through the data...
  smallest count = 8 in the cell      2021824
  Invoking rule 26:21824=31826
  replace dpston3 = 2031826 if inlist(dpston3, 2000026, 2021824)
Pass 44 through the data...
  smallest count = 10 in the cell      1000055
  Invoking rule 55:60=25560
  replace dpston3 = 1025560 if inlist(dpston3, 1000055, 1000060)
Pass 45 through the data...
  smallest count = 10 in the cell      4103660
  Invoking rule 62:103660=113662
  replace dpston3 = 4113662 if inlist(dpston3, 4000062, 4103660)
Pass 46 through the data...
  smallest count = 10 in the cell      5020102
  Invoking rule 8:20102=30108
  replace dpston3 = 5030108 if inlist(dpston3, 5000008, 5020102)
Pass 47 through the data...
  smallest count = 11 in the cell      3000001
  Invoking rule 1:2=20102
  replace dpston3 = 3020102 if inlist(dpston3, 3000001, 3000002)
Pass 48 through the data...
  smallest count = 11 in the cell      4000068
  Invoking rule 68:113662=123668
  replace dpston3 = 4123668 if inlist(dpston3, 4000068, 4113662)
Pass 49 through the data...
  smallest count = 11 in the cell      5031826
  Invoking rule 30:31826=41830
  replace dpston3 = 5041830 if inlist(dpston3, 5000030, 5031826)
Pass 50 through the data...
  smallest count = 12 in the cell      2030108
  Invoking rule 11:30108=40111
  replace dpston3 = 2040111 if inlist(dpston3, 2000011, 2030108)
Pass 51 through the data...
  smallest count = 12 in the cell      3033944
  Invoking rule 36:33944=43644
  replace dpston3 = 3043644 if inlist(dpston3, 3000036, 3033944)
Pass 52 through the data...
  smallest count = 12 in the cell      4031826
  Invoking rule 30:31826=41830
  replace dpston3 = 4041830 if inlist(dpston3, 4000030, 4031826)
Pass 53 through the data...
  smallest count = 13 in the cell      1024950
  Invoking rule 53:24950=34953
  replace dpston3 = 1034953 if inlist(dpston3, 1000053, 1024950)
Pass 54 through the data...
  smallest count = 13 in the cell      3024950
  Invoking rule 53:24950=34953
  replace dpston3 = 3034953 if inlist(dpston3, 3000053, 3024950)
Pass 55 through the data...
  smallest count = 13 in the cell      4020102
  Invoking rule 8:20102=30108
  replace dpston3 = 4030108 if inlist(dpston3, 4000008, 4020102)
Pass 56 through the data...
  smallest count = 15 in the cell      5000036
  Invoking rule 36:93960=103660
  replace dpston3 = 5103660 if inlist(dpston3, 5000036, 5093960)
Pass 57 through the data...
  smallest count = 19 in the cell      3025560
  Invoking rule 62:25560=35562
  replace dpston3 = 3035562 if inlist(dpston3, 3000062, 3025560)
Pass 58 through the data...
  smallest count = 20 in the cell      5026268
  Done collapsing! Exiting...
{\smallskip}
. wgtcellcollapse collapse, variables(daypart alight_id) mincellsize(1) ///
>         zeroes(2 8 36 39 40 44 47 49 50 55 60 62) ///
>         generate(dpstoff3) saving(dpstoff3.do) replace run
Pass 0 through the data...
  smallest count = 1 in the cell      1000002
{\smallskip}
Processing zero cells...
{\smallskip}
  Invoking rule 39:40=23940 to collapse zero cells
  replace dpstoff3 = 1023940 if inlist(dpstoff3, 1000039, 1000040)
Pass 0 through the data...
  smallest count = 1 in the cell      1000002
  Invoking rule 2:8=20208 to collapse zero cells
  replace dpstoff3 = 2020208 if inlist(dpstoff3, 2000002, 2000008)
Pass 0 through the data...
  smallest count = 1 in the cell      1000002
  Invoking rule 30:36=23036 to collapse zero cells
  replace dpstoff3 = 2023036 if inlist(dpstoff3, 2000030, 2000036)
Pass 0 through the data...
  smallest count = 1 in the cell      1000002
  Invoking rule 26:30:36:39=42639 to collapse zero cells
  replace dpstoff3 = 2042639 if inlist(dpstoff3, 2000026, 2000030, 2000036, 2000039)
Pass 0 through the data...
  smallest count = 1 in the cell      1000002
  Invoking rule 24:26:30:36:39:40=62440 to collapse zero cells
  replace dpstoff3 = 2062440 if inlist(dpstoff3, 2000024, 2000026, 2000030, 2000036,
>  2000039, 2000040)
Pass 0 through the data...
  smallest count = 1 in the cell      1000002
  Invoking rule 44:62440=72444 to collapse zero cells
  replace dpstoff3 = 2072444 if inlist(dpstoff3, 2000044, 2062440)
Pass 0 through the data...
  smallest count = 1 in the cell      1000002
  Invoking rule 49:50=24950 to collapse zero cells
  replace dpstoff3 = 2024950 if inlist(dpstoff3, 2000049, 2000050)
Pass 0 through the data...
  smallest count = 1 in the cell      1000002
  Invoking rule 53:55=25355 to collapse zero cells
  replace dpstoff3 = 2025355 if inlist(dpstoff3, 2000053, 2000055)
Pass 0 through the data...
  smallest count = 1 in the cell      1000002
  Invoking rule 60:25355=35360 to collapse zero cells
  replace dpstoff3 = 2035360 if inlist(dpstoff3, 2000060, 2025355)
Pass 0 through the data...
  smallest count = 1 in the cell      1000002
  Invoking rule 62:35360=45362 to collapse zero cells
  replace dpstoff3 = 2045362 if inlist(dpstoff3, 2000062, 2035360)
Pass 0 through the data...
  smallest count = 1 in the cell      1000002
  Invoking rule 2:8=20208 to collapse zero cells
  replace dpstoff3 = 3020208 if inlist(dpstoff3, 3000002, 3000008)
Pass 0 through the data...
  smallest count = 1 in the cell      1000002
  Invoking rule 55:60=25560 to collapse zero cells
  replace dpstoff3 = 3025560 if inlist(dpstoff3, 3000055, 3000060)
Pass 0 through the data...
  smallest count = 1 in the cell      1000002
  Invoking rule 2:8:11=30211 to collapse zero cells
  replace dpstoff3 = 4030211 if inlist(dpstoff3, 4000002, 4000008, 4000011)
Pass 0 through the data...
  smallest count = 1 in the cell      1000002
  Invoking rule 36:39=23639 to collapse zero cells
  replace dpstoff3 = 4023639 if inlist(dpstoff3, 4000036, 4000039)
Pass 0 through the data...
  smallest count = 1 in the cell      1000002
  Invoking rule 40:23639=33640 to collapse zero cells
  replace dpstoff3 = 4033640 if inlist(dpstoff3, 4000040, 4023639)
Pass 0 through the data...
  smallest count = 1 in the cell      1000002
  Invoking rule 44:33640=43644 to collapse zero cells
  replace dpstoff3 = 4043644 if inlist(dpstoff3, 4000044, 4033640)
Pass 0 through the data...
  smallest count = 1 in the cell      1000002
  Invoking rule 49:50=24950 to collapse zero cells
  replace dpstoff3 = 4024950 if inlist(dpstoff3, 4000049, 4000050)
Pass 0 through the data...
  smallest count = 1 in the cell      1000002
  Invoking rule 55:60=25560 to collapse zero cells
  replace dpstoff3 = 4025560 if inlist(dpstoff3, 4000055, 4000060)
Pass 0 through the data...
  smallest count = 1 in the cell      1000002
  Invoking rule 2:8=20208 to collapse zero cells
  replace dpstoff3 = 5020208 if inlist(dpstoff3, 5000002, 5000008)
Pass 0 through the data...
  smallest count = 1 in the cell      1000002
  Invoking rule 36:39=23639 to collapse zero cells
  replace dpstoff3 = 5023639 if inlist(dpstoff3, 5000036, 5000039)
Pass 0 through the data...
  smallest count = 1 in the cell      1000002
  Invoking rule 40:44=24044 to collapse zero cells
  replace dpstoff3 = 5024044 if inlist(dpstoff3, 5000040, 5000044)
Pass 0 through the data...
  smallest count = 1 in the cell      1000002
  Invoking rule 49:50=24950 to collapse zero cells
  replace dpstoff3 = 5024950 if inlist(dpstoff3, 5000049, 5000050)
Pass 0 through the data...
  smallest count = 1 in the cell      1000002
  Invoking rule 53:55=25355 to collapse zero cells
  replace dpstoff3 = 5025355 if inlist(dpstoff3, 5000053, 5000055)
Pass 0 through the data...
  smallest count = 1 in the cell      1000002
  Invoking rule 24950:53:55:60=54960 to collapse zero cells
  replace dpstoff3 = 5054960 if inlist(dpstoff3, 5024950, 5000053, 5000055, 5000060)
Pass 0 through the data...
  smallest count = 1 in the cell      1000002
Pass 24 through the data...
  smallest count = 1 in the cell      1000002
  Done collapsing! Exiting...
{\smallskip}
. wgtcellcollapse collapse, variables(daypart alight_id) mincellsize(20) ///
>         strict feed(dpstoff3) saving(dpstoff3.do) append run
Pass 24 through the data...
  smallest count = 1 in the cell      1000002
  Invoking rule 2:8=20208
  replace dpstoff3 = 1020208 if inlist(dpstoff3, 1000002, 1000008)
Pass 25 through the data...
  smallest count = 1 in the cell      2000011
  Invoking rule 11:18=21118
  replace dpstoff3 = 2021118 if inlist(dpstoff3, 2000011, 2000018)
Pass 26 through the data...
  smallest count = 1 in the cell      2020208
  Invoking rule 20208:21118=40218
  replace dpstoff3 = 2040218 if inlist(dpstoff3, 2020208, 2021118)
Pass 27 through the data...
  smallest count = 1 in the cell      2024950
  Invoking rule 24950:45362=64962
  replace dpstoff3 = 2064962 if inlist(dpstoff3, 2024950, 2045362)
Pass 28 through the data...
  smallest count = 1 in the cell      2072444
  Invoking rule 40218:72444=110244
  replace dpstoff3 = 2110244 if inlist(dpstoff3, 2040218, 2072444)
Pass 29 through the data...
  smallest count = 1 in the cell      3000039
  Invoking rule 39:40=23940
  replace dpstoff3 = 3023940 if inlist(dpstoff3, 3000039, 3000040)
Pass 30 through the data...
  smallest count = 1 in the cell      3000049
  Invoking rule 49:50=24950
  replace dpstoff3 = 3024950 if inlist(dpstoff3, 3000049, 3000050)
Pass 31 through the data...
  smallest count = 1 in the cell      3020208
  Invoking rule 11:20208=30211
  replace dpstoff3 = 3030211 if inlist(dpstoff3, 3000011, 3020208)
Pass 32 through the data...
  smallest count = 1 in the cell      4000018
  Invoking rule 18:24=21824
  replace dpstoff3 = 4021824 if inlist(dpstoff3, 4000018, 4000024)
Pass 33 through the data...
  smallest count = 1 in the cell      4000026
  Invoking rule 26:21824=31826
  replace dpstoff3 = 4031826 if inlist(dpstoff3, 4000026, 4021824)
Pass 34 through the data...
  smallest count = 1 in the cell      4025560
  Invoking rule 53:25560=35360
  replace dpstoff3 = 4035360 if inlist(dpstoff3, 4000053, 4025560)
Pass 35 through the data...
  smallest count = 1 in the cell      5000026
  Invoking rule 24:26=22426
  replace dpstoff3 = 5022426 if inlist(dpstoff3, 5000024, 5000026)
Pass 36 through the data...
  smallest count = 1 in the cell      5020208
  Invoking rule 11:20208=30211
  replace dpstoff3 = 5030211 if inlist(dpstoff3, 5000011, 5020208)
Pass 37 through the data...
  smallest count = 1 in the cell      5054960
  Invoking rule 47:54960=64760
  replace dpstoff3 = 5064760 if inlist(dpstoff3, 5000047, 5054960)
Pass 38 through the data...
  smallest count = 2 in the cell      1000050
  Invoking rule 49:50=24950
  replace dpstoff3 = 1024950 if inlist(dpstoff3, 1000049, 1000050)
Pass 39 through the data...
  smallest count = 2 in the cell      2042639
  WARNING: could not find any rules to collapse dpstoff3 == 2042639
Pass 40 through the data...
  smallest count = 2 in the cell      2064962
  Invoking rule 68:64962=74968
  replace dpstoff3 = 2074968 if inlist(dpstoff3, 2000068, 2064962)
Pass 41 through the data...
  smallest count = 2 in the cell      3000024
  Invoking rule 24:26=22426
  replace dpstoff3 = 3022426 if inlist(dpstoff3, 3000024, 3000026)
Pass 42 through the data...
  smallest count = 2 in the cell      3000044
  Invoking rule 44:23940=33944
  replace dpstoff3 = 3033944 if inlist(dpstoff3, 3000044, 3023940)
Pass 43 through the data...
  smallest count = 2 in the cell      3024950
  Invoking rule 53:24950=34953
  replace dpstoff3 = 3034953 if inlist(dpstoff3, 3000053, 3024950)
Pass 44 through the data...
  smallest count = 2 in the cell      4024950
  Invoking rule 24950:35360=54960
  replace dpstoff3 = 4054960 if inlist(dpstoff3, 4024950, 4035360)
Pass 45 through the data...
  smallest count = 2 in the cell      4043644
  Invoking rule 47:43644=53647
  replace dpstoff3 = 4053647 if inlist(dpstoff3, 4000047, 4043644)
Pass 46 through the data...
  smallest count = 2 in the cell      5023639
  Invoking rule 23639:24044=43644
  replace dpstoff3 = 5043644 if inlist(dpstoff3, 5023639, 5024044)
Pass 47 through the data...
  smallest count = 2 in the cell      5025355
  WARNING: could not find any rules to collapse dpstoff3 == 5025355
Pass 48 through the data...
  smallest count = 3 in the cell      1023940
  Invoking rule 36:23940=33640
  replace dpstoff3 = 1033640 if inlist(dpstoff3, 1000036, 1023940)
Pass 49 through the data...
  smallest count = 3 in the cell      4000062
  Invoking rule 62:68=26268
  replace dpstoff3 = 4026268 if inlist(dpstoff3, 4000062, 4000068)
Pass 50 through the data...
  smallest count = 3 in the cell      5022426
  Invoking rule 18:22426=31826
  replace dpstoff3 = 5031826 if inlist(dpstoff3, 5000018, 5022426)
Pass 51 through the data...
  smallest count = 4 in the cell      2023036
  WARNING: could not find any rules to collapse dpstoff3 == 2023036
Pass 52 through the data...
  smallest count = 4 in the cell      2110244
  Invoking rule 47:110244=120247
  replace dpstoff3 = 2120247 if inlist(dpstoff3, 2000047, 2110244)
Pass 53 through the data...
  smallest count = 4 in the cell      3000036
  Invoking rule 36:33944=43644
  replace dpstoff3 = 3043644 if inlist(dpstoff3, 3000036, 3033944)
Pass 54 through the data...
  smallest count = 4 in the cell      4031826
  Invoking rule 30211:31826=60226
  replace dpstoff3 = 4060226 if inlist(dpstoff3, 4030211, 4031826)
Pass 55 through the data...
  smallest count = 4 in the cell      5043644
  Invoking rule 43644:64760=103660
  replace dpstoff3 = 5103660 if inlist(dpstoff3, 5043644, 5064760)
Pass 56 through the data...
  smallest count = 5 in the cell      1000060
  Invoking rule 55:60=25560
  replace dpstoff3 = 1025560 if inlist(dpstoff3, 1000055, 1000060)
Pass 57 through the data...
  smallest count = 5 in the cell      1024950
  Invoking rule 53:24950=34953
  replace dpstoff3 = 1034953 if inlist(dpstoff3, 1000053, 1024950)
Pass 58 through the data...
  smallest count = 5 in the cell      2074968
  Invoking rule 120247:74968=190268
  replace dpstoff3 = 2190268 if inlist(dpstoff3, 2120247, 2074968)
Pass 59 through the data...
  smallest count = 5 in the cell      3025560
  Invoking rule 34953:25560=54960
  replace dpstoff3 = 3054960 if inlist(dpstoff3, 3034953, 3025560)
Pass 60 through the data...
  smallest count = 6 in the cell      4054960
  Invoking rule 53647:54960=103660
  replace dpstoff3 = 4103660 if inlist(dpstoff3, 4053647, 4054960)
Pass 61 through the data...
  smallest count = 6 in the cell      5030211
  Invoking rule 30211:31826=60226
  replace dpstoff3 = 5060226 if inlist(dpstoff3, 5030211, 5031826)
Pass 62 through the data...
  smallest count = 7 in the cell      5000068
  Invoking rule 62:68=26268
  replace dpstoff3 = 5026268 if inlist(dpstoff3, 5000062, 5000068)
Pass 63 through the data...
  smallest count = 8 in the cell      5000030
  Invoking rule 30:103660=113060
  replace dpstoff3 = 5113060 if inlist(dpstoff3, 5000030, 5103660)
Pass 64 through the data...
  smallest count = 9 in the cell      4026268
  Invoking rule 103660:26268=123668
  replace dpstoff3 = 4123668 if inlist(dpstoff3, 4103660, 4026268)
Pass 65 through the data...
  smallest count = 10 in the cell      3022426
  Invoking rule 18:22426=31826
  replace dpstoff3 = 3031826 if inlist(dpstoff3, 3000018, 3022426)
Pass 66 through the data...
  smallest count = 10 in the cell      3043644
  Invoking rule 30:43644=53044
  replace dpstoff3 = 3053044 if inlist(dpstoff3, 3000030, 3043644)
Pass 67 through the data...
  smallest count = 10 in the cell      4060226
  Invoking rule 30:60226=70230
  replace dpstoff3 = 4070230 if inlist(dpstoff3, 4000030, 4060226)
Pass 68 through the data...
  smallest count = 11 in the cell      1025560
  Invoking rule 34953:25560=54960
  replace dpstoff3 = 1054960 if inlist(dpstoff3, 1034953, 1025560)
Pass 69 through the data...
  smallest count = 12 in the cell      1020208
  Invoking rule 11:20208=30211
  replace dpstoff3 = 1030211 if inlist(dpstoff3, 1000011, 1020208)
Pass 70 through the data...
  smallest count = 12 in the cell      1033640
  Invoking rule 44:33640=43644
  replace dpstoff3 = 1043644 if inlist(dpstoff3, 1000044, 1033640)
Pass 71 through the data...
  smallest count = 13 in the cell      5060226
  Invoking rule 60226:113060=170260
  replace dpstoff3 = 5170260 if inlist(dpstoff3, 5060226, 5113060)
Pass 72 through the data...
  smallest count = 15 in the cell      1000024
  Invoking rule 24:26=22426
  replace dpstoff3 = 1022426 if inlist(dpstoff3, 1000024, 1000026)
Pass 73 through the data...
  smallest count = 15 in the cell      2190268
  Invoking rule 69:190268=200269
  replace dpstoff3 = 2200269 if inlist(dpstoff3, 2000069, 2190268)
Pass 74 through the data...
  smallest count = 15 in the cell      3030211
  Invoking rule 30211:31826=60226
  replace dpstoff3 = 3060226 if inlist(dpstoff3, 3030211, 3031826)
Pass 75 through the data...
  smallest count = 15 in the cell      3054960
  Invoking rule 62:54960=64962
  replace dpstoff3 = 3064962 if inlist(dpstoff3, 3000062, 3054960)
Pass 76 through the data...
  smallest count = 16 in the cell      5026268
  Invoking rule 170260:26268=190268
  replace dpstoff3 = 5190268 if inlist(dpstoff3, 5170260, 5026268)
Pass 77 through the data...
  smallest count = 21 in the cell      4070230
  Done collapsing! Exiting...
\nullskip
\end{stlog}

The result still isn't satisfactory, as some collapsed rules still overlap:

\begin{stlog}
. tab alight_id dpstoff3 if daypart == 2 \& mod(dpstoff3,100*100)>99
{\smallskip}
                      {\VBAR}    Long ID of the interaction
            alight_id {\VBAR}   2023036    2042639    2200269 {\VBAR}     Total
\HLI{22}{\PLUS}\HLI{33}{\PLUS}\HLI{10}
         8. Carmenton {\VBAR}         0          0          1 {\VBAR}         1 
         11. Dogville {\VBAR}         0          0          1 {\VBAR}         1 
         18. East End {\VBAR}         0          0          1 {\VBAR}         1 
       24. Framington {\VBAR}         0          0          1 {\VBAR}         1 
   26. Grand Junction {\VBAR}         0          2          0 {\VBAR}         2 
       30. High Point {\VBAR}         4          0          0 {\VBAR}         4 
      47. Moscow City {\VBAR}         0          0          6 {\VBAR}         6 
     49. Ninth Street {\VBAR}         0          0          1 {\VBAR}         1 
 53. Picadilly Square {\VBAR}         0          0          1 {\VBAR}         1 
      68. Toledo Town {\VBAR}         0          0          3 {\VBAR}         3 
    69. Union Station {\VBAR}         0          0        138 {\VBAR}       138 
\HLI{22}{\PLUS}\HLI{33}{\PLUS}\HLI{10}
                Total {\VBAR}         4          2        153 {\VBAR}       159 
{\smallskip}
{\smallskip}
\nullskip
\end{stlog}

This overlap can be traced back to the collapsing of zero cells:
first, the cell 2023036 came to being by a reasonable, at its face, collapsing 
of the zero cell Irvingtown (36) with non-zero cell High Point (30);
and then the cell 2042639 came to being by a long overreach for the zero cell
Johnsville (39) to be collapsed with a non-zero cell Grand Junction (26).

\subsubsection{The fourth pass of cell collapse and raking: \stcmd{greedy} and \stcmd{maxcat()} options}

The process can be improved with an additional option \stcmd{greedy} that is applicable mostly
to the collapsing of zero cells. It modifies behavior of \stcmd{wgtcellcollapse} to require that,
among the possible candidate rules with the lowest count, the rule with the \textit{greatest} number 
of components is preferred. That way, the long streaks of zeroes from Irvingtown (36) to Limerick (44) in
midday part could be collapsed simultaneously into one cell. To support this option, and avoid complex collapses
of zero cells with the already defined cells, option \stcmd{maxcategory()} specifies the greatest
value of a component of a rule. By specifying \stcmd{maxcategory(99)}, we can instruct \stcmd{wgtcellcollapse}
to only use rules that deal with individual stations, and do not use the rules that involve collapsed
cells (which would have numbers of at least 20102 for the collapsed cell Alewife (1) and Brookline (2)).
In the first run, those collapsed cells will always be empty ghosts, and they should not be used.

Note also that with the \stcmd{greedy} option, one would want to specify the zeroes somewhere in the middle
of the streak, and possibly across multiple categories of the interacting variable. In our example,
specifying \stcmd(zeroes(36)} would collapse the midday streak of zero counts, but the need to collapse
the zeroes in the night and the weekend dayparts would still remain, necessitating something like
\stcmd{zeroes(40)} --- which, in turn, will likely create overlapping artifacts in the midday section.
However, specifying \stcmd{zeroes(40)} without \stcmd(zeroes(36)} would take care of all the streaks
observed in Table \ref{tab:sample:xtab}.

\begin{stlog}
. use trip_sample_rules, clear
{\smallskip}
. wgtcellcollapse collapse, variables(daypart board_id) mincellsize(1) ///
>         zeroes(39 44 49 60) greedy maxcategory(99) ///
>         generate(dpston4) saving(dpston4.do) replace run
Pass 0 through the data...
  smallest count = 1 in the cell      2000039
{\smallskip}
Processing zero cells...
{\smallskip}
  Invoking rule 49:50=24950 to collapse zero cells
  replace dpston4 = 1024950 if inlist(dpston4, 1000049, 1000050)
Pass 0 through the data...
  smallest count = 1 in the cell      2000039
  Invoking rule 40:44=24044 to collapse zero cells
  replace dpston4 = 2024044 if inlist(dpston4, 2000040, 2000044)
Pass 0 through the data...
  smallest count = 1 in the cell      2000039
  Invoking rule 49:50=24950 to collapse zero cells
  replace dpston4 = 2024950 if inlist(dpston4, 2000049, 2000050)
Pass 0 through the data...
  smallest count = 1 in the cell      2000039
  Invoking rule 55:60=25560 to collapse zero cells
  replace dpston4 = 2025560 if inlist(dpston4, 2000055, 2000060)
Pass 0 through the data...
  smallest count = 1 in the cell      2000039
  Invoking rule 36:39:40=33640 to collapse zero cells
  replace dpston4 = 4033640 if inlist(dpston4, 4000036, 4000039, 4000040)
Pass 0 through the data...
  smallest count = 1 in the cell      2000039
  Invoking rule 49:50=24950 to collapse zero cells
  replace dpston4 = 4024950 if inlist(dpston4, 4000049, 4000050)
Pass 0 through the data...
  smallest count = 1 in the cell      2000039
  Invoking rule 49:50:53:55:60=54960 to collapse zero cells
  replace dpston4 = 4054960 if inlist(dpston4, 4000049, 4000050, 4000053, 4000055, 4000060)
Pass 0 through the data...
  smallest count = 1 in the cell      2000039
  Invoking rule 39:40=23940 to collapse zero cells
  replace dpston4 = 5023940 if inlist(dpston4, 5000039, 5000040)
Pass 0 through the data...
  smallest count = 1 in the cell      2000039
  Invoking rule 49:50=24950 to collapse zero cells
  replace dpston4 = 5024950 if inlist(dpston4, 5000049, 5000050)
Pass 0 through the data...
  smallest count = 1 in the cell      2000039
  Invoking rule 55:60=25560 to collapse zero cells
  replace dpston4 = 5025560 if inlist(dpston4, 5000055, 5000060)
Pass 0 through the data...
  smallest count = 1 in the cell      2000039
Pass 10 through the data...
  smallest count = 1 in the cell      2000039
  Done collapsing! Exiting...
{\smallskip}
. wgtcellcollapse collapse, variables(daypart board_id) mincellsize(20) ///
>         strict feed(dpston4) saving(dpston4.do) append run
Pass 10 through the data...
  smallest count = 1 in the cell      2000039
  Invoking rule 39:24044=33944
  replace dpston4 = 2033944 if inlist(dpston4, 2000039, 2024044)
Pass 11 through the data...
  smallest count = 1 in the cell      2024950
  Invoking rule 53:24950=34953
  replace dpston4 = 2034953 if inlist(dpston4, 2000053, 2024950)
Pass 12 through the data...
  smallest count = 1 in the cell      2025560
  Invoking rule 34953:25560=54960
  replace dpston4 = 2054960 if inlist(dpston4, 2034953, 2025560)
Pass 13 through the data...
  smallest count = 1 in the cell      3000039
  Invoking rule 39:40=23940
  replace dpston4 = 3023940 if inlist(dpston4, 3000039, 3000040)
Pass 14 through the data...
  smallest count = 1 in the cell      4000044
  Invoking rule 44:33640=43644
  replace dpston4 = 4043644 if inlist(dpston4, 4000044, 4033640)
Pass 15 through the data...
  smallest count = 1 in the cell      4024950
  Invoking rule 47:24950=34750
  replace dpston4 = 4034750 if inlist(dpston4, 4000047, 4024950)
Pass 16 through the data...
  smallest count = 1 in the cell      5000024
  Invoking rule 18:24=21824
  replace dpston4 = 5021824 if inlist(dpston4, 5000018, 5000024)
Pass 17 through the data...
  smallest count = 1 in the cell      5023940
  Invoking rule 44:23940=33944
  replace dpston4 = 5033944 if inlist(dpston4, 5000044, 5023940)
Pass 18 through the data...
  smallest count = 1 in the cell      5024950
  Invoking rule 53:24950=34953
  replace dpston4 = 5034953 if inlist(dpston4, 5000053, 5024950)
Pass 19 through the data...
  smallest count = 2 in the cell      2000036
  Invoking rule 36:33944=43644
  replace dpston4 = 2043644 if inlist(dpston4, 2000036, 2033944)
Pass 20 through the data...
  smallest count = 2 in the cell      4043644
  Invoking rule 43644:34750=73650
  replace dpston4 = 4073650 if inlist(dpston4, 4043644, 4034750)
Pass 21 through the data...
  smallest count = 2 in the cell      5025560
  Invoking rule 34953:25560=54960
  replace dpston4 = 5054960 if inlist(dpston4, 5034953, 5025560)
Pass 22 through the data...
  smallest count = 3 in the cell      2000024
  Invoking rule 18:24=21824
  replace dpston4 = 2021824 if inlist(dpston4, 2000018, 2000024)
Pass 23 through the data...
  smallest count = 3 in the cell      3023940
  Invoking rule 44:23940=33944
  replace dpston4 = 3033944 if inlist(dpston4, 3000044, 3023940)
Pass 24 through the data...
  smallest count = 3 in the cell      4000024
  Invoking rule 18:24=21824
  replace dpston4 = 4021824 if inlist(dpston4, 4000018, 4000024)
Pass 25 through the data...
  smallest count = 3 in the cell      5000001
  Invoking rule 1:2=20102
  replace dpston4 = 5020102 if inlist(dpston4, 5000001, 5000002)
Pass 26 through the data...
  smallest count = 3 in the cell      5000068
  Invoking rule 62:68=26268
  replace dpston4 = 5026268 if inlist(dpston4, 5000062, 5000068)
Pass 27 through the data...
  smallest count = 4 in the cell      2000001
  Invoking rule 1:2=20102
  replace dpston4 = 2020102 if inlist(dpston4, 2000001, 2000002)
Pass 28 through the data...
  smallest count = 4 in the cell      2000008
  Invoking rule 8:20102=30108
  replace dpston4 = 2030108 if inlist(dpston4, 2000008, 2020102)
Pass 29 through the data...
  smallest count = 4 in the cell      2043644
  Invoking rule 30:43644=53044
  replace dpston4 = 2053044 if inlist(dpston4, 2000030, 2043644)
Pass 30 through the data...
  smallest count = 4 in the cell      3000050
  Invoking rule 49:50=24950
  replace dpston4 = 3024950 if inlist(dpston4, 3000049, 3000050)
Pass 31 through the data...
  smallest count = 4 in the cell      5033944
  Invoking rule 47:33944=43947
  replace dpston4 = 5043947 if inlist(dpston4, 5000047, 5033944)
Pass 32 through the data...
  smallest count = 5 in the cell      1000039
  Invoking rule 39:40=23940
  replace dpston4 = 1023940 if inlist(dpston4, 1000039, 1000040)
Pass 33 through the data...
  smallest count = 5 in the cell      3000060
  Invoking rule 55:60=25560
  replace dpston4 = 3025560 if inlist(dpston4, 3000055, 3000060)
Pass 34 through the data...
  smallest count = 5 in the cell      4000026
  Invoking rule 26:21824=31826
  replace dpston4 = 4031826 if inlist(dpston4, 4000026, 4021824)
Pass 35 through the data...
  smallest count = 5 in the cell      5021824
  Invoking rule 26:21824=31826
  replace dpston4 = 5031826 if inlist(dpston4, 5000026, 5021824)
Pass 36 through the data...
  smallest count = 6 in the cell      2000068
  Invoking rule 62:68=26268
  replace dpston4 = 2026268 if inlist(dpston4, 2000062, 2000068)
Pass 37 through the data...
  smallest count = 6 in the cell      2054960
  Invoking rule 54960:26268=74968
  replace dpston4 = 2074968 if inlist(dpston4, 2054960, 2026268)
Pass 38 through the data...
  smallest count = 6 in the cell      4000002
  Invoking rule 1:2=20102
  replace dpston4 = 4020102 if inlist(dpston4, 4000001, 4000002)
Pass 39 through the data...
  smallest count = 7 in the cell      4054960
  Invoking rule 62:54960=64962
  replace dpston4 = 4064962 if inlist(dpston4, 4000062, 4054960)
Pass 40 through the data...
  smallest count = 8 in the cell      2021824
  Invoking rule 26:21824=31826
  replace dpston4 = 2031826 if inlist(dpston4, 2000026, 2021824)
Pass 41 through the data...
  smallest count = 8 in the cell      5054960
  Invoking rule 43947:54960=93960
  replace dpston4 = 5093960 if inlist(dpston4, 5043947, 5054960)
Pass 42 through the data...
  smallest count = 10 in the cell      1000055
  Invoking rule 55:60=25560
  replace dpston4 = 1025560 if inlist(dpston4, 1000055, 1000060)
Pass 43 through the data...
  smallest count = 10 in the cell      4073650
  Invoking rule 30:73650=83050
  replace dpston4 = 4083050 if inlist(dpston4, 4000030, 4073650)
Pass 44 through the data...
  smallest count = 10 in the cell      5020102
  Invoking rule 8:20102=30108
  replace dpston4 = 5030108 if inlist(dpston4, 5000008, 5020102)
Pass 45 through the data...
  smallest count = 11 in the cell      3000001
  Invoking rule 1:2=20102
  replace dpston4 = 3020102 if inlist(dpston4, 3000001, 3000002)
Pass 46 through the data...
  smallest count = 11 in the cell      4000068
  Invoking rule 68:64962=74968
  replace dpston4 = 4074968 if inlist(dpston4, 4000068, 4064962)
Pass 47 through the data...
  smallest count = 11 in the cell      5031826
  Invoking rule 30:31826=41830
  replace dpston4 = 5041830 if inlist(dpston4, 5000030, 5031826)
Pass 48 through the data...
  smallest count = 12 in the cell      2030108
  Invoking rule 11:30108=40111
  replace dpston4 = 2040111 if inlist(dpston4, 2000011, 2030108)
Pass 49 through the data...
  smallest count = 12 in the cell      3033944
  Invoking rule 36:33944=43644
  replace dpston4 = 3043644 if inlist(dpston4, 3000036, 3033944)
Pass 50 through the data...
  smallest count = 12 in the cell      4031826
  Invoking rule 11:31826=41126
  replace dpston4 = 4041126 if inlist(dpston4, 4000011, 4031826)
Pass 51 through the data...
  smallest count = 13 in the cell      1024950
  Invoking rule 53:24950=34953
  replace dpston4 = 1034953 if inlist(dpston4, 1000053, 1024950)
Pass 52 through the data...
  smallest count = 13 in the cell      3024950
  Invoking rule 53:24950=34953
  replace dpston4 = 3034953 if inlist(dpston4, 3000053, 3024950)
Pass 53 through the data...
  smallest count = 13 in the cell      4020102
  Invoking rule 8:20102=30108
  replace dpston4 = 4030108 if inlist(dpston4, 4000008, 4020102)
Pass 54 through the data...
  smallest count = 15 in the cell      5000036
  Invoking rule 36:41830=51836
  replace dpston4 = 5051836 if inlist(dpston4, 5000036, 5041830)
Pass 55 through the data...
  smallest count = 19 in the cell      3025560
  Invoking rule 62:25560=35562
  replace dpston4 = 3035562 if inlist(dpston4, 3000062, 3025560)
Pass 56 through the data...
  smallest count = 20 in the cell      5026268
  Done collapsing! Exiting...
{\smallskip}
. assert "`r(failed)'" == ""      
{\smallskip}
. wgtcellcollapse collapse, variables(daypart alight_id) mincellsize(1) ///
>         zeroes(2 40 49 50 60) greedy maxcategory(99) ///
>         generate(dpstoff4) saving(dpstoff4.do) replace run
Pass 0 through the data...
  smallest count = 1 in the cell      1000002
{\smallskip}
Processing zero cells...
{\smallskip}
  Invoking rule 39:40=23940 to collapse zero cells
  replace dpstoff4 = 1023940 if inlist(dpstoff4, 1000039, 1000040)
Pass 0 through the data...
  smallest count = 1 in the cell      1000002
  Invoking rule 1:2:8=30108 to collapse zero cells
  replace dpstoff4 = 2030108 if inlist(dpstoff4, 2000001, 2000002, 2000008)
Pass 0 through the data...
  smallest count = 1 in the cell      1000002
  Invoking rule 30:36:39:40:44=53044 to collapse zero cells
  replace dpstoff4 = 2053044 if inlist(dpstoff4, 2000030, 2000036, 2000039, 2000040, 2000044)
Pass 0 through the data...
  smallest count = 1 in the cell      1000002
  Invoking rule 50:53:55:60:62=55062 to collapse zero cells
  replace dpstoff4 = 2055062 if inlist(dpstoff4, 2000050, 2000053, 2000055, 2000060, 2000062)
Pass 0 through the data...
  smallest count = 1 in the cell      1000002
  Invoking rule 1:2:8=30108 to collapse zero cells
  replace dpstoff4 = 3030108 if inlist(dpstoff4, 3000001, 3000002, 3000008)
Pass 0 through the data...
  smallest count = 1 in the cell      1000002
  Invoking rule 55:60=25560 to collapse zero cells
  replace dpstoff4 = 3025560 if inlist(dpstoff4, 3000055, 3000060)
Pass 0 through the data...
  smallest count = 1 in the cell      1000002
  Invoking rule 1:2:8:11=40111 to collapse zero cells
  replace dpstoff4 = 4040111 if inlist(dpstoff4, 4000001, 4000002, 4000008, 4000011)
Pass 0 through the data...
  smallest count = 1 in the cell      1000002
  Invoking rule 36:39:40:44=43644 to collapse zero cells
  replace dpstoff4 = 4043644 if inlist(dpstoff4, 4000036, 4000039, 4000040, 4000044)
Pass 0 through the data...
  smallest count = 1 in the cell      1000002
  Invoking rule 49:50=24950 to collapse zero cells
  replace dpstoff4 = 4024950 if inlist(dpstoff4, 4000049, 4000050)
Pass 0 through the data...
  smallest count = 1 in the cell      1000002
  Invoking rule 55:60=25560 to collapse zero cells
  replace dpstoff4 = 4025560 if inlist(dpstoff4, 4000055, 4000060)
Pass 0 through the data...
  smallest count = 1 in the cell      1000002
  Invoking rule 1:2:8=30108 to collapse zero cells
  replace dpstoff4 = 5030108 if inlist(dpstoff4, 5000001, 5000002, 5000008)
Pass 0 through the data...
  smallest count = 1 in the cell      1000002
  Invoking rule 36:39:40=33640 to collapse zero cells
  replace dpstoff4 = 5033640 if inlist(dpstoff4, 5000036, 5000039, 5000040)
Pass 0 through the data...
  smallest count = 1 in the cell      1000002
  Invoking rule 49:50=24950 to collapse zero cells
  replace dpstoff4 = 5024950 if inlist(dpstoff4, 5000049, 5000050)
Pass 0 through the data...
  smallest count = 1 in the cell      1000002
  Invoking rule 49:50:53:55:60=54960 to collapse zero cells
  replace dpstoff4 = 5054960 if inlist(dpstoff4, 5000049, 5000050, 5000053, 5000055, 5000060)
Pass 0 through the data...
  smallest count = 1 in the cell      1000002
Pass 14 through the data...
  smallest count = 1 in the cell      1000002
  Done collapsing! Exiting...
{\smallskip}
. wgtcellcollapse collapse, variables(daypart alight_id) mincellsize(20) ///
>         strict feed(dpstoff4) saving(dpstoff4.do) append run
Pass 14 through the data...
  smallest count = 1 in the cell      1000002
  Invoking rule 2:8=20208
  replace dpstoff4 = 1020208 if inlist(dpstoff4, 1000002, 1000008)
Pass 15 through the data...
  smallest count = 1 in the cell      2000011
  Invoking rule 11:18=21118
  replace dpstoff4 = 2021118 if inlist(dpstoff4, 2000011, 2000018)
Pass 16 through the data...
  smallest count = 1 in the cell      2000024
  Invoking rule 24:26=22426
  replace dpstoff4 = 2022426 if inlist(dpstoff4, 2000024, 2000026)
Pass 17 through the data...
  smallest count = 1 in the cell      2000049
  Invoking rule 49:55062=64962
  replace dpstoff4 = 2064962 if inlist(dpstoff4, 2000049, 2055062)
Pass 18 through the data...
  smallest count = 1 in the cell      2030108
  Invoking rule 30108:21118=50118
  replace dpstoff4 = 2050118 if inlist(dpstoff4, 2030108, 2021118)
Pass 19 through the data...
  smallest count = 1 in the cell      3000039
  Invoking rule 39:40=23940
  replace dpstoff4 = 3023940 if inlist(dpstoff4, 3000039, 3000040)
Pass 20 through the data...
  smallest count = 1 in the cell      3000049
  Invoking rule 49:50=24950
  replace dpstoff4 = 3024950 if inlist(dpstoff4, 3000049, 3000050)
Pass 21 through the data...
  smallest count = 1 in the cell      3030108
  Invoking rule 11:30108=40111
  replace dpstoff4 = 3040111 if inlist(dpstoff4, 3000011, 3030108)
Pass 22 through the data...
  smallest count = 1 in the cell      4000018
  Invoking rule 18:24=21824
  replace dpstoff4 = 4021824 if inlist(dpstoff4, 4000018, 4000024)
Pass 23 through the data...
  smallest count = 1 in the cell      4000026
  Invoking rule 26:21824=31826
  replace dpstoff4 = 4031826 if inlist(dpstoff4, 4000026, 4021824)
Pass 24 through the data...
  smallest count = 1 in the cell      4025560
  Invoking rule 53:25560=35360
  replace dpstoff4 = 4035360 if inlist(dpstoff4, 4000053, 4025560)
Pass 25 through the data...
  smallest count = 1 in the cell      5000026
  Invoking rule 24:26=22426
  replace dpstoff4 = 5022426 if inlist(dpstoff4, 5000024, 5000026)
Pass 26 through the data...
  smallest count = 1 in the cell      5024950
  Invoking rule 47:24950=34750
  replace dpstoff4 = 5034750 if inlist(dpstoff4, 5000047, 5024950)
Pass 27 through the data...
  smallest count = 1 in the cell      5030108
  Invoking rule 11:30108=40111
  replace dpstoff4 = 5040111 if inlist(dpstoff4, 5000011, 5030108)
Pass 28 through the data...
  smallest count = 2 in the cell      1000050
  Invoking rule 49:50=24950
  replace dpstoff4 = 1024950 if inlist(dpstoff4, 1000049, 1000050)
Pass 29 through the data...
  smallest count = 2 in the cell      2064962
  Invoking rule 68:64962=74968
  replace dpstoff4 = 2074968 if inlist(dpstoff4, 2000068, 2064962)
Pass 30 through the data...
  smallest count = 2 in the cell      3000024
  Invoking rule 24:26=22426
  replace dpstoff4 = 3022426 if inlist(dpstoff4, 3000024, 3000026)
Pass 31 through the data...
  smallest count = 2 in the cell      3000044
  Invoking rule 44:23940=33944
  replace dpstoff4 = 3033944 if inlist(dpstoff4, 3000044, 3023940)
Pass 32 through the data...
  smallest count = 2 in the cell      3024950
  Invoking rule 53:24950=34953
  replace dpstoff4 = 3034953 if inlist(dpstoff4, 3000053, 3024950)
Pass 33 through the data...
  smallest count = 2 in the cell      4024950
  Invoking rule 24950:35360=54960
  replace dpstoff4 = 4054960 if inlist(dpstoff4, 4024950, 4035360)
Pass 34 through the data...
  smallest count = 2 in the cell      4043644
  Invoking rule 47:43644=53647
  replace dpstoff4 = 4053647 if inlist(dpstoff4, 4000047, 4043644)
Pass 35 through the data...
  smallest count = 2 in the cell      5000044
  Invoking rule 44:33640=43644
  replace dpstoff4 = 5043644 if inlist(dpstoff4, 5000044, 5033640)
Pass 36 through the data...
  smallest count = 2 in the cell      5054960
  Invoking rule 62:54960=64962
  replace dpstoff4 = 5064962 if inlist(dpstoff4, 5000062, 5054960)
Pass 37 through the data...
  smallest count = 3 in the cell      1023940
  Invoking rule 36:23940=33640
  replace dpstoff4 = 1033640 if inlist(dpstoff4, 1000036, 1023940)
Pass 38 through the data...
  smallest count = 3 in the cell      2022426
  Invoking rule 50118:22426=70126
  replace dpstoff4 = 2070126 if inlist(dpstoff4, 2050118, 2022426)
Pass 39 through the data...
  smallest count = 3 in the cell      4000062
  Invoking rule 62:68=26268
  replace dpstoff4 = 4026268 if inlist(dpstoff4, 4000062, 4000068)
Pass 40 through the data...
  smallest count = 3 in the cell      5022426
  Invoking rule 18:22426=31826
  replace dpstoff4 = 5031826 if inlist(dpstoff4, 5000018, 5022426)
Pass 41 through the data...
  smallest count = 4 in the cell      2053044
  Invoking rule 47:53044=63047
  replace dpstoff4 = 2063047 if inlist(dpstoff4, 2000047, 2053044)
Pass 42 through the data...
  smallest count = 4 in the cell      3000036
  Invoking rule 36:33944=43644
  replace dpstoff4 = 3043644 if inlist(dpstoff4, 3000036, 3033944)
Pass 43 through the data...
  smallest count = 4 in the cell      4031826
  Invoking rule 40111:31826=70126
  replace dpstoff4 = 4070126 if inlist(dpstoff4, 4040111, 4031826)
Pass 44 through the data...
  smallest count = 4 in the cell      5043644
  Invoking rule 43644:34750=73650
  replace dpstoff4 = 5073650 if inlist(dpstoff4, 5043644, 5034750)
Pass 45 through the data...
  smallest count = 5 in the cell      1000060
  Invoking rule 55:60=25560
  replace dpstoff4 = 1025560 if inlist(dpstoff4, 1000055, 1000060)
Pass 46 through the data...
  smallest count = 5 in the cell      1024950
  Invoking rule 53:24950=34953
  replace dpstoff4 = 1034953 if inlist(dpstoff4, 1000053, 1024950)
Pass 47 through the data...
  smallest count = 5 in the cell      2074968
  Invoking rule 63047:74968=133068
  replace dpstoff4 = 2133068 if inlist(dpstoff4, 2063047, 2074968)
Pass 48 through the data...
  smallest count = 5 in the cell      3025560
  Invoking rule 34953:25560=54960
  replace dpstoff4 = 3054960 if inlist(dpstoff4, 3034953, 3025560)
Pass 49 through the data...
  smallest count = 6 in the cell      2070126
  Invoking rule 70126:133068=200168
  replace dpstoff4 = 2200168 if inlist(dpstoff4, 2070126, 2133068)
Pass 50 through the data...
  smallest count = 6 in the cell      4054960
  Invoking rule 53647:54960=103660
  replace dpstoff4 = 4103660 if inlist(dpstoff4, 4053647, 4054960)
Pass 51 through the data...
  smallest count = 6 in the cell      5040111
  Invoking rule 40111:31826=70126
  replace dpstoff4 = 5070126 if inlist(dpstoff4, 5040111, 5031826)
Pass 52 through the data...
  smallest count = 7 in the cell      5000068
  Invoking rule 68:64962=74968
  replace dpstoff4 = 5074968 if inlist(dpstoff4, 5000068, 5064962)
Pass 53 through the data...
  smallest count = 8 in the cell      5000030
  Invoking rule 30:73650=83050
  replace dpstoff4 = 5083050 if inlist(dpstoff4, 5000030, 5073650)
Pass 54 through the data...
  smallest count = 9 in the cell      4026268
  Invoking rule 103660:26268=123668
  replace dpstoff4 = 4123668 if inlist(dpstoff4, 4103660, 4026268)
Pass 55 through the data...
  smallest count = 10 in the cell      3022426
  Invoking rule 18:22426=31826
  replace dpstoff4 = 3031826 if inlist(dpstoff4, 3000018, 3022426)
Pass 56 through the data...
  smallest count = 10 in the cell      3043644
  Invoking rule 30:43644=53044
  replace dpstoff4 = 3053044 if inlist(dpstoff4, 3000030, 3043644)
Pass 57 through the data...
  smallest count = 10 in the cell      4070126
  Invoking rule 30:70126=80130
  replace dpstoff4 = 4080130 if inlist(dpstoff4, 4000030, 4070126)
Pass 58 through the data...
  smallest count = 11 in the cell      1025560
  Invoking rule 34953:25560=54960
  replace dpstoff4 = 1054960 if inlist(dpstoff4, 1034953, 1025560)
Pass 59 through the data...
  smallest count = 12 in the cell      1020208
  Invoking rule 11:20208=30211
  replace dpstoff4 = 1030211 if inlist(dpstoff4, 1000011, 1020208)
Pass 60 through the data...
  smallest count = 12 in the cell      1033640
  Invoking rule 44:33640=43644
  replace dpstoff4 = 1043644 if inlist(dpstoff4, 1000044, 1033640)
Pass 61 through the data...
  smallest count = 13 in the cell      5070126
  Invoking rule 70126:83050=150150
  replace dpstoff4 = 5150150 if inlist(dpstoff4, 5070126, 5083050)
Pass 62 through the data...
  smallest count = 15 in the cell      1000024
  Invoking rule 24:26=22426
  replace dpstoff4 = 1022426 if inlist(dpstoff4, 1000024, 1000026)
Pass 63 through the data...
  smallest count = 15 in the cell      3040111
  Invoking rule 40111:31826=70126
  replace dpstoff4 = 3070126 if inlist(dpstoff4, 3040111, 3031826)
Pass 64 through the data...
  smallest count = 15 in the cell      3054960
  Invoking rule 62:54960=64962
  replace dpstoff4 = 3064962 if inlist(dpstoff4, 3000062, 3054960)
Pass 65 through the data...
  smallest count = 18 in the cell      5074968
  Invoking rule 69:74968=84969
  replace dpstoff4 = 5084969 if inlist(dpstoff4, 5000069, 5074968)
Pass 66 through the data...
  smallest count = 21 in the cell      2200168
  Done collapsing! Exiting...
{\smallskip}
. assert "`r(failed)'" == ""      
{\smallskip}
\nullskip
\end{stlog}

We have finally been able to produce a clean collapse of everything! Note the use of 
\stcmd{assert "`r(failed)'"==""} in the above code snippet to make sure that all cells have 
the minimal required size of 20.

As a very minor point, we can see some room for improvement in collapsing the cells on the weekend:

\begin{stlog}
. tab alight_id dpstoff4 if daypart == 5 \& mod(dpstoff4,100*100)>100
{\smallskip}
                      {\VBAR}    Long ID of the
                      {\VBAR}      interaction
            alight_id {\VBAR}   5084969    5150150 {\VBAR}     Total
\HLI{22}{\PLUS}\HLI{22}{\PLUS}\HLI{10}
         8. Carmenton {\VBAR}         0          1 {\VBAR}         1 
         11. Dogville {\VBAR}         0          5 {\VBAR}         5 
         18. East End {\VBAR}         0          4 {\VBAR}         4 
       24. Framington {\VBAR}         0          2 {\VBAR}         2 
   26. Grand Junction {\VBAR}         0          1 {\VBAR}         1 
       30. High Point {\VBAR}         0          8 {\VBAR}         8 
       36. Irvingtown {\VBAR}         0          2 {\VBAR}         2 
         44. Limerick {\VBAR}         0          2 {\VBAR}         2 
      47. Moscow City {\VBAR}         0          6 {\VBAR}         6 
     50. Ontario Lake {\VBAR}         0          1 {\VBAR}         1 
 53. Picadilly Square {\VBAR}         2          0 {\VBAR}         2 
    62. Silver Spring {\VBAR}         9          0 {\VBAR}         9 
      68. Toledo Town {\VBAR}         7          0 {\VBAR}         7 
    69. Union Station {\VBAR}       123          0 {\VBAR}       123 
\HLI{22}{\PLUS}\HLI{22}{\PLUS}\HLI{10}
                Total {\VBAR}       141         32 {\VBAR}       173 
{\smallskip}
{\smallskip}
\nullskip
\end{stlog}

Instead of two cells with sizes 141 and 32, it seems like we could produce three cells, with
Union Station (69) being its own cell, and everything else split somewhere in the middle.

\subsubsection{The fifth pass of cell collapse and raking: \ifexp\ conditions}

We will now code the collapsing cells for that day part ``by hand'', and we will put those
custom coded cells upfront before the main run. (Some special treatment had to be given
to the zero cells to avoid overlapping cells around Ninth Street (49) for the night and weekend
day parts; without the separation, it is getting collapsed in a long overreach all the way up to
Redline Circle (60).)

\begin{stlog}
. use trip_sample_rules, clear
{\smallskip}
. wgtcellcollapse collapse, variables(daypart board_id) mincellsize(1) ///
>         zeroes(39 44 49 60) greedy maxcategory(99) ///
>         generate(dpston5) saving(dpston5.do) replace run
Pass 0 through the data...
  smallest count = 1 in the cell      2000039
{\smallskip}
Processing zero cells...
{\smallskip}
  Invoking rule 49:50=24950 to collapse zero cells
  replace dpston5 = 1024950 if inlist(dpston5, 1000049, 1000050)
Pass 0 through the data...
  smallest count = 1 in the cell      2000039
  Invoking rule 40:44=24044 to collapse zero cells
  replace dpston5 = 2024044 if inlist(dpston5, 2000040, 2000044)
Pass 0 through the data...
  smallest count = 1 in the cell      2000039
  Invoking rule 49:50=24950 to collapse zero cells
  replace dpston5 = 2024950 if inlist(dpston5, 2000049, 2000050)
Pass 0 through the data...
  smallest count = 1 in the cell      2000039
  Invoking rule 55:60=25560 to collapse zero cells
  replace dpston5 = 2025560 if inlist(dpston5, 2000055, 2000060)
Pass 0 through the data...
  smallest count = 1 in the cell      2000039
  Invoking rule 36:39:40=33640 to collapse zero cells
  replace dpston5 = 4033640 if inlist(dpston5, 4000036, 4000039, 4000040)
Pass 0 through the data...
  smallest count = 1 in the cell      2000039
  Invoking rule 49:50=24950 to collapse zero cells
  replace dpston5 = 4024950 if inlist(dpston5, 4000049, 4000050)
Pass 0 through the data...
  smallest count = 1 in the cell      2000039
  Invoking rule 49:50:53:55:60=54960 to collapse zero cells
  replace dpston5 = 4054960 if inlist(dpston5, 4000049, 4000050, 4000053, 4000055, 4000060)
Pass 0 through the data...
  smallest count = 1 in the cell      2000039
  Invoking rule 39:40=23940 to collapse zero cells
  replace dpston5 = 5023940 if inlist(dpston5, 5000039, 5000040)
Pass 0 through the data...
  smallest count = 1 in the cell      2000039
  Invoking rule 49:50=24950 to collapse zero cells
  replace dpston5 = 5024950 if inlist(dpston5, 5000049, 5000050)
Pass 0 through the data...
  smallest count = 1 in the cell      2000039
  Invoking rule 55:60=25560 to collapse zero cells
  replace dpston5 = 5025560 if inlist(dpston5, 5000055, 5000060)
Pass 0 through the data...
  smallest count = 1 in the cell      2000039
Pass 10 through the data...
  smallest count = 1 in the cell      2000039
  Done collapsing! Exiting...
{\smallskip}
. wgtcellcollapse collapse, variables(daypart board_id) mincellsize(20) ///
>         strict feed(dpston5) saving(dpston5.do) append run
Pass 10 through the data...
  smallest count = 1 in the cell      2000039
  Invoking rule 39:24044=33944
  replace dpston5 = 2033944 if inlist(dpston5, 2000039, 2024044)
Pass 11 through the data...
  smallest count = 1 in the cell      2024950
  Invoking rule 53:24950=34953
  replace dpston5 = 2034953 if inlist(dpston5, 2000053, 2024950)
Pass 12 through the data...
  smallest count = 1 in the cell      2025560
  Invoking rule 34953:25560=54960
  replace dpston5 = 2054960 if inlist(dpston5, 2034953, 2025560)
Pass 13 through the data...
  smallest count = 1 in the cell      3000039
  Invoking rule 39:40=23940
  replace dpston5 = 3023940 if inlist(dpston5, 3000039, 3000040)
Pass 14 through the data...
  smallest count = 1 in the cell      4000044
  Invoking rule 44:33640=43644
  replace dpston5 = 4043644 if inlist(dpston5, 4000044, 4033640)
Pass 15 through the data...
  smallest count = 1 in the cell      4024950
  Invoking rule 47:24950=34750
  replace dpston5 = 4034750 if inlist(dpston5, 4000047, 4024950)
Pass 16 through the data...
  smallest count = 1 in the cell      5000024
  Invoking rule 18:24=21824
  replace dpston5 = 5021824 if inlist(dpston5, 5000018, 5000024)
Pass 17 through the data...
  smallest count = 1 in the cell      5023940
  Invoking rule 44:23940=33944
  replace dpston5 = 5033944 if inlist(dpston5, 5000044, 5023940)
Pass 18 through the data...
  smallest count = 1 in the cell      5024950
  Invoking rule 53:24950=34953
  replace dpston5 = 5034953 if inlist(dpston5, 5000053, 5024950)
Pass 19 through the data...
  smallest count = 2 in the cell      2000036
  Invoking rule 36:33944=43644
  replace dpston5 = 2043644 if inlist(dpston5, 2000036, 2033944)
Pass 20 through the data...
  smallest count = 2 in the cell      4043644
  Invoking rule 43644:34750=73650
  replace dpston5 = 4073650 if inlist(dpston5, 4043644, 4034750)
Pass 21 through the data...
  smallest count = 2 in the cell      5025560
  Invoking rule 34953:25560=54960
  replace dpston5 = 5054960 if inlist(dpston5, 5034953, 5025560)
Pass 22 through the data...
  smallest count = 3 in the cell      2000024
  Invoking rule 18:24=21824
  replace dpston5 = 2021824 if inlist(dpston5, 2000018, 2000024)
Pass 23 through the data...
  smallest count = 3 in the cell      3023940
  Invoking rule 44:23940=33944
  replace dpston5 = 3033944 if inlist(dpston5, 3000044, 3023940)
Pass 24 through the data...
  smallest count = 3 in the cell      4000024
  Invoking rule 18:24=21824
  replace dpston5 = 4021824 if inlist(dpston5, 4000018, 4000024)
Pass 25 through the data...
  smallest count = 3 in the cell      5000001
  Invoking rule 1:2=20102
  replace dpston5 = 5020102 if inlist(dpston5, 5000001, 5000002)
Pass 26 through the data...
  smallest count = 3 in the cell      5000068
  Invoking rule 62:68=26268
  replace dpston5 = 5026268 if inlist(dpston5, 5000062, 5000068)
Pass 27 through the data...
  smallest count = 4 in the cell      2000001
  Invoking rule 1:2=20102
  replace dpston5 = 2020102 if inlist(dpston5, 2000001, 2000002)
Pass 28 through the data...
  smallest count = 4 in the cell      2000008
  Invoking rule 8:20102=30108
  replace dpston5 = 2030108 if inlist(dpston5, 2000008, 2020102)
Pass 29 through the data...
  smallest count = 4 in the cell      2043644
  Invoking rule 30:43644=53044
  replace dpston5 = 2053044 if inlist(dpston5, 2000030, 2043644)
Pass 30 through the data...
  smallest count = 4 in the cell      3000050
  Invoking rule 49:50=24950
  replace dpston5 = 3024950 if inlist(dpston5, 3000049, 3000050)
Pass 31 through the data...
  smallest count = 4 in the cell      5033944
  Invoking rule 47:33944=43947
  replace dpston5 = 5043947 if inlist(dpston5, 5000047, 5033944)
Pass 32 through the data...
  smallest count = 5 in the cell      1000039
  Invoking rule 39:40=23940
  replace dpston5 = 1023940 if inlist(dpston5, 1000039, 1000040)
Pass 33 through the data...
  smallest count = 5 in the cell      3000060
  Invoking rule 55:60=25560
  replace dpston5 = 3025560 if inlist(dpston5, 3000055, 3000060)
Pass 34 through the data...
  smallest count = 5 in the cell      4000026
  Invoking rule 26:21824=31826
  replace dpston5 = 4031826 if inlist(dpston5, 4000026, 4021824)
Pass 35 through the data...
  smallest count = 5 in the cell      5021824
  Invoking rule 26:21824=31826
  replace dpston5 = 5031826 if inlist(dpston5, 5000026, 5021824)
Pass 36 through the data...
  smallest count = 6 in the cell      2000068
  Invoking rule 62:68=26268
  replace dpston5 = 2026268 if inlist(dpston5, 2000062, 2000068)
Pass 37 through the data...
  smallest count = 6 in the cell      2054960
  Invoking rule 54960:26268=74968
  replace dpston5 = 2074968 if inlist(dpston5, 2054960, 2026268)
Pass 38 through the data...
  smallest count = 6 in the cell      4000002
  Invoking rule 1:2=20102
  replace dpston5 = 4020102 if inlist(dpston5, 4000001, 4000002)
Pass 39 through the data...
  smallest count = 7 in the cell      4054960
  Invoking rule 62:54960=64962
  replace dpston5 = 4064962 if inlist(dpston5, 4000062, 4054960)
Pass 40 through the data...
  smallest count = 8 in the cell      2021824
  Invoking rule 26:21824=31826
  replace dpston5 = 2031826 if inlist(dpston5, 2000026, 2021824)
Pass 41 through the data...
  smallest count = 8 in the cell      5054960
  Invoking rule 43947:54960=93960
  replace dpston5 = 5093960 if inlist(dpston5, 5043947, 5054960)
Pass 42 through the data...
  smallest count = 10 in the cell      1000055
  Invoking rule 55:60=25560
  replace dpston5 = 1025560 if inlist(dpston5, 1000055, 1000060)
Pass 43 through the data...
  smallest count = 10 in the cell      4073650
  Invoking rule 30:73650=83050
  replace dpston5 = 4083050 if inlist(dpston5, 4000030, 4073650)
Pass 44 through the data...
  smallest count = 10 in the cell      5020102
  Invoking rule 8:20102=30108
  replace dpston5 = 5030108 if inlist(dpston5, 5000008, 5020102)
Pass 45 through the data...
  smallest count = 11 in the cell      3000001
  Invoking rule 1:2=20102
  replace dpston5 = 3020102 if inlist(dpston5, 3000001, 3000002)
Pass 46 through the data...
  smallest count = 11 in the cell      4000068
  Invoking rule 68:64962=74968
  replace dpston5 = 4074968 if inlist(dpston5, 4000068, 4064962)
Pass 47 through the data...
  smallest count = 11 in the cell      5031826
  Invoking rule 30:31826=41830
  replace dpston5 = 5041830 if inlist(dpston5, 5000030, 5031826)
Pass 48 through the data...
  smallest count = 12 in the cell      2030108
  Invoking rule 11:30108=40111
  replace dpston5 = 2040111 if inlist(dpston5, 2000011, 2030108)
Pass 49 through the data...
  smallest count = 12 in the cell      3033944
  Invoking rule 36:33944=43644
  replace dpston5 = 3043644 if inlist(dpston5, 3000036, 3033944)
Pass 50 through the data...
  smallest count = 12 in the cell      4031826
  Invoking rule 11:31826=41126
  replace dpston5 = 4041126 if inlist(dpston5, 4000011, 4031826)
Pass 51 through the data...
  smallest count = 13 in the cell      1024950
  Invoking rule 53:24950=34953
  replace dpston5 = 1034953 if inlist(dpston5, 1000053, 1024950)
Pass 52 through the data...
  smallest count = 13 in the cell      3024950
  Invoking rule 53:24950=34953
  replace dpston5 = 3034953 if inlist(dpston5, 3000053, 3024950)
Pass 53 through the data...
  smallest count = 13 in the cell      4020102
  Invoking rule 8:20102=30108
  replace dpston5 = 4030108 if inlist(dpston5, 4000008, 4020102)
Pass 54 through the data...
  smallest count = 15 in the cell      5000036
  Invoking rule 36:41830=51836
  replace dpston5 = 5051836 if inlist(dpston5, 5000036, 5041830)
Pass 55 through the data...
  smallest count = 19 in the cell      3025560
  Invoking rule 62:25560=35562
  replace dpston5 = 3035562 if inlist(dpston5, 3000062, 3025560)
Pass 56 through the data...
  smallest count = 20 in the cell      5026268
  Done collapsing! Exiting...
{\smallskip}
. assert "`r(failed)'" == ""      
{\smallskip}
. wgtcellcollapse collapse, variables(daypart alight_id) mincellsize(1) ///
>         zeroes(2 40 49 50 60) greedy maxcategory(99) ///
>         generate(dpstoff5) saving(dpstoff5.do) replace run
Pass 0 through the data...
  smallest count = 1 in the cell      1000002
{\smallskip}
Processing zero cells...
{\smallskip}
  Invoking rule 39:40=23940 to collapse zero cells
  replace dpstoff5 = 1023940 if inlist(dpstoff5, 1000039, 1000040)
Pass 0 through the data...
  smallest count = 1 in the cell      1000002
  Invoking rule 1:2:8=30108 to collapse zero cells
  replace dpstoff5 = 2030108 if inlist(dpstoff5, 2000001, 2000002, 2000008)
Pass 0 through the data...
  smallest count = 1 in the cell      1000002
  Invoking rule 30:36:39:40:44=53044 to collapse zero cells
  replace dpstoff5 = 2053044 if inlist(dpstoff5, 2000030, 2000036, 2000039, 2000040, 2000044)
Pass 0 through the data...
  smallest count = 1 in the cell      1000002
  Invoking rule 50:53:55:60:62=55062 to collapse zero cells
  replace dpstoff5 = 2055062 if inlist(dpstoff5, 2000050, 2000053, 2000055, 2000060, 2000062)
Pass 0 through the data...
  smallest count = 1 in the cell      1000002
  Invoking rule 1:2:8=30108 to collapse zero cells
  replace dpstoff5 = 3030108 if inlist(dpstoff5, 3000001, 3000002, 3000008)
Pass 0 through the data...
  smallest count = 1 in the cell      1000002
  Invoking rule 55:60=25560 to collapse zero cells
  replace dpstoff5 = 3025560 if inlist(dpstoff5, 3000055, 3000060)
Pass 0 through the data...
  smallest count = 1 in the cell      1000002
  Invoking rule 1:2:8:11=40111 to collapse zero cells
  replace dpstoff5 = 4040111 if inlist(dpstoff5, 4000001, 4000002, 4000008, 4000011)
Pass 0 through the data...
  smallest count = 1 in the cell      1000002
  Invoking rule 36:39:40:44=43644 to collapse zero cells
  replace dpstoff5 = 4043644 if inlist(dpstoff5, 4000036, 4000039, 4000040, 4000044)
Pass 0 through the data...
  smallest count = 1 in the cell      1000002
  Invoking rule 49:50=24950 to collapse zero cells
  replace dpstoff5 = 4024950 if inlist(dpstoff5, 4000049, 4000050)
Pass 0 through the data...
  smallest count = 1 in the cell      1000002
  Invoking rule 55:60=25560 to collapse zero cells
  replace dpstoff5 = 4025560 if inlist(dpstoff5, 4000055, 4000060)
Pass 0 through the data...
  smallest count = 1 in the cell      1000002
  Invoking rule 1:2:8=30108 to collapse zero cells
  replace dpstoff5 = 5030108 if inlist(dpstoff5, 5000001, 5000002, 5000008)
Pass 0 through the data...
  smallest count = 1 in the cell      1000002
  Invoking rule 36:39:40=33640 to collapse zero cells
  replace dpstoff5 = 5033640 if inlist(dpstoff5, 5000036, 5000039, 5000040)
Pass 0 through the data...
  smallest count = 1 in the cell      1000002
  Invoking rule 49:50=24950 to collapse zero cells
  replace dpstoff5 = 5024950 if inlist(dpstoff5, 5000049, 5000050)
Pass 0 through the data...
  smallest count = 1 in the cell      1000002
  Invoking rule 49:50:53:55:60=54960 to collapse zero cells
  replace dpstoff5 = 5054960 if inlist(dpstoff5, 5000049, 5000050, 5000053, 5000055, 5000060)
Pass 0 through the data...
  smallest count = 1 in the cell      1000002
Pass 14 through the data...
  smallest count = 1 in the cell      1000002
  Done collapsing! Exiting...
{\smallskip}
. * special cells for weekend
. wgtcellcollapse collapse if daypart==5 \& inrange(alight_id,1,30), ///
>         variables(daypart alight_id) mincellsize(20) ///
>         strict feed(dpstoff5) saving(dpstoff5.do) append run
Pass 14 through the data...
  smallest count = 1 in the cell      5000026
  Invoking rule 24:26=22426
  replace dpstoff5 = 5022426 if inlist(dpstoff5, 5000024, 5000026)
Pass 15 through the data...
  smallest count = 1 in the cell      5030108
  Invoking rule 11:30108=40111
  replace dpstoff5 = 5040111 if inlist(dpstoff5, 5000011, 5030108)
Pass 16 through the data...
  smallest count = 3 in the cell      5022426
  Invoking rule 18:22426=31826
  replace dpstoff5 = 5031826 if inlist(dpstoff5, 5000018, 5022426)
Pass 17 through the data...
  smallest count = 6 in the cell      5040111
  Invoking rule 40111:31826=70126
  replace dpstoff5 = 5070126 if inlist(dpstoff5, 5040111, 5031826)
Pass 18 through the data...
  smallest count = 8 in the cell      5000030
  Invoking rule 30:70126=80130
  replace dpstoff5 = 5080130 if inlist(dpstoff5, 5000030, 5070126)
Pass 19 through the data...
  smallest count = 21 in the cell      5080130
  Done collapsing! Exiting...
{\smallskip}
. wgtcellcollapse collapse if daypart==5 \& inrange(alight_id,36,68), ///
>         variables(daypart alight_id) mincellsize(20) ///
>         strict feed(dpstoff5) saving(dpstoff5.do) append run
Pass 19 through the data...
  smallest count = 1 in the cell      5024950
  Invoking rule 47:24950=34750
  replace dpstoff5 = 5034750 if inlist(dpstoff5, 5000047, 5024950)
Pass 20 through the data...
  smallest count = 2 in the cell      5000044
  Invoking rule 44:33640=43644
  replace dpstoff5 = 5043644 if inlist(dpstoff5, 5000044, 5033640)
Pass 21 through the data...
  smallest count = 2 in the cell      5054960
  Invoking rule 62:54960=64962
  replace dpstoff5 = 5064962 if inlist(dpstoff5, 5000062, 5054960)
Pass 22 through the data...
  smallest count = 4 in the cell      5043644
  Invoking rule 43644:34750=73650
  replace dpstoff5 = 5073650 if inlist(dpstoff5, 5043644, 5034750)
Pass 23 through the data...
  smallest count = 7 in the cell      5000068
  Invoking rule 68:64962=74968
  replace dpstoff5 = 5074968 if inlist(dpstoff5, 5000068, 5064962)
Pass 24 through the data...
  smallest count = 11 in the cell      5073650
  WARNING: could not find any rules to collapse dpstoff5 == 5073650
Pass 25 through the data...
  smallest count = 18 in the cell      5074968
  WARNING: could not find any rules to collapse dpstoff5 == 5074968
Pass 26 through the data...
  smallest count = .i in the cell      1000002
  Done collapsing! Exiting...
{\smallskip}
. * all other cells
. wgtcellcollapse collapse, variables(daypart alight_id) mincellsize(20) ///
>         strict feed(dpstoff5) saving(dpstoff5.do) append run
Pass 24 through the data...
  smallest count = 1 in the cell      1000002
  Invoking rule 2:8=20208
  replace dpstoff5 = 1020208 if inlist(dpstoff5, 1000002, 1000008)
Pass 25 through the data...
  smallest count = 1 in the cell      2000011
  Invoking rule 11:18=21118
  replace dpstoff5 = 2021118 if inlist(dpstoff5, 2000011, 2000018)
Pass 26 through the data...
  smallest count = 1 in the cell      2000024
  Invoking rule 24:26=22426
  replace dpstoff5 = 2022426 if inlist(dpstoff5, 2000024, 2000026)
Pass 27 through the data...
  smallest count = 1 in the cell      2000049
  Invoking rule 49:55062=64962
  replace dpstoff5 = 2064962 if inlist(dpstoff5, 2000049, 2055062)
Pass 28 through the data...
  smallest count = 1 in the cell      2030108
  Invoking rule 30108:21118=50118
  replace dpstoff5 = 2050118 if inlist(dpstoff5, 2030108, 2021118)
Pass 29 through the data...
  smallest count = 1 in the cell      3000039
  Invoking rule 39:40=23940
  replace dpstoff5 = 3023940 if inlist(dpstoff5, 3000039, 3000040)
Pass 30 through the data...
  smallest count = 1 in the cell      3000049
  Invoking rule 49:50=24950
  replace dpstoff5 = 3024950 if inlist(dpstoff5, 3000049, 3000050)
Pass 31 through the data...
  smallest count = 1 in the cell      3030108
  Invoking rule 11:30108=40111
  replace dpstoff5 = 3040111 if inlist(dpstoff5, 3000011, 3030108)
Pass 32 through the data...
  smallest count = 1 in the cell      4000018
  Invoking rule 18:24=21824
  replace dpstoff5 = 4021824 if inlist(dpstoff5, 4000018, 4000024)
Pass 33 through the data...
  smallest count = 1 in the cell      4000026
  Invoking rule 26:21824=31826
  replace dpstoff5 = 4031826 if inlist(dpstoff5, 4000026, 4021824)
Pass 34 through the data...
  smallest count = 1 in the cell      4025560
  Invoking rule 53:25560=35360
  replace dpstoff5 = 4035360 if inlist(dpstoff5, 4000053, 4025560)
Pass 35 through the data...
  smallest count = 2 in the cell      1000050
  Invoking rule 49:50=24950
  replace dpstoff5 = 1024950 if inlist(dpstoff5, 1000049, 1000050)
Pass 36 through the data...
  smallest count = 2 in the cell      2064962
  Invoking rule 68:64962=74968
  replace dpstoff5 = 2074968 if inlist(dpstoff5, 2000068, 2064962)
Pass 37 through the data...
  smallest count = 2 in the cell      3000024
  Invoking rule 24:26=22426
  replace dpstoff5 = 3022426 if inlist(dpstoff5, 3000024, 3000026)
Pass 38 through the data...
  smallest count = 2 in the cell      3000044
  Invoking rule 44:23940=33944
  replace dpstoff5 = 3033944 if inlist(dpstoff5, 3000044, 3023940)
Pass 39 through the data...
  smallest count = 2 in the cell      3024950
  Invoking rule 53:24950=34953
  replace dpstoff5 = 3034953 if inlist(dpstoff5, 3000053, 3024950)
Pass 40 through the data...
  smallest count = 2 in the cell      4024950
  Invoking rule 24950:35360=54960
  replace dpstoff5 = 4054960 if inlist(dpstoff5, 4024950, 4035360)
Pass 41 through the data...
  smallest count = 2 in the cell      4043644
  Invoking rule 47:43644=53647
  replace dpstoff5 = 4053647 if inlist(dpstoff5, 4000047, 4043644)
Pass 42 through the data...
  smallest count = 3 in the cell      1023940
  Invoking rule 36:23940=33640
  replace dpstoff5 = 1033640 if inlist(dpstoff5, 1000036, 1023940)
Pass 43 through the data...
  smallest count = 3 in the cell      2022426
  Invoking rule 50118:22426=70126
  replace dpstoff5 = 2070126 if inlist(dpstoff5, 2050118, 2022426)
Pass 44 through the data...
  smallest count = 3 in the cell      4000062
  Invoking rule 62:68=26268
  replace dpstoff5 = 4026268 if inlist(dpstoff5, 4000062, 4000068)
Pass 45 through the data...
  smallest count = 4 in the cell      2053044
  Invoking rule 47:53044=63047
  replace dpstoff5 = 2063047 if inlist(dpstoff5, 2000047, 2053044)
Pass 46 through the data...
  smallest count = 4 in the cell      3000036
  Invoking rule 36:33944=43644
  replace dpstoff5 = 3043644 if inlist(dpstoff5, 3000036, 3033944)
Pass 47 through the data...
  smallest count = 4 in the cell      4031826
  Invoking rule 40111:31826=70126
  replace dpstoff5 = 4070126 if inlist(dpstoff5, 4040111, 4031826)
Pass 48 through the data...
  smallest count = 5 in the cell      1000060
  Invoking rule 55:60=25560
  replace dpstoff5 = 1025560 if inlist(dpstoff5, 1000055, 1000060)
Pass 49 through the data...
  smallest count = 5 in the cell      1024950
  Invoking rule 53:24950=34953
  replace dpstoff5 = 1034953 if inlist(dpstoff5, 1000053, 1024950)
Pass 50 through the data...
  smallest count = 5 in the cell      2074968
  Invoking rule 63047:74968=133068
  replace dpstoff5 = 2133068 if inlist(dpstoff5, 2063047, 2074968)
Pass 51 through the data...
  smallest count = 5 in the cell      3025560
  Invoking rule 34953:25560=54960
  replace dpstoff5 = 3054960 if inlist(dpstoff5, 3034953, 3025560)
Pass 52 through the data...
  smallest count = 6 in the cell      2070126
  Invoking rule 70126:133068=200168
  replace dpstoff5 = 2200168 if inlist(dpstoff5, 2070126, 2133068)
Pass 53 through the data...
  smallest count = 6 in the cell      4054960
  Invoking rule 53647:54960=103660
  replace dpstoff5 = 4103660 if inlist(dpstoff5, 4053647, 4054960)
Pass 54 through the data...
  smallest count = 9 in the cell      4026268
  Invoking rule 103660:26268=123668
  replace dpstoff5 = 4123668 if inlist(dpstoff5, 4103660, 4026268)
Pass 55 through the data...
  smallest count = 10 in the cell      3022426
  Invoking rule 18:22426=31826
  replace dpstoff5 = 3031826 if inlist(dpstoff5, 3000018, 3022426)
Pass 56 through the data...
  smallest count = 10 in the cell      3043644
  Invoking rule 30:43644=53044
  replace dpstoff5 = 3053044 if inlist(dpstoff5, 3000030, 3043644)
Pass 57 through the data...
  smallest count = 10 in the cell      4070126
  Invoking rule 30:70126=80130
  replace dpstoff5 = 4080130 if inlist(dpstoff5, 4000030, 4070126)
Pass 58 through the data...
  smallest count = 11 in the cell      1025560
  Invoking rule 34953:25560=54960
  replace dpstoff5 = 1054960 if inlist(dpstoff5, 1034953, 1025560)
Pass 59 through the data...
  smallest count = 11 in the cell      5073650
  Invoking rule 80130:73650=150150
  replace dpstoff5 = 5150150 if inlist(dpstoff5, 5080130, 5073650)
Pass 60 through the data...
  smallest count = 12 in the cell      1020208
  Invoking rule 11:20208=30211
  replace dpstoff5 = 1030211 if inlist(dpstoff5, 1000011, 1020208)
Pass 61 through the data...
  smallest count = 12 in the cell      1033640
  Invoking rule 44:33640=43644
  replace dpstoff5 = 1043644 if inlist(dpstoff5, 1000044, 1033640)
Pass 62 through the data...
  smallest count = 15 in the cell      1000024
  Invoking rule 24:26=22426
  replace dpstoff5 = 1022426 if inlist(dpstoff5, 1000024, 1000026)
Pass 63 through the data...
  smallest count = 15 in the cell      3040111
  Invoking rule 40111:31826=70126
  replace dpstoff5 = 3070126 if inlist(dpstoff5, 3040111, 3031826)
Pass 64 through the data...
  smallest count = 15 in the cell      3054960
  Invoking rule 62:54960=64962
  replace dpstoff5 = 3064962 if inlist(dpstoff5, 3000062, 3054960)
Pass 65 through the data...
  smallest count = 18 in the cell      5074968
  Invoking rule 69:74968=84969
  replace dpstoff5 = 5084969 if inlist(dpstoff5, 5000069, 5074968)
Pass 66 through the data...
  smallest count = 21 in the cell      2200168
  Done collapsing! Exiting...
{\smallskip}
. assert "`r(failed)'" == ""      
{\smallskip}
\nullskip
\end{stlog}

Checking on the last improvement, we see how the manual resolution succeeded:

\begin{stlog}
. tab alight_id dpstoff5 if daypart == 5
{\smallskip}
                      {\VBAR}     Interactions of daypart
                      {\VBAR} alight_id, with some collapsing
            alight_id {\VBAR}   5000069    5094468    5110140 {\VBAR}     Total
\HLI{22}{\PLUS}\HLI{33}{\PLUS}\HLI{10}
         8. Carmenton {\VBAR}         0          0          1 {\VBAR}         1 
         11. Dogville {\VBAR}         0          0          5 {\VBAR}         5 
         18. East End {\VBAR}         0          0          4 {\VBAR}         4 
       24. Framington {\VBAR}         0          0          2 {\VBAR}         2 
   26. Grand Junction {\VBAR}         0          0          1 {\VBAR}         1 
       30. High Point {\VBAR}         0          0          8 {\VBAR}         8 
       36. Irvingtown {\VBAR}         0          0          2 {\VBAR}         2 
         44. Limerick {\VBAR}         0          2          0 {\VBAR}         2 
      47. Moscow City {\VBAR}         0          6          0 {\VBAR}         6 
     50. Ontario Lake {\VBAR}         0          1          0 {\VBAR}         1 
 53. Picadilly Square {\VBAR}         0          2          0 {\VBAR}         2 
    62. Silver Spring {\VBAR}         0          9          0 {\VBAR}         9 
      68. Toledo Town {\VBAR}         0          7          0 {\VBAR}         7 
    69. Union Station {\VBAR}       123          0          0 {\VBAR}       123 
\HLI{22}{\PLUS}\HLI{33}{\PLUS}\HLI{10}
                Total {\VBAR}       123         27         23 {\VBAR}       173 
{\smallskip}
{\smallskip}
\nullskip
\end{stlog}

The resulting do-files can now be applied to producing control totals, and eventually to raking:

\begin{stlog}
. use trip_population, clear
{\smallskip}
. run dpston5.do
{\smallskip}
. total num_pass , over(dpston5)
{\smallskip}
Total estimation                  Number of obs   =        719
{\smallskip}
      1000001: dpston5 = 1000001
      1000002: dpston5 = 1000002
      1000008: dpston5 = 1000008
      1000011: dpston5 = 1000011
      1000018: dpston5 = 1000018
      1000024: dpston5 = 1000024
      1000026: dpston5 = 1000026
      1000030: dpston5 = 1000030
      1000036: dpston5 = 1000036
      1000044: dpston5 = 1000044
      1000047: dpston5 = 1000047
      1000062: dpston5 = 1000062
      1000068: dpston5 = 1000068
      1023940: dpston5 = 1023940
      1025560: dpston5 = 1025560
      1034953: dpston5 = 1034953
      2000047: dpston5 = 2000047
      2031826: dpston5 = 2031826
      2040111: dpston5 = 2040111
      2053044: dpston5 = 2053044
      2074968: dpston5 = 2074968
      3000008: dpston5 = 3000008
      3000011: dpston5 = 3000011
      3000018: dpston5 = 3000018
      3000024: dpston5 = 3000024
      3000026: dpston5 = 3000026
      3000030: dpston5 = 3000030
      3000047: dpston5 = 3000047
      3000068: dpston5 = 3000068
      3020102: dpston5 = 3020102
      3034953: dpston5 = 3034953
      3035562: dpston5 = 3035562
      3043644: dpston5 = 3043644
      4030108: dpston5 = 4030108
      4041126: dpston5 = 4041126
      4074968: dpston5 = 4074968
      4083050: dpston5 = 4083050
      5000011: dpston5 = 5000011
      5026268: dpston5 = 5026268
      5030108: dpston5 = 5030108
      5051836: dpston5 = 5051836
      5093960: dpston5 = 5093960
{\smallskip}
\HLI{13}{\TOPT}\HLI{48}
        Over {\VBAR}      Total   Std. Err.     [95\% Conf. Interval]
\HLI{13}{\PLUS}\HLI{48}
num_pass     {\VBAR}
     1000001 {\VBAR}       1423   967.7508     -476.9595    3322.959
     1000002 {\VBAR}       7198    4895.91     -2414.011    16810.01
     1000008 {\VBAR}      19254   13675.81     -7595.347    46103.35
     1000011 {\VBAR}      12626   9682.022     -6382.456    31634.46
     1000018 {\VBAR}       2470   1943.224     -1345.081    6285.081
     1000024 {\VBAR}        634   509.3549     -366.0031    1634.003
     1000026 {\VBAR}       2208   1774.996     -1276.802    5692.802
     1000030 {\VBAR}       4319   3665.427     -2877.235    11515.24
     1000036 {\VBAR}       1221   1046.817     -834.1873    3276.187
     1000044 {\VBAR}       1021    881.426     -709.4802     2751.48
     1000047 {\VBAR}       3300   2970.321     -2531.552    9131.552
     1000062 {\VBAR}       3402       3176     -2833.357    9637.357
     1000068 {\VBAR}       5085          .             .           .
     1023940 {\VBAR}        491    348.709     -193.6112    1175.611
     1025560 {\VBAR}        601     350.65     -87.42178    1289.422
     1034953 {\VBAR}       1286   774.5361     -234.6262    2806.626
     2000047 {\VBAR}        776   701.0001     -600.2549    2152.255
     2031826 {\VBAR}        426   223.0489     -11.90601     863.906
     2040111 {\VBAR}       1385   714.1628     -17.09692    2787.097
     2053044 {\VBAR}        527   364.2392     -188.1011    1242.101
     2074968 {\VBAR}        443   221.0935      8.933009     877.067
     3000008 {\VBAR}       3739   2665.175     -1493.467    8971.467
     3000011 {\VBAR}       3476   2669.777     -1765.503    8717.503
     3000018 {\VBAR}       1263    997.019     -694.4209    3220.421
     3000024 {\VBAR}       1296   1032.175     -730.4418    3322.442
     3000026 {\VBAR}        439   357.3421     -262.5603     1140.56
     3000030 {\VBAR}       3740   3175.677     -2494.723    9974.723
     3000047 {\VBAR}        984   888.5095     -760.3871    2728.387
     3000068 {\VBAR}        744          .             .           .
     3020102 {\VBAR}        992   553.0017     -93.69354    2077.694
     3034953 {\VBAR}        893   588.1798     -261.7577    2047.758
     3035562 {\VBAR}       1187   894.6009     -569.3461    2943.346
     3043644 {\VBAR}        713   398.6235     -69.60702    1495.607
     4030108 {\VBAR}       1154   529.0201      115.3888    2192.611
     4041126 {\VBAR}       1135   530.8674      92.76204    2177.238
     4074968 {\VBAR}        659   307.4955      55.30219    1262.698
     4083050 {\VBAR}        741   395.9312     -36.32126    1518.321
     5000011 {\VBAR}       1270    834.301      -367.961    2907.961
     5026268 {\VBAR}        557   364.4324     -158.4805    1272.481
     5030108 {\VBAR}        610   263.2061      93.25444    1126.746
     5051836 {\VBAR}        622   215.5712      198.7749    1045.225
     5093960 {\VBAR}        473   261.8954     -41.17225    987.1723
\HLI{13}{\BOTT}\HLI{48}
{\smallskip}
. matrix dpston5 = e(b)
{\smallskip}
. matrix coleq dpston5 = _one
{\smallskip}
. matrix rownames dpston5 = dpston5
{\smallskip}
. run dpstoff5.do
{\smallskip}
. total num_pass , over(dpstoff5)
{\smallskip}
Total estimation                  Number of obs   =        719
{\smallskip}
      1000018: dpstoff5 = 1000018
      1000030: dpstoff5 = 1000030
      1000047: dpstoff5 = 1000047
      1000062: dpstoff5 = 1000062
      1000068: dpstoff5 = 1000068
      1000069: dpstoff5 = 1000069
      1022426: dpstoff5 = 1022426
      1030211: dpstoff5 = 1030211
      1043644: dpstoff5 = 1043644
      1054960: dpstoff5 = 1054960
      2000069: dpstoff5 = 2000069
      2200168: dpstoff5 = 2200168
      3000047: dpstoff5 = 3000047
      3000068: dpstoff5 = 3000068
      3000069: dpstoff5 = 3000069
      3053044: dpstoff5 = 3053044
      3064962: dpstoff5 = 3064962
      3070126: dpstoff5 = 3070126
      4000069: dpstoff5 = 4000069
      4080130: dpstoff5 = 4080130
      4123668: dpstoff5 = 4123668
      5084969: dpstoff5 = 5084969
      5150150: dpstoff5 = 5150150
{\smallskip}
\HLI{13}{\TOPT}\HLI{48}
        Over {\VBAR}      Total   Std. Err.     [95\% Conf. Interval]
\HLI{13}{\PLUS}\HLI{48}
num_pass     {\VBAR}
     1000018 {\VBAR}        929   360.7303      220.7878    1637.212
     1000030 {\VBAR}       2189   868.0319      484.8161    3893.184
     1000047 {\VBAR}       1746   630.7528      507.6598     2984.34
     1000062 {\VBAR}       1134   382.7765       382.505    1885.495
     1000068 {\VBAR}       1372   426.3969      534.8662    2209.134
     1000069 {\VBAR}      53193   15995.88      21788.72    84597.28
     1022426 {\VBAR}        980    273.542      442.9623    1517.038
     1030211 {\VBAR}       2986   1614.166     -183.0484    6155.048
     1043644 {\VBAR}        737   159.5597      423.7407    1050.259
     1054960 {\VBAR}       1273   259.8915      762.7619    1783.238
     2000069 {\VBAR}       3038   938.8099      1194.859    4881.141
     2200168 {\VBAR}        519   71.80393      378.0292    659.9708
     3000047 {\VBAR}        556   187.4945      187.8971    924.1029
     3000068 {\VBAR}        444   126.0503      196.5289    691.4711
     3000069 {\VBAR}      16007   4295.998      7572.781    24441.22
     3053044 {\VBAR}        787    249.935      296.3092    1277.691
     3064962 {\VBAR}        759   141.1029      481.9765    1036.023
     3070126 {\VBAR}        913   335.9457      253.4468    1572.553
     4000069 {\VBAR}       2733   728.6906      1302.381    4163.619
     4080130 {\VBAR}        480   132.6806      219.5117    740.4883
     4123668 {\VBAR}        476   72.94794      332.7832    619.2168
     5084969 {\VBAR}       2945   999.9897      981.7468    4908.253
     5150150 {\VBAR}        587   137.3569      317.3308    856.6692
\HLI{13}{\BOTT}\HLI{48}
{\smallskip}
. matrix dpstoff5 = e(b)
{\smallskip}
. matrix coleq dpstoff5 = _one
{\smallskip}
. matrix rownames dpstoff5 = dpstoff5
{\smallskip}
. use trip_sample, clear
{\smallskip}
. run dpston5
{\smallskip}
. run dpstoff5
{\smallskip}
. gen byte _one = 1       
{\smallskip}
. ipfraking [pw=_one], ctotal(dpston5 dpstoff5) gen(raked_weight5)
{\smallskip}
 Iteration 1, max rel difference of raked weights = 37.856256
 Iteration 2, max rel difference of raked weights = .0250943
 Iteration 3, max rel difference of raked weights = .00252004
 Iteration 4, max rel difference of raked weights = .00030004
 Iteration 5, max rel difference of raked weights = .00003571
 Iteration 6, max rel difference of raked weights = 4.250e-06
 Iteration 7, max rel difference of raked weights = 5.058e-07
The worst relative discrepancy of  5.6e-08 is observed for dpstoff5 == 5150150     
Target value =        587; achieved value =        587
{\smallskip}
   Summary of the weight changes
{\smallskip}
              {\VBAR}    Mean    Std. dev.    Min        Max       CV
\HLI{14}{\PLUS}\HLI{50} 
Orig weights  {\VBAR}        1          0         1           1       0
Raked weights {\VBAR}   26.487       5.74    14.593      38.634   .2167
Adjust factor {\VBAR}  26.4869              14.5933     38.6339
{\smallskip}
. whatsdeff raked_weight5
{\smallskip}
    Group     {\VBAR}   Min     {\VBAR}   Mean    {\VBAR}   Max     {\VBAR}    CV   {\VBAR}   DEFF  {\VBAR}   N   {\VBAR}  N eff
\HLI{14}{\PLUS}\HLI{11}{\PLUS}\HLI{11}{\PLUS}\HLI{11}{\PLUS}\HLI{9}{\PLUS}\HLI{9}{\PLUS}\HLI{7}{\PLUS}\HLI{8}
      Overall {\VBAR}     14.59 {\VBAR}     26.49 {\VBAR}     38.63 {\VBAR}  0.2167 {\VBAR}  1.0470 {\VBAR}  3654 {\VBAR} 3490.13
{\smallskip}
\nullskip
\end{stlog}

\subsubsection{Informative labels}
\label{subsec:wgtcellcollapse:labels}

As the final touch, let us consider the variety of labels that can be attached to the resulting
collapsed cells.

\begin{stlog}
. wgtcellcollapse label, var(dpston5) 
(language default renamed unlabeled_ccells)
(language numbered_ccells now current language)
(language texted_ccells now current language)
{\smallskip}
To attach the numeric labels (of the kind "dpston5==1000001"), type:
   label language numbered_ccells
To attach the text labels (of the kind "dpston5==AM Peak; 1. Alewife"), type:
   label language texted_ccells
The original state, which is also the current state, is:
   label language unlabeled_ccells
{\smallskip}
{\smallskip}
. wgtcellcollapse label, var(dpstoff5) 
{\smallskip}
To attach the numeric labels (of the kind "dpstoff5==1000018"), type:
   label language numbered_ccells
To attach the text labels (of the kind "dpstoff5==AM Peak; 18. East End"), type:
   label language texted_ccells
The original state, which is also the current state, is:
   label language unlabeled_ccells
{\smallskip}
{\smallskip}
. label language numbered_ccells
{\smallskip}
. tab dpstoff5 if daypart==5
{\smallskip}
   Long ID of the interaction {\VBAR}      Freq.     Percent        Cum.
\HLI{30}{\PLUS}\HLI{35}
    daypart==5, alight_id==69 {\VBAR}        123       71.10       71.10
 daypart==5, alight_id==94468 {\VBAR}         27       15.61       86.71
daypart==5, alight_id==110140 {\VBAR}         23       13.29      100.00
\HLI{30}{\PLUS}\HLI{35}
                        Total {\VBAR}        173      100.00
{\smallskip}
. label language texted_ccells
{\smallskip}
. tab dpstoff5 if daypart==5
{\smallskip}
             Long ID of the interaction {\VBAR}      Freq.     Percent        Cum.
\HLI{40}{\PLUS}\HLI{35}
             Weekend; 69. Union Station {\VBAR}        123       71.10       71.10
Weekend; 44. Limerick to 68. Toledo Tow {\VBAR}         27       15.61       86.71
 Weekend; 1. Alewife to 40. King Street {\VBAR}         23       13.29      100.00
\HLI{40}{\PLUS}\HLI{35}
                                  Total {\VBAR}        173      100.00
{\smallskip}
. label language unlabeled_ccells
{\smallskip}
. tab dpstoff5 if daypart==5
{\smallskip}
Interaction {\VBAR}
       s of {\VBAR}
    daypart {\VBAR}
 alight_id, {\VBAR}
  with some {\VBAR}
 collapsing {\VBAR}      Freq.     Percent        Cum.
\HLI{12}{\PLUS}\HLI{35}
    5000069 {\VBAR}        123       71.10       71.10
    5094468 {\VBAR}         27       15.61       86.71
    5110140 {\VBAR}         23       13.29      100.00
\HLI{12}{\PLUS}\HLI{35}
      Total {\VBAR}        173      100.00
{\smallskip}
\nullskip
\end{stlog}

Using the mechanics of labels in multiple languages, \stcmd{wgtcellcollapse label} defines three
``languages'' to describe the cells. The language \stcmd{numbered\_ccells} may be convenient
for debugging purposes in fine-tuning the collapsing algorithms, while the language
\stcmd{texted\_ccells} would prove useful for \stcmd{ipfraking\_report} in creating human-readable
labels.

\section{Linear calibrated weights}
\label{subsec:linear}

Using the existing example, let me demonstrate the linear calibration option of 
\stcmd{ipfraking}.

\begin{stlog}
. cap drop raked_weight5*
{\smallskip}
. set rmsg on
r; t=0.00 12:12:08
{\smallskip}
. ipfraking [pw=_one], ctotal(dpston5 dpstoff5) nograph gen(raked_weight5)
{\smallskip}
 Iteration 1, max rel difference of raked weights = 37.856256
 Iteration 2, max rel difference of raked weights = .0250943
 Iteration 3, max rel difference of raked weights = .00252004
 Iteration 4, max rel difference of raked weights = .00030004
 Iteration 5, max rel difference of raked weights = .00003571
 Iteration 6, max rel difference of raked weights = 4.250e-06
 Iteration 7, max rel difference of raked weights = 5.058e-07
The worst relative discrepancy of  5.6e-08 is observed for dpstoff5 == 5150150     
Target value =        587; achieved value =        587
{\smallskip}
   Summary of the weight changes
{\smallskip}
              {\VBAR}    Mean    Std. dev.    Min        Max       CV
\HLI{14}{\PLUS}\HLI{50} 
Orig weights  {\VBAR}        1          0         1           1       0
Raked weights {\VBAR}   26.487       5.74    14.593      38.634   .2167
Adjust factor {\VBAR}  26.4869              14.5933     38.6339
r; t=1.100 12:12:10
{\smallskip}
. ipfraking [pw=_one], ctotal(dpston5 dpstoff5) nograph gen(raked_weight5l) linear
{\smallskip}
Linear calibration
The worst relative discrepancy of  2.0e-14 is observed for dpstoff5 == 5150150     
Target value =        587; achieved value =        587
{\smallskip}
   Summary of the weight changes
{\smallskip}
              {\VBAR}    Mean    Std. dev.    Min        Max       CV
\HLI{14}{\PLUS}\HLI{50} 
Orig weights  {\VBAR}        1          0         1           1       0
Raked weights {\VBAR}   26.487     5.7387    12.875      38.204   .2167
Adjust factor {\VBAR}  26.4869              12.8752     38.2040
r; t=0.78 12:12:11
{\smallskip}
. set rmsg off
{\smallskip}
. label variable raked_weight5l "Linear calibrated weights"
{\smallskip}
. compare raked_weight5 raked_weight5l
{\smallskip}
                                        \HLI{10} difference \HLI{10}
                            count       minimum      average     maximum
\HLI{72}
raked_w{\tytilde}5<raked_{\tytilde}5l          1871     -1.813144    -.0408154   -5.84e-10
raked_w{\tytilde}5>raked_{\tytilde}5l          1783      2.75e-08     .0428298    2.405758
                       \HLI{10}
jointly defined              3654     -1.813144     1.20e-10    2.405758
                       \HLI{10}
total                        3654
{\smallskip}
\nullskip
\end{stlog}

\begin{figure}[h!]
    \begin{center}
    \epsfig[scale=0.2]{file=raked_linear}
    \end{center}
    \caption{Linear and raked weights}
    \label{fig:linear:raked}
\end{figure}

The speed advantages of \stcmd{linear} calibration are quite clear, even though
convergence of raking in 8 iterations is lighting-fast, in author's experience. 
The weights are very similar to one another, with the lowest of the linearly
calibrated weights being slightly smaller than comparable raked weights.
As mentioned before, in the extreme situations, linearly calibrated weights
may become negative.


















\section*{Acknowledgements}

The author is grateful to Jason Brinkley and Tom Guterbock for bug reports and 
functionality suggestions.
The opinions stated in this paper
are of the author only, and do not represent the position of Abt Associates.

\bibliographystyle{sj}
\bibliography{everything}
% \bibliography{ipfraking}

\begin{aboutauthor}
  Stanislav (Stas) Kolenikov is a Senior Scientist at Abt Associates.
  His work involves applications of statistical methods in data collection
  for public opinion research, public health, transportation, and other disciplines
  that utilize collection of survey data.
  Within survey methodology, his expertise includes advanced sampling techniques,
  survey weighting, calibration, missing data imputation, variance estimation,
  nonresponse analysis and adjustment, small area estimation, and mode effects.
  Besides survey statistics, Stas has extensive experience developing and applying
  statistical methods in social sciences, with focus on structural equation
  modeling and microeconometrics. He has been writing Stata programs since
  1998 when Stata was version 5.
\end{aboutauthor}
